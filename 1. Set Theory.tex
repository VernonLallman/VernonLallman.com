% Document Type Management
    \documentclass{book}
    \usepackage[utf8]{inputenc}
    \usepackage[english]{babel}
     
 
%  Packages for Bibiography Management
    \usepackage{biblatex}
    \usepackage{csquotes}
    \addbibresource{references.bib}


% AMS Packages
    \usepackage{amsfonts}
    \usepackage{amssymb}
    \usepackage{amsmath}
    \usepackage{amsthm}
    \usepackage{esvect}
    \usepackage{blindtext}


% Images Management Packages
    \usepackage{graphicx}
    \graphicspath{ {images/} }


% Packages to Draw Images
    \usepackage{float}
    \usepackage{tikz}
    \usetikzlibrary{shapes,backgrounds}
    \usetikzlibrary{positioning}


% Custom Section Labels   
    \newtheorem{theorem}{Theorem}[section]
        \newtheorem{conjecture}[theorem]{Conjecture}
        \newtheorem{property}{Property}[theorem]
        \newtheorem{corollary}{Corollary}[theorem]
        \newtheorem{lemma}[theorem]{Lemma}

    \theoremstyle{definition}
        \newtheorem{definition}{Definition}[section]
        \newtheorem{example}{Example}[definition]
        \newtheorem{entry}{Entry}[definition]
 
    \theoremstyle{remark}
        \newtheorem{remark}{Remark}

% Working Environment
    \newenvironment{working}{\textit{Workings. }}


% Custom Chapters and Sections
    \usepackage[explicit]{titlesec}
    
    \usepackage[many]{tcolorbox}
    \tcbset{colback=green!10!white}
    \tcbsetforeverylayer{colframe=green!10!white}
    
    \usepackage{fancyhdr}
    \usepackage{lmodern}
    \usepackage{lipsum}
    
    \definecolor{titlebgdark}{RGB}{0,163,243}
    \definecolor{titlebglight}{RGB}{191,233,251}
    
    \newlength{\chaptertopspacing}
    \setlength{\chaptertopspacing}{130pt}
    
    
    \titleformat{\chapter}[display]
      {\normalfont\huge\bfseries}
      {}
      {-\chaptertopspacing}
      {%
        \begin{tcolorbox}[
          enhanced,
          colback=titlebgdark,
          boxrule=0.25cm,
          colframe=titlebglight,
          arc=0pt,
          outer arc=0pt,
          leftrule=0pt,
          rightrule=0pt,
          fontupper=\color{white}\sffamily\bfseries\huge,
          enlarge left by=-1in-\hoffset-\oddsidemargin, 
          enlarge right by=-\paperwidth+1in+\hoffset+\oddsidemargin+\textwidth,
          width=\paperwidth, 
          left=1in+\hoffset+\oddsidemargin, 
          right=\paperwidth-1in-\hoffset-\oddsidemargin-\textwidth,
          top=0.6cm, 
          bottom=0.6cm,
          overlay={
            \node[
              fill=titlebgdark,
              draw=titlebglight,
              line width=0.15cm,
              inner sep=0pt,
              text width=1.7cm,
              minimum height=1.7cm,
              align=center,
              font=\color{white}\sffamily\bfseries\fontsize{30}{36}\selectfont
            ] 
            (chapname)
            at ([xshift=-4cm]frame.north east)
            {\thechapter};
            \node[font=\small,anchor=south,inner sep=2pt] at (chapname.north)
            {\MakeUppercase\chaptertitlename};  
          } 
        ]
        #1
        \end{tcolorbox}%
      }
    \titleformat{name=\chapter,numberless}[display]
      {\normalfont\huge\bfseries}
      {}
      {-\chaptertopspacing}
      {%
        \begin{tcolorbox}[
          enhanced,
          colback=titlebgdark,
          boxrule=0.25cm,
          colframe=titlebglight,
          arc=0pt,
          outer arc=0pt,
          remember as=title,
          leftrule=0pt,
          rightrule=0pt,
          fontupper=\color{white}\sffamily\bfseries\huge,
          enlarge left by=-1in-\hoffset-\oddsidemargin, 
          enlarge right by=-\paperwidth+1in+\hoffset+\oddsidemargin+\textwidth,
          width=\paperwidth, 
          left=1in+\hoffset+\oddsidemargin, 
          right=\paperwidth-1in-\hoffset-\oddsidemargin-\textwidth,
          top=0.6cm, 
          bottom=0.6cm, 
        ]
        #1
        \end{tcolorbox}%
      }
    \titlespacing*{\chapter}
      {0pt}{0pt}{40pt}
    \makeatother


% Custom Commands '
    \newcommand{\bb}[1]{\mathbb{#1}}
    \newcommand{\cc}[1]{\mathcal{#1}}
    \newcommand{\ovec}{\big \langle}
    \newcommand{\cvec}{\big \rangle}
    \newcommand{\m}{\cdot}



\begin{document}

\begin{titlepage}
    \begin{center}
        \vspace*{1cm}
        
        \textbf{MATHEMATICAL PROOF STRUCTURES}
        
        \vspace{0.5cm}
        Module 1: Logic and Set Theory
        
        \vspace{1.5cm}
        
        \textbf{Vernon V. Lallman}
        
        \vfill
        
%        A thesis presented for the degree of\\
%        Doctor of Philosophy
        
        \vspace{0.8cm}
        
        \includegraphics[width=0.4\textwidth]{university}
        
        Mathematics Department\\
        State University of New York \\
        Geneseo\\
        \date{\today}
        
    \end{center}
\end{titlepage}

\tableofcontents


\newpage
\chapter{TOPICS IN LOGIC}
\section{Statements and Conditional Statements}

\begin{definition}
Statement \\

A statement is a declarative sentence that is either true or false but not both. \\
\end{definition}

\begin{definition}
Conditional Statements

A Conditional Statement is a statement that can be written in the form {\bf If $P$, then $Q$}, where $P$ and $Q$ are sentences. For this conditional statement, $P$ is called the {\bf hypothesis} and $Q$ is called the conclusion. \\

\begin{minipage}{1\textwidth}
1. If $P$ is true and $Q$ is true, then $P$ implies $Q$ is a true \\
2. If $P$ is true and $Q$ is false, then $P$ implies $Q$ is a false \\
3. If $P$ is false and $Q$ is true, then $P$ implies $Q$ is a true \\
4. If $P$ is false and $Q$ is false, then $P$ implies $Q$ is a true \\
\end{minipage}
\end{definition}

\begin{definition}
Even and Odd Integers \\

An integer $a$ is an {\bf even integer} provided tha there exisits an integer $n$ such that $a = 2n$. An integer $a$ is an {\bf odd integer} provided there exists an integer $n$ such that $a = 2n + 1$
\end{definition}


\begin{definition}
Closure Properties of Number Systems

Three of the basic properties of the integers are that that the set $\bb{Z}$ is a {\bf closed under addition}, the set $\bb{Z}$ is {\bf closed under multiplication}, and the set of integers is {\bf closed under subtraction}. This means that: \\

$\bb{Z}$ Closed Under Addition: If $x$ and $y$ are integers, then $x+y$ is an integer. \\
 
$\bb{Z}$ Closed Under Multiplication: If $x$ and $y$ are integers, then $x*y$ is an integer. \\

$\bb{Z}$ Closed Under Subtraction: If $x$ and $y$ are integers, then $x+y$ is an integer. \\

\end{definition}

\newpage
\section{Mathematical Proof Structures}

\begin{definition}
Mathematical Definition

A mathematical definition is an agreement that a partiular word or phrase will stand for some object, property, or other concept that we expect to refer to often. In many elementary proofd, the answer to the question: {\bf How do we prove a certain proposition}, is often answered by means of a definition. 
\end{definition}


\begin{definition}
Mathematical Proof \\

A mathematical proof is a convincing argument (within the accepted standards of the mathematical community) that a certain mathematical statement is necessarily true. A proof generally uses deductive reasoning and logic but also contains some amount of ordinary language. Here are some general guidelines for writing proofs: \\
1. Begin with a carefully worded statement of the theorem or result to be proven \\
2. Begin the proof with a statement of your assumptions \\
3. Use the pronoun "we" \\
4. Use italcs for variables when using a word processor \\
5. Display important equations and mathematical expressions \\
6. Tell the reader when the proof has been completed \\
\end{definition}







\newpage
\section{Logical Operators}

1. The {\bf conjunction} of statements $P$ and $Q$ is the statement $P \wedge Q$. The Truth Table is as follows: 

A logical operator (or connective) on mathematical statements is a word or combination of words that combines one or more mathematical statements into a compound (mathematical) statement. The logical operators are: \\

\begin{definition}
Conjunction \\

The {\bf conjunction} of statements with variables $P$ and $Q$ is the statement $P \wedge Q$. The truth table is as follows: \\
\begin{center}
\begin{tabular}{|c|c|c|}
\hline 
P & Q & $P \wedge Q$ \\ 
\hline 
T & T & T \\ 
\hline 
T & F & F \\ 
\hline 
F & T & F \\ 
\hline 
F & F & F \\ 
\hline 
\end{tabular} 
\end{center}
\end{definition}


\begin{definition}
Inclusive Or \\

The {\bf inclusive or} of statements with variables $P$ and $Q$ is the statement $P \vee Q$. The truth table is as follows: \\
\begin{center}
\begin{tabular}{|c|c|c|}
\hline 
P & Q & $P \vee Q$ \\ 
\hline 
T & T & T \\ 
\hline 
T & F & T \\ 
\hline 
F & T & T \\ 
\hline 
F & F & F \\ 
\hline 
\end{tabular} 
\end{center}
\end{definition}

\begin{definition}
Exclusive Or

The {\bf exclusive or} of statements with variables $P$ and $Q$ is the statement $P \oplus Q$. The truth table is as follows: \\
\begin{center}
\begin{tabular}{|c|c|c|}
\hline 
P & Q & $P \oplus Q$ \\ 
\hline 
T & T & F \\ 
\hline 
T & F & T \\ 
\hline 
F & T & T \\ 
\hline 
F & F & F \\ 
\hline 
\end{tabular} 
\end{center}
\end{definition}


\begin{definition}
Negation \\

The {\bf negation} of statements with variable $P$ is the statement $\neg P$. The Truth Table is as follows: \\
\begin{center}
\begin{tabular}{|c|c|c|}
\hline 
P & $\neg P$ \\ 
\hline 
T & F \\ 
\hline 
F & T \\ 
\hline 
\end{tabular} 
\end{center}
\end{definition}


\begin{definition}
Implication \\

The {\bf implication} of statements with variables $P$ and $Q$ is the statement $P \to Q$. The Truth Table is as follows: \\
\begin{center}
\begin{tabular}{|c|c|c|}
\hline 
P & Q & $P \to Q$ \\ 
\hline 
T & T & T \\ 
\hline 
T & F & F \\ 
\hline 
F & T & T \\ 
\hline 
F & F & T \\ 
\hline 
\end{tabular} 
\end{center}
\end{definition}

\begin{definition}
Converse Statement \\

The {\bf converse} of the conditional statement $P \to Q$ is the conditional statement $Q \to P$
\end{definition}

\begin{definition}
Contrapositive Statement \\

The {\bf contrapositive} of the conditional statement $P \to Q$ is the conditional statement $\neg Q \to \neg P$
\end{definition}


\begin{definition}
Concept of Necessary Condition \\
A condition $P$ is said to be necessary for a condition $Q$, if (and only if) the falsity (/nonexistence /non-occurrence) [as the case may be] of $P$ guarantees (or brings about) the falsity (/nonexistence /non-occurrence) of $Q$. \\

\begin{center}
\begin{tabular}{|c|c|c|}
\hline 
P & Q & $P Necessary Q$ \\ 
\hline 
T & T & T \\ 
\hline 
T & F & T \\ 
\hline 
F & T & F \\ 
\hline 
F & F & T \\ 
\hline 
\end{tabular} 
\end{center}

\end{definition}

\begin{definition}
Concept of Sufficient Condition \\
A condition $P$ is said to be sufficient for a condition $Q$, if (and only if) the truth (/existence /occurrence) [as the case may be] of $P$ guarantees (or brings about) the truth (/existence /occurrence) of $Q$.

\begin{center}
\begin{tabular}{|c|c|c|}
\hline 
P & Q & $P Sufficient Q$ \\ 
\hline 
T & T & T \\ 
\hline 
T & F & F \\ 
\hline 
F & T & T \\ 
\hline 
F & F & T \\ 
\hline 
\end{tabular} 
\end{center}
\end{definition}


\begin{definition}
Converse Statement of a Necessary or Sufficient Implication \\

The converse of the conditional statement {\bf $P$ is sufficient for $Q$} is the conditional statement {\bf $Q$ is necessary $P$}. While the converse of the conditional statement {\bf $Q$ is necessary for $P$} is the conditional statement {\bf $P$ is sufficient for $Q$}.
\end{definition}


\begin{definition}
Contrapositive Statement of a Necessary or Sufficient Implication \\

The contrapositive of the conditional statement {\bf $P$ is sufficient for $Q$} is the conditional statement {\bf $\neg Q$ is necessary for  $\neg P$}. While the contrapositive of the conditional statement {\bf $Q$ is necessary for $P$} is the conditional statement {\bf $\neg P$ is sufficient for $\neg Q$}. 
\end{definition}



\begin{definition}
Equivalence of Necessary and Sufficient Conditions \\
The conditional statement {\bf $P$ is sufficient for $Q$} is equivalent to the statement {\bf the condition $Q$ is necessary for $P$}. While, the statement {\bf $P$ is a necessary for $Q$} is equivalent to the statement {\bf $Q$ is sufficient for $P$}. \\
\end{definition}


%\begin{figure}
%	\wedgetion{Graph of inter-relationships between common 4-sided polygons}
%	\centering
%	\includegraphics{4-Sided-Polygon}
%\end{figure}

\newpage
\begin{tcolorbox}
\begin{definition}
Quadrilateral 

A polygon is a quadrilateral if and only if it's a plane figure with four straight sides and angles. 
\end{definition}


\begin{definition}
Kite 

A quadrilateral is a kite if and only if it has the following properties: \\
1. Pairs of sides are congruent by definition \\
2. Only One pair of diagonally opposite angles is equal
3. Only one diagonal bisect the other angle \\
4. The diagonals are perpendicular bisectors of each other \\
\end{definition}


\begin{definition}
Parallelogram 

A quadrilateral is a parallelogram if and only if it has the following properties: \\
1. Parallel sides
2. Opposite angles are congruent 
3. Consecutive angles are supplementary
\end{definition}


\begin{definition}
Rhombus 

A parallelogram is a rhombus if and only if it has the following properties: \\
1. All the properties of a parallelogram \\
2. All sides are congruent by definition \\
3. The diagonals bisect the angles \\
4. The diagonals are perpendicular bisectors of each other \\
\end{definition}


\begin{definition}
Rectangle 

A parallelogram is a rectangle if and only if it has the following properties: \\
1. All the properties of a parallelogram \\
2. All angles are right angles by definition \\
3. The diagonals are congruent \\
\end{definition}


\begin{definition}
Square

A parallelogram is a square if and only if it has the following properties: \\
1. All the properties of a rhombus \\
2. All the properties of a rectangle apply \\
\end{definition}
\end{tcolorbox}

\newpage
\begin{definition}
4-Sided Polygon Theorems Examples \\

Examples of converse statements with necessary conditions that are not sufficient on the left and sufficient conditions that are not necessary on the right: \\

1. {\bf A Rectangle and Rhombus are necessary conditions of a plane figure to be a Square} is the converse ofdd the statement {\bf A Square is a sufficient condition for a plane figure to be a either a Rectangle or a Rhombus} \\


2. {\bf A Parallelogram is a necessary condition for a plane figure to be a a Rectangle} is the converse of the statement {\bf A Rectangle is a sufficient condition for a plane figure to be a Parallelogram}\\

3. {\bf A Kite or a Parallelogram is a necessary condition for a plane figure to be Rhombus} is the converse of the statement {\bf A Rhombus is a sufficient condition for a plane figure to be either a Kite or Parallelogram}\\

4. {\bf A Quadrilateral is a necessary condition for a plane figure to be a Kite} is the converse of the statement {\bf A Kite is a sufficient condition for a plane figure to to be a Quadrilateral} \\

5. {\bf A Quadrilateral is a necessary condition for a plane figure to be a Parallelogram} is the converse of the statement {\bf A Parallelogram is a sufficient condition for a plane figure to be Quadrilateral} \\
\end{definition}

\newpage
\begin{definition}
Conjunctive and Disjunctive Statements involving Necessary and Sufficient Conditions \\

The relationship between the two conditions must be exactly one of the following four possibilities: \\
1. $nec \wedge suf$ CASE: {\bf $[P$ necessary $Q] \wedge [P$ sufficient $Q]$} \\
\begin{center}
\begin{tabular}{|c|c|c|c|c|}
\hline 
P & Q & P nec Q & P suf Q & $[P$ nec $Q] \wedge [P$ suf $Q]$  \\ 
\hline 
T & T & T & T & T\\ 
\hline 
T & F & T & F & F\\ 
\hline 
F & T & F & T & F\\ 
\hline 
F & F & T & T & T\\ 
\hline 
\end{tabular} 
\end{center}

2. $nec \vee suf$ CASE: {\bf $[P$ necessary $Q] \vee [P$ sufficient $Q]$} \\
\begin{center}
\begin{tabular}{|c|c|c|c|c|}
\hline 
P & Q & P nec Q & P suf Q & $[P$ nec $Q] \wedge [P$ suf $Q]$  \\ 
\hline 
T & T & T & T & T\\ 
\hline 
T & F & T & F & T\\ 
\hline 
F & T & F & T & T\\ 
\hline 
F & F & T & T & T\\ 
\hline 
\end{tabular} 
\end{center}

3. $\neg nec \wedge suf$ CASE: {\bf $\neg[P$ necessary $Q] \wedge [P$ sufficient $Q]$} \\
\begin{center}
\begin{tabular}{|c|c|c|c|c|}
\hline 
P & Q & P nec Q & P suf Q & $\neg[P$ nec $Q] \wedge [P$ suf $Q]$  \\ 
\hline 
T & T & T & T & F\\ 
\hline 
T & F & T & F & F\\ 
\hline 
F & T & F & T & T\\ 
\hline 
F & F & T & T & F\\ 
\hline 
\end{tabular} 
\end{center}


4. $nec \wedge \neg suf$ CASE: {\bf $[P$ necessary $Q] \wedge \neg[P$ sufficient $Q]$} \\
\begin{center}
\begin{tabular}{|c|c|c|c|c|}
\hline 
P & Q & P nec Q & P suf Q & $\neg[P$ nec $Q] \wedge \neg[P$ suf $Q]$  \\ 
\hline 
T & T & T & T & F\\ 
\hline 
T & F & T & F & T\\ 
\hline 
F & T & F & T & F\\ 
\hline 
F & F & T & T & F\\ 
\hline 
\end{tabular} 
\end{center}

5. $\neg nec \vee suf$ CASE: {\bf $\neg[P$ necessary $Q] \vee [P$ sufficient $Q]$} \\
\begin{center}
\begin{tabular}{|c|c|c|c|c|}
\hline 
P & Q & P nec Q & P suf Q & $\neg[P$ nec $Q] \vee [P$ suf $Q]$  \\ 
\hline 
T & T & T & T & T\\ 
\hline 
T & F & T & F & F\\ 
\hline 
F & T & F & T & T\\ 
\hline 
F & F & T & T & T\\ 
\hline 
\end{tabular} 
\end{center}


6. $nec \vee \neg suf$ CASE: {\bf $[P$ necessary $Q] \vee \neg[P$ sufficient $Q]$} \\
\begin{center}
\begin{tabular}{|c|c|c|c|c|}
\hline 
P & Q & P nec Q & P suf Q & $[P$ nec $Q] \vee \neg[P$ suf $Q]$  \\ 
\hline 
T & T & T & T & T\\ 
\hline 
T & F & T & F & T\\ 
\hline 
F & T & F & T & F\\ 
\hline 
F & F & T & T & T\\ 
\hline 
\end{tabular} 
\end{center}


7. $\neg(nec \wedge suf)$ CASE: {\bf $\neg([P$ necessary $Q] \wedge [P$ sufficient $Q])$} \\
\begin{center}
\begin{tabular}{|c|c|c|c|c|}
\hline 
P & Q & P nec Q & P suf Q & $\neg[P$ nec $Q] \vee \neg[P$ suf $Q]$  \\ 
\hline 
T & T & T & T & F\\ 
\hline 
T & F & T & F & T\\ 
\hline 
F & T & F & T & T\\ 
\hline 
F & F & T & T & F\\ 
\hline 
\end{tabular} 
\end{center}

8. $\neg(nec \vee suf)$ CASE: {\bf $\neg([P$ necessary $Q] \vee [P$ sufficient $Q])$} \\
\begin{center}
\begin{tabular}{|c|c|c|c|c|}
\hline 
P & Q & P nec Q & P suf Q & $\neg[P$ nec $Q] \wedge \neg[P$ suf $Q]$  \\ 
\hline 
T & T & T & T & F\\ 
\hline 
T & F & T & F & F\\ 
\hline 
F & T & F & T & F\\ 
\hline 
F & F & T & T & F\\ 
\hline 
\end{tabular} 
\end{center} 
\end{definition}

\begin{definition}
Synonyms of Implication

\begin{center}
\begin{tabular}{|c|c|}
\hline 
FORM 1 & FORM 2 \\ 
\hline 
If $P$, then $Q$ & $P$ if $Q$ \\ 
\hline 
$P$ implies $Q$ & $Q$ is necessary for $P$ \\ 
\hline 
$P$ only if $Q$ & $Q$ is true whenever $P$ is true \\ 
\hline 
$P$ is sufficient for $Q$ & • \\ 
\hline 
Whenever $P$ is true, Q is true & • \\ 
\hline 
$Q$ if $P$ & • \\ 
\hline 
\end{tabular} 
\end{center}
\end{definition}



\newpage
\section{Logically Equivalent Statements}
\begin{definition}
Logically Equivalent Statement \\

Two expressions are logically equivalent provided that they have the same truth value for all possible combinations of truth values for all variables appearing in the two expressions. In this case, we write $X \equiv Y$ and say that $X$ is logically equivalent to $Y$ \\
\end{definition}

\begin{tcolorbox}
\begin{theorem}
Logical Equivalent Biconditional Statement \\
For the statements $P$, $Q$, and $R$: \\
\begin{tabular}{|c|c|c|}
\hline 
Num & Theorem Name & Logically Equivalent Statements \\ 
\hline 
1 & De Morgan Law & $\neg(P \wedge Q) \equiv \neg P \vee \neg Q$ \\ 
\hline 
2 & De Morgan Law & $\neg(P \vee Q) \equiv \neg P \wedge \neg Q$ \\ 
\hline 
3 & Conditional Statements  & $P \to Q \equiv \neg P \vee Q$ \\ 
\hline 
4 & Conditional Statements & $\neg (P \to Q) \equiv P \wedge \neg Q$ \\ 
\hline 
5 & Conditional Statements & $P \to Q \equiv \neg Q \to \neg P$ \\ 
\hline 
6 & Biconditional Statements & $P \leftrightarrow Q \equiv (P \to Q) \wedge (Q \to P)$ \\ 
\hline 
7 & Double Negation & $\neg(\neg P) \equiv P$ \\ 
\hline 
8 & Distribution Law & $P \vee (Q \wedge R) \equiv (P \vee Q) \wedge (P \vee R)$ \\ 
\hline 
9 & Distribution Law & $P \wedge (Q \vee R) \equiv (P \wedge Q) \vee (P \wedge R)$ \\ 
\hline 
10 & Conditional with Disjunction & $P \to (Q \vee R) \equiv (P \wedge \neg Q) \to R$ \\ 
\hline 
10 & Conditional with Disjunction & $(P \vee Q) \to R \equiv (P \to R) \wedge (Q \to R)$ \\ 
\hline 
\end{tabular} 
\end{theorem}
\end{tcolorbox}


\newpage
\section{Propositional Logic Repository} 

\begin{example}
Statements \cite[Chap.1, P.C.1.1, Q.1-13]{ted} \\

Which of the following sentences are statements? 
Do not worry about determining whether a statement is true or false; just determine whether each sentence is a statement or not. \\


1. $3+4 = 8$ \\
{\it ANSWER: This is a statement} \\

2. $2*7 + 8 = 22$ \\
{\it ANSWER: This is a statement} \\

3. $(x-1)=\sqrt{x+11}$ \\
{\it ANSWER: This is not a statement since it is not known what $x$ represents} \\

4. $2x+5y=7$ \\
{\it ANSWER: This is not a statement since it is not known what $x$ and $y$ represents} \\

5. There are integers $x$ and $y$ such that $2x +5y =7$ \\
{\it ANSWER: This is a statement} \\

6. There are integers $x$ and $y$ such that $23x +37y =52$ \\
{\it ANSWER: This is a statement} \\

7. Given a line "L" and a point "P" not on a line, there is a unique line through "P" that does not intersect "L" \\
{\it ANSWER: This is a statement} \\

8. $(a+b)^3 = a^3 + 3a^2b + 3ab^2 + b^3$ \\
{\it ANSWER: This is not a statement since it is not known what $a$ and $b$ represents} \\

9. $(a+b)^3 = a^3 + 3a^2b + 3ab^2 + b^3$ for all real numbers $a$ and $b$\\
{\it ANSWER: This is a statement} \\

10. The derivative of $f(x) = sin(x)$ is $f'(x)=cos(x)$ \\
{\it ANSWER: This is not a statement since it is not known what $x$ represents} \\

11. Does the equation $3x^2 - 5x -7 = 0$ have two real number solutions 
{\it ANSWER: This is not a statement it is a question} \\

12. If $ABC$ is a right angle triangle with right angle at vertex $B$, and if $D$ is the midpoint of the hypotenuse, then the line segment connecting vertex $B$ to $D$ is half the length of the hypotenuse. \\
{\it ANSWER: This is a statement} \\

13. There do not exist three integers $x$, $y$ and $z$ such that $x^3 + y^3 = z^3$ \\
{\it ANSWER: This is a statement} \\
\end{example}


\begin{example}
Statements \cite[Chap.1, P.C.1.2, Q.1-4]{ted} \\

Use the technique of exploration to investigate each of the following statements.
Can you make a conjecture as to whether the statement is true or false? Can you determine whether it is true or false? \\


1. $(a+b)^2 = a^2 + b^2$, for all real numbers $a$ and $b$ \\
{\it ANSWER: This conjecture is because $(a+b)^2 = a^2 + 2ab + b^2$} \\

2. There are integers $x$ and $y$ such that $2x + 5y = 41$ \\
{\it ANSWER: This conjecture appears to true because there exists at least one value for $x$ and $y$ that will yield sum $41$} \\

3. If $x$ is an even integer, then $x^2$ is an even integer. \\
{\it ANSWER: This conjecture appears to be true because when square the even integers $2$ and $30$, the solution is an even integer} \\

4. If $x$ and $y$ are odd integers, then $x*y$ is an odd integer \\
{\it ANSWER: This conjecture appears to be true becuase when multiply the odd integers $3$ and $5$, the solution is an odd integer} \\
\end{example}


\begin{example}
Statements \cite[Chap.1, P.C.1.4, Q.1]{ted} \\

Use the technique of exploration to investigate each of the following statements. Can you make a conjecture as to whether the statement is true or false? Can you determine whether it is true or false? \\

\begin{tcolorbox}
    \begin{theorem}
        If $x$ is a positive real number, then $x^2 + 8x$ is a positive real number.
    \end{theorem}
\end{tcolorbox}

{\it NOTE: Although the hypothesis and conclusion of this conditional sentence are not statements, the conditional sentence itself can be considered to be a statement as long as we know what possible numbers may be used for the variable $x$. From the context of this sentence, it seems that we can substitute $0$ for $x$ or a negative real number from $x$ provided that we are willing to work with a false hypothesis in the conditional statement.} \\

1.A. Notice that  if $x=-3$ then $x^2 + 8x = -15$, which is negative. Does this mean that the given conditional statement is false? \\
{\it ANSWER: Given that both the hypothesis and conclusion are false, the the statement is true} \\

1.B. Notice that if $x=4$, then $x^2 + 8x = -15$, which is positive. Does this mean that the given conditional statement is true? \\
{\it ANSWER: Given that both the hypothesis and conclusion are true, the the statement is true} \\

1.C. Do you think this conditional statement is true or false? Record the results for at least five different examples where the hypothesis of this conditional statement is true. \\
{\it ANSWER: The conditional statement might be true because every possible positive real number hypothesis yield a conclusion that is a non positive real number } \\
\end{example}


\begin{example}
Statements \cite[Chap.1, P.C.1.4, Q.2]{ted} \\

\begin{tcolorbox}
    \begin{theorem}
        If $n$ is a positive integer, then $n^2 - n +41$ is a prime number
    \end{theorem}
\end{tcolorbox}

{\it NOTE: Remember that a prime number is a positive integer greater than $1$ whose only positive factors are $1$ and itself.} \\

2.A. To explore whether or not this statement is true, try using (and recording your result) for $n=1, n=2, n=4, n=5$ and $n=10$. Then record the results for atleast four other values of $n$ \\

ANSWER: \\
\begin{center}
\begin{tabular}{|c|c|c|}
\hline 
$n$ & $n^2 - n +41$ & Prime? \\ 
\hline 
$n=1$ & $(1)^2-(1)+41 = 41$ & Yes \\ 
\hline 
$n=2$ & $(2)^2-(2)+41 = 43$ & Yes \\
\hline 
$n=3$ & $(3)^2-(3)+41 = 47$ & Yes \\
\hline 
$n=4$ & $(4)^2-(4)+41 = 53$ & Yes \\
\hline 
$n=5$ & $(5)^2-(5)+41 = 61$ & Yes \\
\hline 
$n=10$ & $(10)^2-(10)+41 = 131$ & Yes \\
\hline 
$n=11$ & $(11)^2-(11)+41 = 151$ & Yes \\
\hline 
$n=12$ & $(12)^2-(12)+41 = 173$ & Yes \\
\hline 
$n=13$ & $(13)^2-(13)+41 = 197$ & Yes \\
\hline 
$n=14$ & $(14)^2-(14)+41 = 223$ & Yes \\
\hline 
\end{tabular} 
\end{center}

2.B. Does this conditional statement appears to be true? \\
{\it Given that the conditional statement is true for every positive integer tested, the statement might be true} \\
\end{example}


\begin{example}
Statements \cite[Chap.1, P.C.1.5]{ted} \\

Although one example can be used to prove that a conditional statement is false, in most cases, we cannot use examples to prove that a conditional statement is true.  \\

The following statement is a true statement, which is proven in many calculus texts. \\

\begin{tcolorbox}
    \begin{theorem}
        If the function $f$ is differentiable at $a$, then the function $f$ is a continuous at $a$
    \end{theorem}
\end{tcolorbox}

Using only this true statement, it is possible to make a conclusion about the function in each of the following cases? \\


1. It is known that the function $f$, where $f(x)= sin(x)$, is differentiable at 0. \\
{\it ANSWER: Since $\dfrac{d}{dx}sin(x)_{x=0}=cos(0)=1$, differentiabe implies continuous Theorem holds true because $f(x)= sin(x)$ is both differentiable and continuous at $x=0$ } \\

2. It is known that the function $f$, where $f(x)= \sqrt[3]{x}$, is not differentiable at 0. \\
{\it ANSWER: Since $\dfrac{d}{dx}\sqrt[3]{x}_{x=0}= \dfrac{1}{3(x)^{2/3}}, undefined$, differentiabe implies continuous Theorem holds true because $f(x)= \sqrt[3]{x}$ is continuous but not differentiable at $x=0$ } \\

3. It is known that the function $f$, where $f(x)= |x|$, is continuous at 0. \\
{\it ANSWER: Since $\dfrac{d}{dx}|x|_{x=0} = \dfrac{x}{|x|}, undefined$, differentiabe implies continuous Theorem holds true because $f(x)= |x|$ is continuous but not differentiable at $x=0$ } \\


4. It is known that the function $f$, where $f(x)= \dfrac{|x|}{x}$, is differentiable at 0. \\
{\it ANSWER: Since $\dfrac{d}{dx}\frac{|x|}{x}_{x=0} = 0$, differentiabe implies continuous Theorem holds true because $f(x)= \dfrac{|x|}{x}$ is differentiable but not continuous at $x=0$ } \\
\end{example}

\begin{example}
Only-If Statements \cite[Chap.2, P.C.2.1]{ted} \\

Recall that a quadrilateral  is a four-sided polygon. Let S represent the following true conditional statement:
\begin{center}
    If a quadrilateral is a square, then it is a rectangle
\end{center} 

Write this conditional statement in English using: \\
\begin{enumerate}
    \item the word "whenever"
    \item the word "only if"
    \item the phrase "is necessary for"
    \item the phrase "is sufficient for"
\end{enumerate}

Here are the answers: \\

\begin{enumerate}
    \item Whenever a quadrilateral is a square it is a rectangle 
    \item A quadrilateral is a square, only if it is a rectangle 
    \item A rectangle is a necessary for a quadrilateral to be square 
    \item A square is sufficient for quadrilateral to be a rectangle 
\end{enumerate}

\end{example}

\begin{example}
Truth Tables \cite[Chap.2, P.C.2.2]{ted} \\

Construct a truth table for each of the following statements: \\


1. $P \wedge \neg Q$  \\

\begin{center}
    \begin{tabular}{|c|c|c|c|}
        \hline 
        $P$ & $Q$ & $\neg Q$ & $P \wedge \neg Q$ \\ 
        \hline 
        T & T & F & F \\ 
        \hline 
        T & F & T & T \\ 
        \hline 
        F & T & F & F \\ 
        \hline 
        F & F & T & F \\ 
        \hline 
    \end{tabular} 
\end{center}

2. $\neg (P \wedge Q)$ \\

\begin{center}
    \begin{tabular}{|c|c|c|c|}
        \hline 
        $P$ & $Q$ & $P 	\wedge Q$ & $\neg (P \wedge Q)$ \\ 
        \hline 
        T & T & T & F \\ 
        \hline 
        T & F & F & T \\ 
        \hline 
        F & T & F & T \\ 
        \hline 
        F & F & F & T \\ 
        \hline 
    \end{tabular} 
\end{center}

3. $\neg P \wedge Q$ \\

\begin{center}
    \begin{tabular}{|c|c|c|c|c|}
        \hline 
        $P$ & $Q$ & $\neg P$ & $\neg Q$ & $\neg P \wedge \neg Q$\\ 
        \hline 
        T & T & F & F & F \\ 
        \hline 
        T & F & F & T & F \\ 
        \hline 
        F & T & T & F & F \\ 
        \hline 
        F & F & T & T & T \\ 
        \hline 
    \end{tabular} 
\end{center}

4. $\neg P \vee \neg Q$ \\

\begin{center}
    \begin{tabular}{|c|c|c|c|c|}
        \hline 
        $P$ & $Q$ & $\neg P$ & $\neg Q$ & $\neg P \vee \neg Q$\\ 
        \hline 
        T & T & F & F & F \\ 
        \hline 
        T & F & F & T & T \\ 
        \hline 
        F & T & T & F & T \\ 
        \hline 
        F & F & T & T & T \\ 
        \hline 
    \end{tabular} 
\end{center}


Do any of these statements have the same truth tables? \\ 
The statements $\neg (P \wedge Q)$ and $\neg P \vee \neg Q$ have the same truth table
\end{example}


\begin{example}
Truth Tables for the Bi-condtional Statement \cite[Chap.2, P.C.1.1]{ted} \\

Complete a truth table for $[(Q \to P) \wedge (P \to Q)]$.

\begin{center}
    \begin{tabular}{|c|c|c|c|c|}
        \hline 
        $P$ & $Q$ & $P \to Q$ & $Q \to P$ & $(P \to Q) \wedge (Q \to P)$ \\ 
        \hline 
        T & T & T & T & T \\ 
        \hline 
        T & F & F & T & F \\ 
        \hline 
        F & T & T & F & F \\ 
        \hline 
        F & F & T & T & T \\ 
        \hline 
    \end{tabular} 
\end{center}
\end{example}


\begin{example}
Tautologies and Contradictions \cite[Chap.2, P.C.2.4]{ted} \\

For statements $P$ and $Q$: \\


1. Use a truth table to show that $P \vee \neg P$ is tautology \\

\begin{center}
    \begin{tabular}{|c|c|c|}
        \hline 
        $P$ & $\neg P$ & $P \vee \neg P$ \\ 
        \hline 
        T & F & T \\ 
        \hline 
        T & F & T \\ 
        \hline 
        F & T & T \\ 
        \hline 
        F & T & T \\ 
        \hline 
    \end{tabular} 
\end{center}

2. Use a truth table to show that $P \wedge ]\neg P$ is a contradiction \\

\begin{center}
    \begin{tabular}{|c|c|c|}
        \hline 
        $P$ & $\neg P$ & $P \wedge \neg P$ \\ 
        \hline 
        T & F & F \\ 
        \hline 
        T & F & F \\ 
        \hline 
        F & T & F \\ 
        \hline 
        F & T & F \\ 
        \hline 
    \end{tabular} 
\end{center}

3. Use a truth table to determine if $P \to (P \vee Q)$ is a tautology, a contradiction or neither \\

\begin{center}
    \begin{tabular}{|c|c|c|c|}
        \hline 
        $P$ & $Q$ & $P \vee Q$ & $P \to (P \vee Q)$ \\ 
        \hline 
        T & T & T & T \\ 
        \hline 
        T & F & T & T \\ 
        \hline 
        F & T & T & T \\ 
        \hline 
        F & F & F & T \\ 
        \hline 
    \end{tabular} 
\end{center}

$P \to (P \vee Q)$ is a tautology \\
\end{example}


\newpage
\section{Quantifiers and Negation}

\begin{definition}
Universal Quantifier \\

The phrase "for every" (or its equivalent) is called a {\bf Universal Quanfifier}. The symbols $\forall$ is used to denote a Universal Quantifier. \\
Using this notation, the statement, "For each real number $x$, $x^2 > 0$ could be written in symbolic form as $(\forall x \in \bb{R})(x^2 > 0)$.  

\end{definition} 


\begin{definition}
Existential Quantifiers \\

The phrase "there exists" (or its equivalents) is called an {\bf Existential Quantifier}. The symbol $\exists$ is used to denote an existential quantifier. \\
Using this notation, the statement, "There exists an integer $x$ such that $3x - 2 = 0$" could be written in symbolic form as $(\exists x \in \bb{Z})(3x - 2 = 0)$
\end{definition} 


\begin{definition}
Properties of Quantifiers \\

\begin{center}
	\begin{tabular}{|c|c|c|}
		\hline 
		{\bf A Statement Involving} & {\bf Often has the form} & {\bf Statement is true provided that} \\ 
		\hline 
		$(\forall x, P(x))$ & For every $x$, $P(x)$ & Every value of $x$, $P(x)$ true \\ 
		\hline 
		$(\exists x, P(x))$ & There exists an $x$, $P(x)$ & At least one value of $x$, $P(x)$ true \\  
		\hline 
	\end{tabular} 
\end{center}
\end{definition}



\begin{example}
Consider the following statements written in symbolic form: $(\forall \in \bb{Z})(x \text{ is a multiple of }2)$. \\
A. As an English sentence: For all integers $x$, $x$ is a multiple of $2$ \\
B. Truth Value: This statement is false because all integers cannot be written in form $2a$ \\
C. Negation: There exists some integer $x$ such that $x$ is not a multiple of $2$ \\
E. Negation: $(\exists \in \bb{Z})(x \text{ is not a multiple of }2)$ \\
\end{example}


\begin{example}
Consider the following statements written in symbolic form: $(\exists \in \bb{Z})(x^3 > 2)$. \\
A. As an English sentence: There exist some integer $x$ such that $x^3 > 0$
B. Truth Value: This statement is true becuase at least one integer meets the criteria of the predicate \\
C. Negation: For all integers $x$, $x^3 \leqq 0$ \\
E. Negation: $(\forall \in \bb{Z})(x^3 \leqq 0)$ \\
\end{example}


\begin{definition}
Negation of Quantified Statements \\

For any open sentence $P(x)$, \[ \neg(\forall x \in U)[P(x)] \equiv (\exists x \in U)[\neg P(x)]\]  and \[ \neg(\exists x \in U)[P(x)] \equiv (\forall x \in U)[\neg P(x)]\]
\end{definition}


\begin{example}
Consider $(\forall x \in \bb{R})(x^3 \geqq x^2)$ \\

We can write this statement  as an English sentence in several ways. The following are two different ways to do so: \\
1. For each real number $x$, $x^3 \geqq x^2$ \\
2. If $x$ is a real number, then $x^3$ is greater than or equal to $x^2$ \\
We can write the negation of the statement as \[ \neg (\forall x \in \bb{R})(x^3 \geqq x^2) \equiv (\exists x \in \bb{R}) \neg (x^3 \geqq x^2) \equiv (\exists x \in \bb{R})(x^3 < x^2) \]

Similarly, we can write the negation as an English sentence is several ways. The following are two differenct ways to do so: \\
1. These exists a real number $x$ such that $x^3 < x^2$ \\
2. There exists an $x$ such that $x$ is a real number and $x^3 < x^2$
\end{example}


\begin{definition}
Counterexample \\

Suppose $(\forall x \in \bb{R})(x^3 \geqq x^2)$, The real number $x = -1$ can use to show that this statement is false. This is called a {\bf counterexample} to the statement. \\
In general, a {\bf counterexample} to a statement of the form $(\forall)[P(x)]$ is an object $a$ in the universal set U for which $P(a)$ is false. It is an example that proves that $(\forall)[P(x)]$ is a false statement, and hence its negation, $(\exists x)[\neg P(x)]$, is a true statement
\end{definition}

\begin{definition}
Negation of Conditional Statements \\

The negation of the conditional statement "If $P$ then $Q$" is the statement "$P$ and not $Q$". Symbolically, this can be written as follows: \[ \neg (P \to Q) \equiv P \wedge \neg Q \]. So when we specifically include the universal quantifier, the symbolic form of the negation of a conditional statement is \[ \neg(\forall x \in U)[P(x) \to Q(x)] \equiv (\exists x \in U)\neg[P(x) \to Q(x)] \equiv (\exists x \in U)[P(x) \wedge \neg Q(x)] \]. That is \[ \neg(\forall x \in U)[P(x) \to Q(x)] \equiv (\exists x \in U)[P(x) \wedge \neg Q(x)] \]
\end{definition}


\begin{definition}
Quantifiers in Definitions \\

Definitions of terms in mathematics often involve quantifiers even though the quantifer symbolism might not be present. For instance, let's examine the definition of square numbers (perfect squares), 
	\begin{center}
	\bf A natural number $n$ is a perfect square provided that there exists a natural number $k$ such that $n = k^2$
	\end{center}
This definition of perfect squares can be represented as a quantified definition as follows:
	\begin{center}
	\bf A natural number $n$ is a perfect square provided $(\exists k \in \bb{N})(n = k^2)$
	\end{center}
On the other hand,, when we say that a natural number $n$ is not a perfect, we need to negate the condition that there exists a natural number $k$ such that $n = k^2$. We use the symbolic form to do this. 
	\begin{center}
	$\neg(\exists k \in \bb{N})(n = k^2) \equiv (\forall k \in \bb{N})\neg(n = k^2) \equiv (\forall k \in \bb{N})(n \neq k^2)$
	\end{center}
This negated definition can easily be translated into an English sentence, 
	\begin{center}
	\bf A natural number $n$ is not a perfect square provided that for every natural number $k$, $n \neq k^2$
	\end{center}
\end{definition}


\begin{definition}
Compound Quantified Statements \\

When a predicate contain more than variable, each variable must be quantified to create a statement. For example, assume  the universal set is the set of integers, $\bb{Z}$, and let $P(x,y)$ be the predicate $x + y = 0$. We can create a statement from this predicate in several ways: \\
1. $(\forall x \in \bb{Z})(\forall y \in \bb{Z})(x + y = 0)$ is read "For all integers $x$ and $y$, $x + y = 0$" \\
2. $(\forall x \in \bb{Z})(\exists y \in \bb{Z})(x + y = 0)$ is read "For every integer $x$, there exists an integer $y$ such that $x + y = 0$"\\
3. $(\exists x \in \bb{Z})(\forall y \in \bb{Z})(x + y = 0)$ is read "There exists and integer $x$ such that for every integer $y$, $x + y = 0$"\\
4. $(\exists x \in \bb{Z})(\exists y \in \bb{Z})(x + y = 0)$ is read "There exists integers $x$ and $y$ such that $x + y = 0$" \\
\end{definition}


\begin{definition}
Negation of Compound Quantified Statements \\

When we negate a statement with more than one quantifier, we consider each quantifer in turn and apply the negation of quantified statement theorem. As an example, we will negate the statement $(\exists x \in \bb{Z})(\forall y \in \bb{Z})(x + y = 0)$. 
Firstly, we will treat this as a statement in the following form $(\exists x \in \bb{Z})[P(x)]$, where $P(x)$ is the predicate $(\forall y \in \bb{Z})(x + y = 0)$. Using the negation of quantified statements theorem, we have: \[ \neg (\exists x \in \bb{Z})[P(x) \equiv (\forall x \in \bb{Z})[\neg P(x)] \] Now let's  examine the negation of the predicate $\neg P(x)$. Again, using the negation of quantified statements theorem, we have: 
	\begin{eqnarray}
		\neg P(x) & = & \neg (\forall y \in \bb{Z})(x + y = 0) \nonumber \\
		& = & (\exists y \in \bb{Z})\neg (x + y = 0) \nonumber \\
		& = & (\exists y \in \bb{Z})(x + y \neq 0) \nonumber \\
	\end{eqnarray}
Combining these two results, we obtain \[ \neg (\exists x \in \bb{Z})(\forall y \in \bb{Z})(x + y = 0) \equiv (\forall x \in \bb{Z})(\exists y \in \bb{Z})(x + y \neq 0) \].

\end{definition}


\newpage
\section{Predicate Logic Repository} 


\begin{example}
Negation of Quantified Statements \cite[Chap.2, P.C.2.18, Q.1]{ted} \\

For each of the following statements: \\
\begin{enumerate}
    \item Write the statement in the form of an English sentence that does not use the symbols for quantifiers
    \item Write the negation of the statement in a symbolic form that does not use the negation symbol.
    \item Write the negation of the statement in the form of an English sentence that does not use the symbols for quantifiers.    
\end{enumerate}

For: \\
\begin{center}
    $(\forall a \in \bb{R})(a+0=a)$
\end{center}

\begin{enumerate}
    \item In English: For each real number $a$, $a + 0 = a$
    \item Negation: $\neg (\forall a \in \bb{R})(a+0=a) \equiv (\exists a \in \bb{R})(a+0 \neq a)$
    \item Negation in English: These exists a real number $a$ such that $a+0 \neq a$  
\end{enumerate}
\end{example}



\begin{example}
Negation of Quantified Statements \cite[Chap.2, P.C.2.18, Q.2]{ted} \\

For: 
\begin{center}
    $(\forall x \in \bb{R})(\sin(2x) = 2(\sin(x)\cos(x))$
\end{center}

\begin{enumerate}
    \item In English: For each real number $x$, $\sin(2x) = 2(\sin(x)\cos(x))$
    \item Negation: $\neg (\exists x \in \bb{R})(\sin(2x) = 2(\sin(x)\cos(x))) \equiv (\exists x \in \bb{R})(\sin(2x) \neq 2(\sin(x)\cos(x)))$
    \item Negation in English: These exists a real number $x$ such that $\sin(2x) \neq 2(\sin(x)\cos(x))$    
\end{enumerate}
\end{example}


\begin{example}
Negation of Quantified Statements \cite[Chap.2, P.C.2.18, Q.3]{ted} \\

For: 
\begin{center}
    $(\forall a \in \bb{R})(\tan^2 x + 1 = \sec^2 x)$
\end{center}

\begin{enumerate}
    \item In English: For each real number $x$, $\tan^2 x + 1 = \sec^2 x$
    \item Negation: $\neg (\forall x \in \bb{R})(\tan^2 x + 1 = \sec^2 x) \equiv (\exists x \in \bb{R})(\tan^2 x + 1 \neg \sec^2 x)$
    \item Negation in English: These exists a real number $x$ such that $\tan^2 x + 1 \neg \sec^2 x$    
\end{enumerate}
\end{example}


\begin{example}
Negation of Quantified Statements \cite[Chap.2, P.C.2.18, Q.4]{ted} \\

For: 
\begin{center}
    $(\exists x \in \mathbb{Q})(x^2 - 3x - 7 = 0)$
\end{center}

\begin{enumerate}
    \item In English: There exists a complex number $x$ such that $x^2 - 3x - 7 = 0$ 
    \item Negation: $\neg (\exists x \in \mathbb{Q})(x^2 - 3x - 7 = 0) \equiv (\forall x \in \mathbb{Q})(x^2 - 3x - 7 \neq 0)$
    \item Negation in English: For each complex number $x$ such that $x^2 - 3x - 7 \neq 0$   
\end{enumerate}
\end{example}


\begin{example}
Negation of Quantified Statements \cite[Chap.2, P.C.2.18, Q.5]{ted} \\

For: 
\begin{center}
    $(\exists x \in \bb{R})(x^2 + 1 = 0)$
\end{center}

\begin{enumerate}
    \item In English: There exists a real number $x$ such that $x^2 + 1 = 0$ \\
\item Negation: $\neg (\exists x \in \bb{R})(x^2 + 1 = 0) \equiv (\forall x \in \bb{R})(x^2 + 1 \neq 0)$ \\
\item Negation in English: For each real number $x$ such that $x^2 + 1 \neq 0$ \\  
\end{enumerate}
\end{example}
 



\begin{example}
Multiples of Three \cite[Chap.2, P.C.2.20]{ted} \\

\begin{tcolorbox}
    {\bf Definition:} An integer $n$ is a multiple of $3$ provided that there exists an integer $k$ such that $n = 3k$
\end{tcolorbox}

PART A: \\
Write this definition in symbolic form using quantifiers by completing the following: {\it An integer $n$ is a multiple of 3 provided that $\cdots$} \\
	\begin{center}
	    \bf An integer $n$ is a multiple of $3$ provided that $(\exists k \in \bb{N})(n = 3k)$
	\end{center}
	
PART B: \\
Give several examples of integers (including negative integers) that are multiples of 3. \\

The truth set of the quantified statement $(\exists k \in \bb{N})(n = 3k)$ is set of all integers, $k$. \\

PART C: \\
Give several examples of integers (including negative integers) that are not multiples of 3. \\

The truth set of the quantified statement $(\exists k \in \bb{N})(n = 3k)$ is set of all integers, $k$. The  the set of all integers that are not multiples of $3$ is the empty set, $\emptyset$. 

PART D: \\
Use the symbolic form of the definition of a multiple of $3$ to complete the following sentence: {\it An integer $n$ is not a multiple of $3$ provided that $\cdots$} \\
	\begin{center}
	\bf An integer $n$ is not a multiple of $3$ provided that $(\forall k \in \bb{N})(n \neq 3k)$
	\end{center}

PART E: \\
Without using the symbols for quantifiers, complete the following sentence: {\it An integer $n$ is not a multiple of $3$ provided that $cdots$} \\
	\begin{center}
	\bf An integer $n$ is not a multiple of $3$ provided that for each integer $k$, $n \neq 3$ 
	\end{center}
\end{example}



\begin{example}
Negative a Multi-Quantifer Statement \cite[Chap., P.C.2.21]{ted} \\

Write the negation of the statement \[ (\forall x \in \bb{Z})(\forall y \in \bb{Z})(x + y = 0) \] in symbolic form and as a sentence in English. \\


Lets treat this compound quantified statement as a statement of the form $(\forall x \in \bb{Z})[P(x)]$ where $P(x)$ is the predicate $(\forall y \in \bb{Z})(x + y = 0)$. Using the negation of quantified statements theorem, we have   
	\begin{equation}
		\label{shell}
		\neg (\forall x \in \bb{Z})[P(x)] \equiv (\exists x \in \bb{Z})[\neg P(x)]
	\end{equation}
Now we will examine the negation of the predicate $P(x)$. Using the negation of quantified statements theorem, we get
	\begin{eqnarray}
		\neg P(x) & = & \neg (\forall y \in \bb{Z})(x + y = 0) \nonumber \\
		& = & (\exists y \in \bb{Z})\neg (x + y = 0) \nonumber \\
		& = & (\exists y \in \bb{Z})(x + y \neq 0) \label{px}
	\end{eqnarray}
Combining (\ref{shell}) and (\ref{px}), we obtain \[ (\forall x \in \bb{Z})(\forall y \in \bb{Z})(x + y = 0 \equiv (\exists x \in \bb{Z})(\exists y \in \bb{Z})(x + y \neq 0)  \] And, the English translation of this statement is:
	\begin{center}
		There exist integers $x$ and $y$, such that $x + y \neq 0$	
	\end{center}
\end{example}









\newpage
\chapter{TOPICS IN PROOF STRUCTURES}
\section{Proof Structure: Trivial and Vacuous Proofs}

\begin{definition}
Trivial Proof

When the quantified statement $\forall x \in S$, $P(x) \to Q(x)$ is expressed as a result or theorem, we often write such a statement as: 
	\begin{center}
		For $x \in S$, if $P(x)$, then $Q(x)$ 
	\end{center}
or as 
	\begin{center}
		Let $x \in S$, if $P(x)$, then $Q(x)$
	\end{center}
Thus, the proposition $(\forall x \in S)[P(x) \to Q(x)]$ is true if $P(x) \to Q(x)$ is a true statement for  each $x \in S$, while the proposition $(\forall x \in S)[P(x) \to Q(x)]$ is false if $P(x) \to Q(x)$ is false for at least one element $x \in S$. In the proposition $(\forall x \in S)[P(x) \to Q(x)]$, if $Q(x)$ us true for all $x \in S$ if $P(x)$ is false for all $x \in S$, then determining the truth or falseness of the proposition $(\forall x \in S)[P(x) \to Q(x)]$ becomes considerably easier. \\
Indeed, if it can be shown that $Q(x)$ is true for all $x \in S$ (regardless of the truth value of $P(x)$ ), then, according to the truth table for the implication, is true. This constitutes a proof of the proposition $(\forall x \in S)[P(x) \to Q(x)]$ and is called a {\bf Trivial Proof}. For instance: 


\begin{tcolorbox}
	\begin{theorem}
		For all real numbers $x$. If $x>0$, then $x^2 + 5 > 0$
	\end{theorem}
\end{tcolorbox}

\begin{proof}
We assume $x$ is real number and that $x>0$, we will show via a trivial proof that $x^2 + 5 > 0$. Since $x^2 \geqq 0$ for all real numbers $x$, it follows that:
	\begin{equation}
		(x^2 + 5) > x^2 \geqq 0 \nonumber 
	\end{equation}
Hence, $x^2 + 5 \geqq 0$. Consequently, if $x>0$, then $x^2 + 5 > 0$
\end{proof}
\end{definition}


\begin{definition}
Vacuous Proof

Let $P(x)$ and $Q(x)$ be open sentences over a domain $S$. the $\forall x \in S$, $P(x) \to Q(x)$ is a true statement if it can be shown that $P(x)$ is false for all $x \in S$ (regardless of the trith value $Q(x)$), according to the truth table for implications. Such a proof is called a {\bf Vacuous Proof} of $\forall x \in S$, $P(x) \to Q(x)$. For instance, 

\begin{tcolorbox}
	\begin{theorem}
		For all real numbers $x$. If $x^2 + 1 < 0$, then $x^5 \geqq 4$
	\end{theorem}
\end{tcolorbox}

\begin{proof}
We assume $x$ is real number and that $x^2 + 1 < 0$, we will show via a vacuous proof that $x^5 > 4$. Observe that 
	\begin{equation}
		(x^2 + 1) > x^2 \geqq 0 \nonumber 
	\end{equation}

Hence, $x^2 + 1 < 0$ is false for all value of the real number $x$. Consequently, if $x^2 + 1 < 0$, then $x^5 \geqq 4$
\end{proof}

\end{definition}



\newpage
\section{Proof Repository: Trivial and Vacuous Proofs}

\begin{example}
Let $x \in \bb{R}$. Prove that $0<x<1$, then $x^2-2x + 2 \neq 0$

\begin{tcolorbox}
	\begin{theorem}
		If $0<x<1$, then $x^2-2x + 2 \neq 0$
	\end{theorem}
\end{tcolorbox}

\begin{proof}
We assume $x$ is real number and that $0<x<1$, we will show via a trivial proof that $x^2 -2x + 2 \neg 0$. By factoring the consequent, 
	\begin{equation}
		x^2 - 2x + 2 = (x-2)(x-1)  \nonumber 
	\end{equation}
Since the solution set of the consequent is $(\forall x \in \bb{R})[ 1, 2]$. Consequently, If $0<x<1$, then $x^2-2x + 2 \neg 0$

\end{proof}
\end{example}


\begin{example}
Let $n \in \bb{N}$. Prove that $|n-1|+|n+1| \leqq 1$, then $|n^2-1| \leqq 4$

\begin{tcolorbox}
	\begin{theorem}
		If $|n-1|+|n+1| \leqq 1$, then $|n^2-1| \leqq 4$
	\end{theorem}
\end{tcolorbox}

\begin{proof}
We assume $n$ is natural number and that $|n-1|+|n+1| \leqq 1$, we will show via a trivial proof that $|n^2-1| \leqq 4$. By the definition of absolute value function $|n^2-1| \leqq 4$. We can rewrite the consequent as follows: 
	\begin{equation}
		|n^2-1| \leqq 4 \equiv \left\{ 
			\begin{array}{rcl}\
				(n-1)(n+1)\leqq 4 & \mbox{for} & n \leqq 2 \\ 
				(1-n)(n+1) > 4 & \mbox{for} & n > 2 
			\end{array} \right.  \nonumber 
	\end{equation}
Since the consequent holds trivially true for all values of the natural number $n$. We conclude that if $|n-1|+|n+1| \leqq 1$, then $|n^2-1| \leqq 4$

\end{proof}
\end{example}


\begin{example}
Let $r \in \mathbb{Q}^+$. Prove that $\frac{r^2 + 1}{r} \leqq 1$, then $\frac{r^2 + 2}{r} \leqq 2$

\begin{tcolorbox}
	\begin{theorem}
		If $\frac{r^2 + 1}{r} \leqq 1$, then $\frac{r^2 + 2}{r} \leqq 2$
	\end{theorem}
\end{tcolorbox}

\begin{proof}
We assume $r$ is a positive rational number where $\frac{r^2 + 1}{r} \leqq 1$, we will show via a vacuous proof that $\frac{r^2 + 2}{r} \leqq 2$. we will proceed by demonstrating that the antecedent is a false proposition by showing that the negation of the antecedent is true statement. Since the negation of the antecedent is that the exists a positive rational number $r$ such that $\frac{r^2 + 1}{r} > 1$. \\ 
Suppose that $r = \frac{2}{3}$, it follows that; 

	\begin{equation}
		\frac{r^2 + 1}{r}>\frac{(\frac{2}{3})^2+1}{\frac{2}{3}} = 2\frac{1}{6}>1 \nonumber 
	\end{equation}

Since negation of the antecedent is a true proposition, $\frac{r^2 + 1}{r} \leqq 1$ is vacuously false for all values of the positive rational number $r$. Consequently, if $\frac{r^2 + 1}{r} \leqq 1$, then $\frac{r^2 + 2}{r} \leqq 2$.
\end{proof}
\end{example}


\begin{example}
Let $x \in \bb{R}$. Prove that $x^3 -5x - 1 \geqq 0$, then $(x-1)(x-3)\geqq -2$

\begin{tcolorbox}
	\begin{theorem}
		If $x^3 -5x - 1 \geqq 0$, then $(x-1)(x-3)\geqq -2$
	\end{theorem}
\end{tcolorbox}

\begin{proof}
We assume $x$ is real number with $x^3 -5x - 1 \geqq 0$, and we will show via a trivial proof that $(x-1)(x-3)\geqq -2$. Using algebra, we can rewrite the consequent in the form: 
\begin{alignat*}{2}
 	(x-1)(x-3) 			& \geqq -2 & \\
 	x^2 -4x + 3  		& \geqq -2\\
 	(x^2 -4x + 4) - 1 	& \geqq -2& \\
 	(x-2)^2 -1			& \geqq -2& \\
 	(x-2)^2				& \geqq -1& \\
\end{alignat*}
Since, 
	\begin{equation}
		(x-1)(x-3) \geqq -2 \equiv (x-2)^2 \geqq -1  \nonumber 
	\end{equation}
we conclude that the consequent is trivally true. In order words, if $x^3 -5x - 1 \geqq 0$, then $(x-1)(x-3)\geqq -2$. 
\end{proof}
\end{example}



\begin{example}
Let $n \in \bb{N}$. Prove that $n + \frac{1}{n} < 2$, then $n^2 + \frac{1}{n^2} < 4$

\begin{tcolorbox}
	\begin{theorem}
		If $n + \frac{1}{n} < 2$, then $n^2 + \frac{1}{n^2} < 4$
	\end{theorem}
\end{tcolorbox}

\begin{proof}
We assume $n$ is a natural number where $n + \frac{1}{n} < 2$, and we will show via a vacuous proof that $n^2 + \frac{1}{n^2} < 4$. We will proceed by demonstrating that the antecedent is a false proposition by showing that the negation of the antecedent is true statement. Since the negation of the antecedent is that the exists a natural number $n$ such that $n + \frac{1}{n} \geqq 2$. \\ 
Suppose that $n = 2$, it follows that; 

	\begin{equation}
		n + \frac{1}{n} \geqq 2 + \frac{1}{2} \geqq 2\frac{1}{2} \geqq 2 \nonumber 
	\end{equation}

Since negation of the antecedent is a true proposition, $n + \frac{1}{n} < 2$ is vacuously false for all values of the natural number $n$. Consequently, if $n + \frac{1}{n} < 2$, then $n^2 + \frac{1}{n^2} < 4$
\end{proof}
\end{example}




\newpage
\section{Proof Structure: Direct Proof}
\begin{definition}
Direct Proof \\

A {\bf direct proof of a conditional statement} is a demonstration that the conclusion of the conditional statement follows logically from the hypothesis of the conditional statement. Definitions and previously proven propositions are used to justiy each step in the proof. \\
\end{definition}

\begin{definition}
Constructing a Proof of a Conditional Statement

In order to prove that a conditional statement $P \to Q$ is true, we only need to prove that $Q$ is true whenever $P$ is true. This is becuase the conditional statement is true whenever then hypothesis is false. So in a direct proof of $P \to Q$, we assume that $P$ is true, and using this assumption, we proceed through a logical sequence of steps to arrive at the conclusion that $Q$ is true. \\
\end{definition}


\section{Proof Repository: Direct Proofs}
\begin{example}

Prove that {\bf If $x$ and $y$ are odd integers, then $x \m y$ is an odd integer} \\

\begin{tcolorbox}
	\begin{theorem}
		If $x$ and $y$ are odd integers, then $x \m y$ is an odd integer
	\end{theorem}
\end{tcolorbox}

\begin{proof}
We assume that $x$ and $y$ are odd integers and will show via a direct proof that $x \m y$ is an odd integer. Since $x$ and $y$ are odd, there exists integers $m$ and $n$ such that $x = 2m + 1$ and $y = 2n + 1$. Substituting these expressions into $x \m y$ yields:

\begin{eqnarray*}
	x \m y & = & (2m + 1)(2n + 1) \nonumber \\
	& = & 4mn + 2m + 2n + 1 \nonumber \\
	& = & 2(2mn + m + n) + 1 \nonumber \\
\end{eqnarray*}

Where $2mn + m + n$ is an integer because integers are closed under addition and multiplication. Since $x \m y = 2q + 1$ for some integer $q = 2mn + m + n$, we conclude $x \m y$ is an odd integer. Consequently, it has been proven that if $x$ and $y$ are odd integers, then $x \m y$ is an odd integer \\
\end{proof}
\end{example}


\begin{example}
{\bf Reference: e.1.2; Question:3.C} \\
If $n$ is an odd integer, then $n^2$ is an odd integer. \\

\begin{tcolorbox}
	\begin{theorem}
		If $n$ is an odd integer, then $n^2$ is an odd integer
	\end{theorem}
\end{tcolorbox}

\begin{proof}
We assume that $n$ is an odd integer, and we will show that $n^2$ is odd. Using the definitions of odd integers, we see that $n = 2a + 1$ for some integer $a$. Expressing $n^2$ in terms of $n = 2a + 1$, using algebra we get
\begin{eqnarray*}
n^2 & = & (2a + 1)^2  \nonumber \\
& = & 2(2a^2 + 2a) + 1 \nonumber \\
& = & 2q + 1 \nonumber \\
\end{eqnarray*}
Since $a$ is an integer closed under multiplication and addition, we conclude that $q$ is an integer. Such that $n^2 = 2q + 1$ for some integer $q$, hence, $n^2$ is an odd integer. Consequently, it has been proven that if $n$ is an odd integers, then $n^2$ is an odd integer. \\
\end{proof}
\end{example}



\newpage
\section{Proof Structure: Counterexample}

\begin{example}
Proof by Counterexample

\begin{tcolorbox}
	\begin{conjecture}
		For each integer $n$, if $5$ divides $(n^2 + 1 )$, then $5$ divides $(n + 1)$
	\end{conjecture}
\end{tcolorbox}

The integer $n = 4$ is a counterexample that proves this conjecture is false. Notice that when $n=4$, $n^2 - 1 = 15$ and $5$ divides $15$. Hence the hypothesis of the conjecture is true in this case. In addition, $n - 1 = 3$ and $5$ does not divide $3$ and so the conclusion is false in this case. Since this is an example where the hypothesis is true and the conlcusion is false, the conjecture is false. 
\end{example}

\section{Proof Repository: Counterexample}

\begin{example}
\cite[Chap.2, P.C.2.19, Q.1]{ted} \\

Use counterexamples to explain why each of the following statements is false. 

\begin{center}
    For each integer $n$, $(n^2 + n + 1) \text{ is a prime number}$ \\
\end{center}

\begin{proof}
    If $x = 7$, then $7^2 + 7 + 1 = 57$ and so $57$ is not a prime number. Thus, $x = 7$ is a counterexample to the statement $(\forall x \in \bb{Z})(n^2 + n + 1)$. \\
\end{proof}
\end{example}


\begin{example}
\cite[Chap.2, P.C.2.19, Q.2]{ted} \\

Use counterexamples to explain why each of the following statements is false. 

\begin{center}
    For each real number $x$, if $x$ is positive, then $2x^2 > x $ \\
\end{center}

\begin{proof}
    If $x = 0.10$, the $2(0.10)^2 < 0.1$. Thus $x = 0.10$ is a counterexample to the statement $[\forall (x \in \bb{R}) \wedge (x > 0)](2x^2 > x)$. \\
\end{proof}
\end{example}

\begin{example}
\cite[Chap.3, P.C.3.3]{ted} \\

Use a counterexample to prove the following statement is false: 

\textbf{\begin{tcolorbox}
	\begin{conjecture}
		For all integers $a$ and $b$, if $5$ divides $a$ or $5$ divides $b$, then $5$ divides $(5a + b)$
	\end{conjecture}
\end{tcolorbox}}

\begin{proof}
    The integers $a = 20$ and $b = 21$ is a counterexample that proves this conjecture is false. Notice that when $a=20$  we see that $5$ divides $20$ and when $b = 21$ we see that $5$ does not divide $21$. Hence the hypothesis of the conjecture is true in this case. In addition, $5a + b = 121$ and $5$ does not divide $121$ and so the conclusion is false in this case. Since this is an example where the hypothesis is true and the conclusion is false, the conjecture is false. \\ 
\end{proof}
\end{example}














\newpage
\section{Proof Structure: Logical Equivalency}
When proving a biconditional statment using the logical equivalency $P \leftrightarrow Q \equiv (P \to Q) \wedge (Q \to P)$, we actually need to prove two conditional statements. The proof of each conditional statement can be considered as one of two parts of the proof of the biconditional statement. Make sure that the start and end of each parts is indicated clearly. This is clearly indicated in the biconditional proof examples below: 


\section*{Proof Repository: Logical Equivalency}

\begin{example}
Let $x$ be a real number such that the real number $x$ equals $2$ if and only if $x^3 - 2x^2 + x = 2$

\begin{tcolorbox}
	\begin{theorem}
		The real number $x = 2$ if and only if $x^3 - 2x^2 + x = 2 $
	\end{theorem}
\end{tcolorbox}

\begin{proof}
We will prove this biconditional statement by proving the following two conditional statements: \\
1. For each real number $x$, if $x$ equals $2$, then $x^3 -2x^2 + x = 2$ \\
2. For each real number $x$, if $x^3 -2x^2 + x = 2$, then $x$ equals $2$ \\

For the forward direction, we assume $x$ is a real number such that $x = 2$. And, we will prove via direct proof that $x^3 -2x^2 + x = 2$. We will proceed by substituting $x = 2$ into the expression $x^3 -2x^2 + x = 2$, this gives:
	\begin{eqnarray*}
		2 & = & (2)^3 -2(2)^2 + (2) + 2 \nonumber \\
		& = & 8 - 8 + 2 \nonumber \\
		& = & 2 \nonumber \\
	\end{eqnarray*}
Since the expression becomes $2 = 2$ after substituting $x = 2$, we conclude that if $x$ equals $2$, then $x^3 -2x^2 + x = 2$ holds. This completes the proof of the forward direction. \\
For the backwards direction, we assume that $x$ is a real number such that if $x$ equals $2$, then $x^3 -2x^2 + x = 2$. And we will show via direct proof that $x = 2$. Factoring the expression $x^3 -2x^2 + x = 2$ yields: 
	\begin{eqnarray*}
		0 & = & x^3 -2x^2 + x - 2 \nonumber \\
		& = & x^2(x - 2) + (x - 2) \nonumber \\
		& = & (x - 2)(x^2 + 1) \nonumber \\
	\end{eqnarray*}
Now, in the real numbers, if a product of two factors is equal to zero, then one of the factors must be zero. So the last equation implies that $ (x = 2) \vee (x^2 = 1)$. Since $x^2 = 1$ has no real solution, we conclude that $x = 2$. This completes the backwards direction  proof. \\
Since, we have now proven both the directions of the biconditional, we have proven that the real number $x = 2$ if and only if $x^3 - 2x^2 + x = 2 $
\end{proof}

\end{example}


\begin{example}
\cite[Chap.2, P.C.2.7, Q.1]{ted} \\

Although it is possible to use truth tables to show that $P \to (Q \vee R)$ is logically equivalent to $(P \wedge \neg Q) \to R$, we instead use previously proven logical equivalencies to prove this logical equivalency. In this case, it may be easier to start working with $(P \wedge \neg Q) \to R$. Starting with $(P \wedge \neg Q) \to R \leftrightarrow \neg (P \wedge \neg Q) \vee R$ which is justified by the logical equivalency. Continue by using one of De Morgan's law on $\neg (P \wedge \neg Q)$ \\ 


\begin{tcolorbox}
    \begin{theorem}
        $(P \wedge \neg Q) \to R \leftrightarrow \neg (P \wedge \neg Q) \vee R$
    \end{theorem}
\end{tcolorbox}

\begin{proof}
    We assume $P$, $Q$, and $R$ are propositions such that $(P \wedge \neg Q) \to R$ and we will show via a direct proof that that $(P \wedge \neg Q) \to R \leftrightarrow \neg (P \wedge \neg Q) \vee R$. Using the the logical equivalency of conditional statements and De Morgan's Law, we have: 
    
    \begin{eqnarray}
        (P \wedge \neg Q) \to R & = & \neg (P \wedge \neg Q) \vee R \nonumber \\
        & = & \neg P \vee (Q \vee R) \nonumber \\
        & = & P \to (Q \vee R) \nonumber \\
    \end{eqnarray}
    
    Hence $(P \wedge \neg Q) \to R$ is logically equivalent to $P \to (Q \vee R)$
\end{proof}
\end{example}



\begin{example}
\cite[Chap.2, P.C.2.7, Q.2]{ted} \\

Let $a$ and $b$ be integers. Suppose we are trying to prove that 
	\begin{center}
		If $3$ is a factor of $a \m b$, then $3$ is a factor of $a$ or $3$ is a factor of $b$
	\end{center} 

Explain why we will have proven this statement is we prove that:
	\begin{center}		
		If $3$ is a factor of $a \m b$ and $3$ is not a factor of $a$, then $3$ is a factor of $b$
	\end{center}

\begin{tcolorbox}
	\begin{theorem}
		If $A$, $B$, and $C$ are propositions, then $A \to (B \vee C) \leftrightarrow (A \wedge \neg B) \to C$.
	\end{theorem}
\end{tcolorbox} 
 
\begin{proof}
    We assume that $A = 3$ is a factor of $a \m b$, $B = 3$ is a factor of $a$ and $C = 3$ is a factor of $b$. \\
    Since $A \to (B \vee C) \leftrightarrow (A \wedge \neg B) \to C$, we can conclude that 
    	\begin{center}
    		If $3$ is a factor of $a \m b$, then $3$ is a factor of $a$ or $3$ is a factor of $b$
    	\end{center}
    is logically equivalent to the following proposition 
    	\begin{center}
    		If $3$ is a factor of $a \m b$ and $3$ is not a factor of $a$, then $3$ is a factor of $b$
    	\end{center}
    \end{proof}
\end{example}





\newpage
\section{Proof Structure: Contrapositive}
One of the most useful logical equivalencies is a conditional statement $P \to Q$ is logically equivalent to its contrapositive $\neg Q \to \neg P$. This means that if we prove the contrapositve of the conditional statement, then we have proven the conditional statement. The following are some important points to remember: \\
1. A conditional statement is logically equivlent to its contrapositive \\
2. Use a direct prove to prove that $\neg Q \to \neg P$ is true \\
3. CAUTION: One difficulty with this tyoe of proof is in the formation of correct negations. (We need to be very careful doing this.) \\
4. We might consider using a proof by contraposive when the statement $P$ and $Q$ are stated as negations. \\ 


\begin{example}
If $n^2$ is an odd integer, then $n$ is an odd integer. \\

\begin{tcolorbox}
	\begin{theorem}
	\label{the}		
		If $n^2$ is an odd integer, then $n$ is an odd integer
	\end{theorem}
\end{tcolorbox}

\begin{proof}
We will prove this result by proving the contrapositve of this theorem which is
	\begin{center}
		If $n$ is an even integer, then $n^2$ is even integer
	\end{center}

We assume that $n$ is an even integer, and we will show that $n^2$ is even. Using the definitions of even integers, we see that $n = 2a$ for some integer $a$. Expressing $n^2$ in terms of $n = 2a$, using algebra we get
	\begin{eqnarray*}
		n^2 & = & (2a)^2  \nonumber \\ 
		& = & 2(2a^2) \nonumber \\
		& = & 2q \nonumber \\
	\end{eqnarray*}
Since $a$ is an integer closed under multiplication and addition, we conclude that $q$ is an integer. Such that $n^2 = 2q$ for some integer $q$, hence, $n^2$ is an even integer. We have thus proved the contrapostive of the theorem, and consequently, we have prove that if $n^2$ is an odd integer, then $n$ is an even integer. \\
\end{proof}
\end{example}



\section{Proofs by Contradiction Repository}
begin{example}
Let $x$ be an integer. If $5x - 7$ is an even integer, then $x$ is an odd integer. \\

\begin{tcolorbox}
	\begin{theorem}
	\label{the}		
		If $5x - 7$ is an even integer, then $x$ is an odd integer
	\end{theorem}
\end{tcolorbox}

\begin{proof}
We will prove this result by proving the contrapositve of this Theorem which is
	\begin{center}
		If $x$ is an even integer, then $5x - 7$ is an odd integer
	\end{center}

We assume that $x$ is an even integer, and we will show that $5x - 7$ is odd. Using the definitions of even integers, we see that $x = 2a$ for some integer $a$. Expressing $5x - 7$ in terms of $x = 2a$, using algebra we get
	\begin{eqnarray*}
		5x - 7 & = & 5(2a) - 7  \nonumber \\
		& = & 10a - 8 + 1 \nonumber \\
		& = & 2(5a - 4) + 1 \nonumber \\
		& = & 2q + 1 \nonumber \\
	\end{eqnarray*}
Since $5a - 4$ is an integer closed under multiplication and addition, we conclude that $q$ is an integer. Such that $5x - 7 = 2q + 1$ for some integer $q$, hence, $5x - 7$ is an odd integer. We have thus proved the contrapostive of the theorem, and consequently, we have prove that if $5x - 7$ is an even integer, then $x$ is an odd integer. \\
\end{proof}


















\begin{example}
Let $n \in \bb{Z}$. If $n^2 \neq n (\text{mod 3})$, then $n \neq 0 (\text{mod 3})$ and $n \neq 1 (\text{mod 3})$

\begin{tcolorbox}
	\begin{theorem}
	\label{the3}
		If $n^2 \neq n (\text{mod 3})$, then $n \neq 0 (\text{mod 3})$ and $n \neq 1 (\text{mod 3})$
	\end{theorem}
\end{tcolorbox}

We will prove this result by proving the contrapositve of Theorem \ref{the3} which is
	\begin{center}
		If $n = 0 (\text{mod 3})$ or $n = 1 (\text{mod 3})$, then $n^2 = n (\text{mod 3})$
	\end{center}
PROOF REQUIRED

\end{example}




\newpage
\section{Proof Structure: Existence Theorem I}

\begin{definition} 
Existence Theorem I

In an {\bf existence theorem} (sometimes called Constructive Proof or Existence Proof) the existence of an object (or objects) possessing some specified propery or properties is asserted. Typically then, an existence theorem is concerning an open sentence $P(x)$ over a domain $S$ can be expressed as a quantified statement
	\begin{center}
		$\{ \exists x \in S | P(x) \}$: There exists an $x \in S$ such that $P(x)$
	\end{center}
This proof technieque requires that we actually name, describe or explain how to construct some object in the universe that makes $P(x)$ true.
\end{definition}

\begin{example}
If $a$, $b$, and $c$ are real numbers with $a \neq 0$, then the linear equation $ax + b = c$ has exactly one one real number solution, which is $x = \frac{c-b}{a}$ \\

\begin{tcolorbox}
	\begin{theorem}
		The Linear equation $ax + b = c$ has exactly one one real number solution, which is $x = \frac{c-b}{a}$
	\end{theorem}
\end{tcolorbox}

\begin{proof}
We assume that $ax + b = c$ is a linear equation  with real number coefficients $a$, $b$, and $c$, such that $a \neq 0$; we will show via a constructive proof that the linear equation $ax + b = c$ has exactly one one real number solution, which is $x = \frac{c-b}{a}$. \\
We can solve the linear equation by adding $-b$ to both sides of the equation and then dividing both sides of the resulting equation by $a$, to obtain: 
	\begin{eqnarray*}
		x & = & \frac{c-b}{a}
	\end{eqnarray*}
This shows that if there is a solution, then it must be $x = \frac{c-b}{a}$. We also see that if $x = \frac{c-b}{a}$, then
	\begin{eqnarray*}
		ax + b & = & a(\frac{c-b}{a}) + b  \nonumber \\
		& = & (c-b) + b \nonumber \\
		& = & c \nonumber \\
	\end{eqnarray*}
Therefore, the linear equation $ax + b = c$ has exactly one real number solution and the solution is $x = \frac{c-b}{a}$. 
\end{proof}
\end{example}


\newpage
\section{Proof Structure: Existence Theorem II}

\begin{definition}
Existence Theorem II

Another type of proof that is often used to prove  an Existence Theorem is the {\bf Nonconstructive Proof}. For this type of proof, we make an argument that an object in the universal set that makes $P(x)$ true must exists but we never construct or name the object that makes $P(x)$ true. \\
For example, there are theorems in mathematics that tells us that every polynomial of odd degree with real coefficients has at least one real number in its solution set, but we don't know to find a real number solution for every such polynomial. 
\end{definition}

\begin{example}
The proof of the {\bf Intermediate Value Theorem} is an example of an Nonconstructive Proof.

\begin{tcolorbox}
	\begin{theorem}
		If $f$ is a continuous function on the closed interval $[a,b]$ and if $q$ is any real number strictly between $f(a)$ and $f(b)$, then there exists a number $c$ in the interval $(a,b)$ such that $f(c) = q$
	\end{theorem}
\end{tcolorbox}

\begin{proof}
Assume that $x$ is a real number. We will use the Intermediate Value Theorem to prove that the equation $x^3 - x + 1 = 0$ has a real number solution. \\
To investigate solutions of the equation $x^3 - x + 1 = 0$, we will use the function: 
	\begin{equation}
		f(x) = x^3 - x + 1 \nonumber
	\end{equation}
Notice that $f(-2) = - 5$ and that $f(0) = 1$. Since $f(-2) < 0$ and $f(0) > 0$, the Intermediate Value Theorem tells us that there is a real number $c$ in the interval $(-2, 0)$ such that $f(c) = 0$. This means that there exists a real number $c$ between $-2$ and $0$ such that: 
	\begin{equation}
		c^3 - c + 1 = 0 \nonumber
	\end{equation}
and hence $c$ is a real number solution of the equation $x^3 - x  + 1 = 0$. This proves that the equation $x^3 - x + 1 = 0$ has at least one real number solution. 
\end{proof}

NOTICE: this proof of the Intermediate Value Theorem does not tell us how to find the exact value of $c$. It does however, suggest a method for approximating the value of $c$. This can be done by finding a smaller and smaller interval $[a, b]$ such that $f(a)$ and $f(b)$ have opposite signs. 
\end{example}









\newpage
\section{Proof Structure: Contradiction}
Suppose, as usual, that we would like to show that a certain mathematical statement $R$ is true. If $R$ is expressed as the quantified statement: 
	\begin{center}
 		$\forall x \in S$, $P(x) \to Q(x)$
	\end{center}	  
Then we already have two possible proof techniques to prove such as statement: a direct proof and a proof by contrapositive. We now introduce a third proof technique that can be used to establish the truth of $R$, regardless of whether $R$ is expressed in term of an implication. \\ 

Suppose we assume $R$ is an false statement, from this assumption, we are able to arrive at or deduce a statement that contradicts some assumption we made in the proof or some known fact. (The known fact might be a definition, an axiom, or a previously proven theorem). If we denote this assumption or known fact by $P$, then what we have deduced is $\neg P$ and have thus produced the contridiction $C:P \wedge \neg P$. We  have therefore established the truth of the implication: 
	\begin{center}
		$\neg R \to C$
	\end{center}	 
However, because $\neg R \to C$ is true $C$ is false, it logically follows that $\neg R$ is false and so $R$ is true, as desired. This technique is called {\bf Proof by Contradiction}.\\

If $R$ is the quantified statement $\forall x \in S$, $P(x) \to Q(x)$, then a proof by contridiction of this statement consists of verifying the implication:
	\begin{center}
		$\neg [\forall x \in S$, $P(x) \to Q(x)] \to C$
	\end{center}
For some contradictory mathematical statement $C$. However, since  

	\begin{alignat*}{2}
 		\neg [\forall x \in S, P(x) \to Q(x)]	& \equiv \exists x \in S, \neg [P(x) \to Q(x)] & \\
 		& \equiv \exists x \in S,P(x) \wedge \neg Q(x) & \\	
	\end{alignat*} 
it follows that a proof by contridiction of $\forall x \in S, P(x) \to Q(x)$ would begin by assuming the \ of some element $x \in S$ such that $P(x)$ is true and $Q(x)$ is false.\\
{\bf That is, a proof by contradiction begins by assuming the existence of a counterexample of this quantified statement $\forall x \in S, P(x) \to Q(x)$ exists.} 


\newpage
\section{Proof Repository: Contradiction}
\begin{example}
For each real number $x$, if $0 < x < 1$, then $\frac{1}{x(1-x)} \geqq 4$

\begin{tcolorbox}
	\begin{theorem}
	\label{the2}		
		If $0 < x < 1$, then $\frac{1}{x(1-x)} \geqq 4$
	\end{theorem}
\end{tcolorbox}

\begin{proof}

We will prove this result by proving the negation of the proposition which is
	\begin{center}
		There exists some real number $x$ such that $0 < x < 1$ and $\frac{1}{x(1-x)} < 4$
	\end{center}

We assume that $x$ is a real number, where $0 < x < 1$. We note that since $0 < x < 1$, we conclude that $x > 0$ and $1-x > 0$ if we multiply both sides of the inequality $\frac{1}{x(1-x)} < 4$ by $x(1-x)$ we obtain: 
	\begin{center}
		$1 < 4x(1-x)$
	\end{center}
We can now use algebra to rewrite the last inequality as follows: 
	\begin{alignat*}{2}
		1 				&< 4x(1-x)& \\
		4x^2 - 4x + 1 	&< 0& \\
		(2x - 1)^2 		&< 0& \\
	\end{alignat*} 
However, $(2x - 1)$ is a real number and the last inequality say that a real number squared is less than zero. This is a contradiction since the square of any real number must be greater or equal to zero. Hence, the proposition cannot be false. Consequently, for each real number $x$, $0 < x < 1$, then $\frac{1}{x(1-x)} \geqq 4$

\end{proof}
\end{example}


\newpage
\begin{example}
For each real numbers $x$ and $y$, if $x$ is rational and $x \neq 0$ and $y$ is irrational, then $x \m y$ is irrational
\begin{tcolorbox}
	\begin{theorem}
	\label{the2}		
		If $x$ is rational and $x \neq 0$ and $y$ is irrational, then $x \m y$ is irrational
	\end{theorem}
\end{tcolorbox}

\begin{proof}

We will prove this result by proving the negation of the proposition which is
	\begin{center}
		There exists some real number $x$ and $y$ such that $x$ is rational, $x \neq 0$, $y$ is irrational and $x \m y$ rational
	\end{center}

We assume that $x$ and $y$ are real numbers such that $x$ is rational, $x \neg 0$, $y$ is irrational and $x \m y$ rational. Since $x \neg 0$, we can divide by $x$ and since the rational numbers are closed under division by nonzero rational numberes, we know that $\frac{1}{x} \in \mathbb{Q}$. We now know that $x \m y$ and $\frac{1}{x}$ are rational numbers and since the rational numbers are closed under multiplication, we conclude that: 
	\begin{equation}
		\frac{1}{x} \m (xy) \in \mathbb{Q}
	\end{equation}
However, $\frac{1}{x} \m (xy) = y$ and hence, $y$ must be rational number. Since a real number cannot be both rational and irrational, this is a contradiction to the assumption that $y$ is irrational. We have therefore proved that for all real numbers $x$ and $y$, if $x$ is rational and $x \neq 0$ and $y$ is irrational, then $x \m y$ is irrational

\end{proof}
\end{example}


\newpage
\begin{example}
There is not smallest positive real number
\begin{tcolorbox}
	\begin{theorem}
	\label{the2}		
		There is no smallest positive real number
	\end{theorem}
\end{tcolorbox}

\begin{proof}

We will prove this result by proving the negation of the proposition which is
	\begin{center}
		There exists a smallest positive real number
	\end{center}

We assume that $r$ is the smallest positive real number. Since $0 < \frac{r}{2} < r$, it follows that $\frac{r}{2}$ is a positive real number that is smaller than $r$. This however is a contradiction. We have therefore proved that there is no smallest positive real number. 

\end{proof}
\end{example}

\newpage
\begin{example}
No odd integer can be expressed as the sum of three even integers
\begin{tcolorbox}
	\begin{theorem}
	\label{the2}		
		No odd integer can be expressed as the sum of three even integers
	\end{theorem}
\end{tcolorbox}

\begin{proof}
    We will prove this result by proving the negation of the proposition which is
    	\begin{center}
    		There exists a odd integer which can be expressed as the sum of three even integers
    	\end{center}
    
    We assume that $x$, $y$, $z$, and $n$ are integers, such that $x$, $y$ and $z$ are even integers. We also assume that $n$ is an odd integer. We can express $n$ as the sum of three even integers $n = x + y + z$. By the definition of even integers, we let $x=2a$, $y=2b$ and $z=2c$, where $a$, $b$, and $c$ are integers. Now, substituting $x$, $y$ and $z$ into the equation $n = x + y + z$ yields
    		\begin{eqnarray*}
    			n & = & x + y + z \nonumber \\
    			& = & 2a + 2b + 2c \nonumber \\
    			& = & 2(a + b + c) \nonumber \\
    		\end{eqnarray*}
    Since $a + b + c$ are integers and integers are closed under addition. Then, $n=2k$ for some integer $k = a + b + c$. This is a contradiction. We have therefore proved that there is no odd integer can be expressed as the sum of three even integers.
\end{proof}
\end{example}


\newpage
\begin{example}
For all integers $a$ and $b$
\begin{tcolorbox}
	\begin{theorem}
	\label{the2}		
		If $a$ is an even integer and $b$ is an odd integer, then $4 \nmid (a^2 + b^2)$ 
	\end{theorem}
\end{tcolorbox}

\begin{proof}
    We will prove this result by proving the negation of the proposition which is
    	\begin{center}
    		There exists an even integer $a$ and an odd integer $b$, such that $4 | (a^2 + 2b^2)$
    	\end{center}
    
    Using the definition of odd and even integers, $a = 2x$ and $b = 2y + 1$, for some integer $x$ and $y$; using the divisibility property of integers, we see that $a^2 + b^2 = 4z$, for some integer $z$. Thus, substituting $a = 2x$ and $b = 2y + 1$ into $a^2 + b^2 = 4z$ yields: 
        \begin{align*}
            (2x)^2 + (2y + 1)^2 & = 4z \\
            4x^2 + 2(4y^2 + 4y + 1) & = 4z \\
            4x^2 + 8y^2 + 8y + 2 & = 4z \\
            2 & = 4z - 4x^2 - 8y^2 - 8y  \\
            2 & = 4(z - x^2 - 2y^2 - 2y) \\
        \end{align*}
    Since $z - x^2 - 2y^2 - 2y$ is an integer becuase integers are closed under addition and multiplication, such that $4 | 2$. By the divibility propery of integers, we have a contradition because $4 \nmid 2$. Hence the proposition is true. 
\end{proof}
\end{example}



\newpage
\begin{example}
Prove that the integer $100$ cannot be written as the sum of three integers, an odd number of which are odd. 
\begin{tcolorbox}
    \begin{theorem}
        The integer $100$ cannot be written as the sum of three integers, an odd number of which are odd 
    \end{theorem}
\end{tcolorbox}

\begin{proof}
    We assume that $a$, $b$, and $c$ are integers and will proceed by cases to prove the negation of the proposition that  
        \begin{center}
            For an odd number of odd integers $a$, $b$, and $c$,
                \begin{equation*}
                    100 = a + b + c
                \end{equation*}
        \end{center}
    \textit{Case 1: Exactly one of $a$, $b$ and $c$ is odd} \\
        We assume, without loss of generality, that $a$ is odd. By the definition of integer, $a = 2x + 1$, $b = 2y$, and $c = 2z$, for some integers $x$, $y$ and $z$. adding these integers, yield:
            \begin{align*}
                100 & = a + b + c \\
                100 & = (2x + 1) + 2y + 2z \\
                100 & = 2x + 2y + 2z + 1 \\
                100 & = 2(x + y + z) + 1 \\
            \end{align*}
        Since $x + y + z$ is an integer, we see that $2(x + y + z) + 1$ is odd integer and that $100$ is an even intger. Since odd integer is not equal to an even intger, we have a contradiction. Consequently, $100$ cannot be written as the sum of a odd and two even integers. \\
    
    \textit{Case 2: All of $a$, $b$ and $c$ are odd} \\
        Again, using the definition of integgers, $a = 2p + 1$, $b = 2q + 1$, and $c = 2r + 1$, for some integers $p$, $q$ and $r$; adding these integers, yield:
            \begin{align*}
                100 & = a + b + c \\
                100 & = (2p + 1) + (2q + 1) + (2r + 1) \\
                100 & = 2p + 2q + 2r + 2 + 1 \\
                100 & = 2(p + q + r + 1) + 1 \\
            \end{align*}
        Since $p + q + r + 1$ is an integer, we see that $2(p + q + r + 1) + 1$ is an odd integer and that $100$ is an even integer. Since integers cannot have dual parity, we have a contrdition. Consequently, $100$ cannot be written as the sum of three odd integers. \\
        
    Because we have proven both conditional statements, we have proven that the integer $100$ cannot be written as the sum of three integers, an odd number of which are odd. 
\end{proof}
\end{example}

\newpage
\begin{example}
Prove that for every $m$ such that $2 | m$ and $4 \nmid m$, there exists no integers $x$ and $y$ for which $x^2 + 3y^2 = m$
\begin{tcolorbox}
    \begin{theorem}
        For every integer $m$ such that $2 | m$ and $4 \nmid m$, there exists no integers $x$ and $y$ for which $x^2 + 3y^2 = m$
    \end{theorem}
\end{tcolorbox}

\begin{proof}
    We will prove this result by proving the negation of the proposition which is $\neg (\forall m \in \bb{Z})(\exists x \in \bb{Z})(\exists y \in \bb{Z}) \{ \text{If } (2 | m \wedge 4 \nmid m) \text{, then }x^2 + 3y^2 \neq m\}$:
        \begin{align*}
            & \equiv (\exists m \in \bb{Z})(\forall x \in \bb{Z})(\forall y \in \bb{Z}) \neg \{ \text{If } (2 | m \wedge 4 \nmid m) \text{, then }x^2 + 3y^2 \neq m\} \\
            & \equiv (\exists m \in \bb{Z})(\forall x \in \bb{Z})(\forall y \in \bb{Z}) \{ (2 | m) \wedge (4 \nmid m) \wedge (x^2 + 3y^2 = m\} \\ 
        \end{align*}
    So we will attempt the prove the contrary, that there exists an integer $m$ such that $2 | m$, $4 \nmid m$ and $x^2 + y^2 = m$, for any integers $x$ and $y$. Since $2 | m$ it logically follows that $x^2$ and  $y^2$ have the same parity. Hence, we will proceed by the following cases to show that $m$ is even and $4 \nmid m$: \\
   
    \textit{Case 1: Both $x^2$ and $3y^2$ are even integers} \\
        Using the definition of even integers, $x^2 = (2a)^2$ and $3y^2 = 3(2b)^2$, for some integers $a$ and $b$. Adding these integers yields:
            \begin{align*}
                x^2 + 3y^2 & = (2a)^2 + 3(2b)^2 \\
                    & = 4a^2 + 12b^2 \\
                    & = 4(a^2 + 3b^2) \\
            \end{align*}
        Since $a^2 + 3b^2$ is an integer, it follows that $4 | X^2 + 3y^2 \equiv 4 | m$ which is obviously a contradition. Because $4 \nmid m$ \\
    
    \textit{Case 2: Both $x^2$ and $3y^2$ are odd integers} \\
        Using the definition of odd integers, $x^2 = (2c + 1)^2$ and $3y^2 = 3(2d + 1)^2$, for some integers $c$ and $d$. Adding these integers yields: 
            \begin{align*}
                 x^2 + 3y^2 & = (2c + 1)^2 + 3(2d + 1)^2 \\
                    & = 4c^2 + 4c + 1 + 3(4d^2 + 4d + 1) \\
                    & = 4c^2 + 12d^2 + 4c + 12d + 4  \\
                    & = 4(c^2 + 3d^2 + c + 3d + 1)  \\
            \end{align*}
        Since $c^2 + 3d^2 + c + 3d + 1$ is an integer,  it follows that $4 | X^2 + 3y^2 \equiv 4 | m$ which is obviously a contradition. Because $4 \nmid m$ \\
        
    Because we have proven both conditional statements, we have proven that the integer $100$ cannot be written as the sum of three integers, an odd number of which are odd. 
\end{proof}
\end{example}


\newpage
\begin{example}
Prove that there is no largest positive rational number
\begin{tcolorbox}
    \begin{theorem}
        Theere is no largest positive rational number
    \end{theorem} 
\end{tcolorbox}


\begin{proof}

We will prove this result by proving the negation of the proposition which is
	\begin{center}
		There exists a largest positive rational number
	\end{center}

We assume that $\frac{a}{b}$, where $b \neq 0$, is the largest positive real number. Since $0 < \frac{a}{b} < 2(\frac{a}{b})$, it follows that $2(\frac{a}{b})$ is a positive rational number that is larger than $\frac{a}{b}$. This however is a contradiction. We have therefore proved that there is no largest positive rational number.
\end{proof}
\end{example}

\newpage
\begin{example}
Prove that there is no smallest positive irrational number
\begin{tcolorbox}
    \begin{theorem}
        There is no smallest positive irrational number
    \end{theorem} 
\end{tcolorbox}


\begin{proof}

We will prove this result by proving the negation of the proposition which is
	\begin{center}
		There exists a largest positive irrational number
	\end{center}

We assume that $a$ is the largest positive irrational number. Since $0 < \frac{a}{2}$, it follows that $\frac{a}{2}$ is a positive rational number that is smaller than $\frac{a}{2}$. This however is a contradiction. We have therefore proved that there is no smallest positive irrational number.
\end{proof}
\end{example}

\newpage
\begin{example}
Use a proof by contradiction to prove the following. Let $m \in \bb{Z}$. If $3 \nmid (m^2 - 1)$, then $3 | m$
\begin{tcolorbox}
    \begin{theorem}
        If $3 \nmid (m^2 - 1)$, then $3 | m$
    \end{theorem} 
\end{tcolorbox}

\begin{proof}
We will prove this result by proving the negation of the proposition which is
	\begin{center}
		$3 \nmid (m^2 - 1)$ and $3 \nmid m$
	\end{center}


\end{proof}
\end{example}

































\newpage
\section{Proof Structure: Disproving Existence}
\begin{definition}

Disproving Existence

Let $P(x)$ be a statement for each element $x$ in a domain $S$, We have already seen that to disprove a quantified statement  of the type $\{ \forall x \in S | P(x) \}$, it suffice to produce a counterexample (that is, an element $x$ in $S$ for which $P(x)$ is false). However, disproving a quantified statement of the type $\{ \exists x \in S | P(x) \}$ requires a totally different approach. Since
	\begin{equation}
		\neg \{ \exists x \in S | P(x) \} \equiv \{ \forall x \in S | \neg P(x) \}
	\end{equation}
it follows that the statement $\{ \exists x \in S | P(x) \}$ is false if $P(x)$ is false for every $x \in S$
\end{definition}


\begin{example}
Disprove the statement:

\begin{tcolorbox}
	\begin{theorem}
	\label{the2}		
		There exists an odd integer $n$ such that $n^2 + 2n + 3$ is odd. 
	\end{theorem}
\end{tcolorbox}

\begin{proof}
In order to disprove the proposition, we will proving it's negation stated below: \\
	\begin{center}
		For all odd integers $n$, $n^2 + 2n + 3$ is even
	\end{center}
We assume that $n$ be an odd integer and we will proceed via a direct proof to show that $n^2 + 2n + 3$ is even. By the definition of odd integers, $n = 2k + 1$, for some integer $k$. By substituting the $n$ into the expression $n^2 + 2n + 3$ yields:
	\begin{eqnarray*}
		n^2 + 2n + 3 & = & (2k + 1)^2 + 2(2k + 1) + 3 \nonumber \\
		& = & 4k^2 + 8k + 6 \nonumber \\
		& = & 2(2k^2 + 4k + 3) \nonumber \\
	\end{eqnarray*}	 
Since $2k^2 + 4k + 3$ is an integer and integers are closed under multiplication and addition. We conclude that $n^2 + 2n + 3$ is even. 
\end{proof}
\end{example}







\newpage
\section{Proof Structure: Cases}

\begin{example}
For each integer $n$, $n$ is an odd integer if and only if $n^2$ is an odd integer. \\

\begin{tcolorbox}
	\begin{theorem}
	\label{the2}		
		$n$ is an odd integer if and only if $n^2$ is an odd integer
	\end{theorem}
\end{tcolorbox}

\begin{proof}
    We assume that $n$ is a integer and will proceed by cases according to the logical equivalency of the proposition \ref{the2}: \\
    
    {\it Case 1: If $n$ is an odd integer, then $n^2$ is an odd integer } \\
    We assume that $n$ is an odd integer, and we will show that $n^2$ is odd. Using the definitions of odd integers, we see that $n = 2a + 1$ for some integer $a$. Expressing $n^2$ in terms of $n = 2a + 1$, using algebra we get
        \begin{eqnarray*}
        n^2 & = & (2a + 1)^2  \nonumber \\
        & = & 2(2a^2 + 2a) + 1 \nonumber \\
        & = & 2q + 1 \nonumber \\
        \end{eqnarray*}
    Since $a$ is an integer closed under multiplication and addition, we conclude that $q$ is an integer. Such that $n^2 = 2q + 1$ for some integer $q$, hence, $n^2$ is an odd integer. \\
    
    {\it Case 2: If $n^2$ is an odd integer, then $n$ is an odd integer} \\
    We will prove this result by proving the contrapositve of \ref{the} which is
    	\begin{center}
    		If $n$ is an even integer, then $n^2$ is even integer
    	\end{center}
    
    We assume that $n$ is an even integer, and we will show that $n^2$ is even. Using the definitions of even integers, we see that $n = 2a$ for some integer $a$. Expressing $n^2$ in terms of $n = 2a$, using algebra we get
        \begin{eqnarray*}
        n^2 & = & (2a)^2  \nonumber \\
        & = & 2(2a^2) \nonumber \\
        & = & 2q \nonumber \\
        \end{eqnarray*}
    Since $a$ is an integer closed under multiplication and addition, we conclude that $q$ is an integer. Such that $n^2 = 2q$ for some integer $q$, hence, $n^2$ is an even integer. \\
    
    Because we have proven both conditional statements, we have proven that $n$ is an odd integer if and only if $n^2$ is an odd integer.
\end{proof}
\end{example}


\newpage
\

































\newpage
\section{Proof Structure: Composite Proofs}

\begin{example}
\cite[Chap.1, P.C.1.11]{ted} \\

The {\bf Pythagorean Theorem} for right angle triangles states if $a$ and $b$ are the lengths of the legs of a right triangle and $c$ is the length of the hypotenuse, then $a^2 + b^2 = c^2$. \\
For example, if $a=5$ and $b=12$ are the lengths of the two sides of a right triangle and if $c$ is the length of the hypotenuse, then $c^2 = 5^2 + 12^2$ and so $c^2=169$. Since $c$ is a length and must be positive, we conclude that $c=13$ \\

Construct and provide a well-written proof for the proposition that { \bf If $m$ is a real number and $m$, $m+1$ and $m+2$ are the lengths of the three sides of a right triangle, then $m=3$}.\\ 


\begin{tcolorbox}
	\begin{lemma}
		The hypotenuse is the longest side of any right triangle
	\end{lemma}
\end{tcolorbox}

\begin{proof}
    We assume that the positive real numbers $a$, $b$, and $c$ are the lengths of the sides of a right triangle, such that $a$ is the length of the hypotenuse and, $b$ and $c$, the lengths of the other two sides. We will show via a direct proof that the hypotenuse is the longest side of the right triangle. \\
    
    Using Pythagoras's Theorem, we known that the relationship between the lengths of the sides of a right triangle is $a^2 = b^2 + c^2$, such that $a^2 > b^2$ and $a^2 > c^2$. \\ 
    
    Since $a > b$ and $a > c$, we can conclude that the hypotenuse is the longest side of the right triangle. 
\end{proof}

\begin{tcolorbox}
	\begin{theorem}
		If $m$ is a real number and $m$, $m+1$ and $m+2$ are the lengths of the three sides of a right triangle, then $m=3$
	\end{theorem}
\end{tcolorbox}

\begin{proof}
    We assume that $m+2$ is the hypotenuse and that $m$ and $m+1$ are the other two sides of the right triangle. We will show via a direct proof that $m=3$. \\
    
    Using the lemma above, we know that $m+2$ is the longest side of the right triangle and that the relationship between the sides of the right triangle is $(m + 2)^2 = m^2 + (m + 1)^2$. Solving for the $m$ yields: \\
    
    \begin{eqnarray}
    	0 & = & m^2 - (m+2)^2 + (m+1)^2 \nonumber \\
    	& = & m^2 - (m + 2)(m + 2) + (m + 1)(m + 1) \nonumber \\
    	& = & m^2 - (m^2 + 4m + 4) + m^2 + 2m + 1 \nonumber \\
    	& = & m^2 -2m - 3 \nonumber \\
    	& = & (m-3)(m+1) \nonumber \\ 
    \end{eqnarray}
    
    Since $m = -1,3$ and the length of sides of the triangle is a positive real number, we conclude that $m=3$. Consequently, if $m$ is a positive real number and $m$, $m+1$ and $m+2$ are the lengths of the three sides of a right triangle, then $m=3$ 
\end{proof}
\end{example}




\newpage
\begin{example}
\label{sdf1}
Source: \cite[Chap.6, S.6.2, Result 6.13]{gray} \\ 


Prove that for all integers $x \in \bb{Z}$, that every integer $n \geqq 2$, $x^n$ is even if and only if $n$ is even
    \begin{tcolorbox}
        \begin{theorem}
            For each natural number $n$,
                \begin{center}
                    Let $x \in \bb{Z}$. For every integer $n \geqq 2$, $x^n$ is even if and only if $n$ is even
                \end{center}
        \end{theorem}
    \end{tcolorbox}

    \begin{proof}
        We will prove this biconditional propostion by proving the following two conditional statements: \\
            \begin{enumerate}
                \item Let $x \in \bb{Z}$. For every integer $n \geqq 2$, if $x^n$ is even, then $x$ is even
                \item Let $x \in \bb{Z}$. For every integer $n \geqq 2$, if $n$ is even, then $x^n$ is even
            \end{enumerate}
        
        \textbf{For the backwards direction}, we assume that $x$ is even integer. And, we will show via a direct proof that $x^n$ is an even integer. Using the definition of integers, $x = 2y$, for some integer $y$. Hence, 
                \begin{equation*}
                    x^n = x \m x^{n-1} = (2y) \m x^{n-1} = 2(y \m x^{n-1})
                \end{equation*}
            Since $y \m x ^{n-1}$ is an integer, we conclude that $x^n$ is even.
    
        \textbf{For the forward direction}, we will use a proof by the First Principle of Mathematical Induction. For each natural number $n$, we let $P(n)$ be
                \begin{center}
                    If $x^n$ is even, where $n \geqq 2$, then $x$ is even
                \end{center}
            We first prove that $P(2)$ is true. Notice that since $2$ is even, it follows from the bavkwards direction proof that $x^2$ is even. The basis step has been established.   
            
            For the inductive step, we prove that for all $k \in \bb{N}$ with $k \geqq 2$, if $P(k)$, then $P(k+1)$. So let $k$ be a natural number and assume that $P(k)$ is true. That is, we assume that 
                \begin{center}
                    If $x^k$ is even, where $k \geqq 2$, then $x$ is even
                \end{center}
            
            The goal is to prove that $P(k+1)$ is true. That is, it must be proved that  
                \begin{center}
                    If $x^{k+1}$ is even, where $k+1 \geqq 2$, then $x$ is even   
                \end{center}
            
            To do this, we will assume that $x^{k+1}$ is even. And, we will show via a direct proof that $x$ is even. Using the definition of integers, $x = 2w$, for some integer $w$. Hence, 
                \begin{equation*}
                    x^{k+1} = x \m x^k = 2w \m x^k = 2(w \m x^k)    
                \end{equation*}
            Since $w \m x^k$ is an integer, we conclude that $x$ is even. 
        Hence, the inductive step has been established, and by the First Principle of Mathematical Induction, we have proven that for each natural number $n \geqq 2$,
            \begin{center}
                Let $x \in \bb{Z}$. For every integer $n \geqq 2$, $x^n$ is even if and only if $n$ is even
            \end{center}
    \end{proof}
\end{example}




























\begin{example}
We known that an integer $x$ is even if $x = 2q$, for some integer $q$, while $y$ is odd if $y = 2q + 1$ for some integer $q$. Furthermore, the two integers $x$ and $y$ are of the same parity if they both even or are both odd. From this, it follows that $x$ and $y$ are of the same parity if and only if $2 | (x-y)$

\begin{tcolorbox}
	\begin{theorem}
		$a$ and $b$ are of the same parity if and only if $2 | (a - b)$
	\end{theorem}
\end{tcolorbox}

\begin{proof}
We will prove this biconditional statement by proving the following two conditional statements: 
1. $a$ and $b$ are of the same parity if and only if $2 | (a - b)$




We assume $a$, $b$ are integers with $a \neq 0$ and will proceed by cases according to whether both $a$ and $b$ are odd or even. 

{\it Case 1: If $a$ and $b$ are even integers then $2 | (a-b)$} \\
Suppose there exists even integers $a = 2x$ and $b = 2y$, where $x, y \in \bb{Z}$. Substituting these expressions into $a-b$ yields:
\begin{eqnarray*}
	a-b & = & 2x - 2y \nonumber \\	
	& = & 2(x-y) \nonumber \\
\end{eqnarray*}
Since $x-y$ is an integer because integers are closed under subtraction, we conclude that $2 | (a - b)$ \\


{\it Case 2: If $a$ and $b$ are odd integers then $2 | (a-b)$} \\
Suppose there exists odd integers $a = 2x + 1$ and $b = 2y + 1$, where $x, y \in \bb{Z}$. Substituting these expressions into $a-b$ yields:
\begin{eqnarray*}
	a-b & = & 2x + 1 - 2y - 1 \nonumber \\	
	& = & 2(x - y - 1) \nonumber \\
\end{eqnarray*}
Since $x-y-1$ is an integer because integers are closed under subtraction, we conclude that $2 | (a - b)$. Becuase we have proven both conditional statements, we have proven that $2 | (a-b)$ if and only if $a$ and $b$ have the same parity. \\
\end{proof}



\begin{tcolorbox}
NOTICE CONGRUENCE PROPERTY

Consequently, $2 | (a -b)$ if and only if $a$ and $b$ have the same remainder when divided by $2$. For integers $a$ and $b$ and $n \geqq 2$, we say that $a$ is {\bf congruent to $b$ modulo $n$}, written $a \equiv b (\text{mod n})$, if $n | (a - b)$. \\
\end{tcolorbox}
\end{example} 













































\newpage
\chapter{TOPICS IN SET THEORY}
\section{Basic Definitions and Notations}

\begin{definition}
Set \\

    \begin{tcolorbox}
        A {\bf set}\footnotemark is a well-defined collection of objects that can be thought of as a single entity itself. The objects in the set are called {\bf elements} of the set \\
    \end{tcolorbox}
    
    \footnotetext{
    We  will adopt the convention that capital letters are used to denote the name of sets, where as lowercase letters denote ojbects viewed as possible elements of sets. Furthermore, the expression $a \in A$ represents the assertion that the object $a$ is an  element of the set $A$; while, the expression $b \notin B$ represents the assertion that the object $b$ is not an element of the $B$.} 
\end{definition}


\begin{definition}
Variables, Constants and the Universal Set \\

    \begin{tcolorbox}
        A {\bf constant} is a specific member of the universal set $U$.
    \end{tcolorbox}
    
    \begin{tcolorbox}
        A {\bf variable} is a symbol that holds a place for a constant from a given universal set $U$.
    \end{tcolorbox}

    \begin{tcolorbox}
        The set $U$ is called the {\bf universal set for the variable}. It is the set of specified objects which objects may be chosen to substitute for the variable. The universal set is sometimes called the {\it sample space}.
    \end{tcolorbox}
    
    In other words, a constant is a name of a particular thing. We say that a constant \textit{names} or \textit{denotes} the thing of which it is a name. For instance, the expressions "$1 + 1$" and "$-2 + 5 - \frac{8}{5} + \frac{6}{3} + 1$" are constants and both denote the number two. Continuing with the example, "$2$" is also a constant. It is the name of a mathematical object - a number.
\end{definition}


\newpage
\begin{definition}
Uniquness of Sets  \\

    \begin{tcolorbox}
        The expression "$x=y$" means that $x$ and $y$ are the same object. The symbol "$=$" is called \textbf{equals}. "$\neq$" means that $x$ and $y$ are not the same object.         
    \end{tcolorbox}
\end{definition}



\begin{definition}
Elementhood  \\

    \begin{tcolorbox}
        If an object $x$ is a member of a set $A$, we say that $x$ is an \textbf{element} of $A$ and write
            \begin{equation*}
                x \in A
            \end{equation*}
    \end{tcolorbox}
    
    For instance, the integer $1$ is an element of the set $\bb{Z}$; therefore, we write $1 \in \bb{Z}$. \\
    
    \begin{tcolorbox}
        If an object $u$ is a not member of a set $B$, we say that $x$ is an \textbf{not an element} of $B$ and write
            \begin{equation*}
                x \notin A
            \end{equation*}
    \end{tcolorbox}
    An thus the rational $\frac{1}{2}$ is not an element of the set $\bb{Z}$, similarly, we write $\frac{1}{2} \notin \bb{Z}$ \\
    
    We shall have occasions to refer to sets which have a single element. For this purpose, we shall use the notation "$\{ x \}$"; thus $\{ x \}$ is understood as the set containing the element $x$, and no others. More precisely, 
        \begin{equation*}
            \{ x \} = \{U | U = x \}    
        \end{equation*}
    
    Similarly, 
        \begin{equation*}
            \{ x,y \} = \{U | (U = x) \vee (U = y) \}             
        \end{equation*}
    Now, let consider what happens when a given constant has different names. If $x=y$ (i.e. if $x$ and $y$ are variables for the same constant) then
        \begin{equation*}
            \{ x,y \} = \{ x \} = \{y \}
        \end{equation*}
    Similarly, if $x=z$ and $v=w-y$, then
        \begin{equation*}
            \{x,y,v,w \} = \{x,y \} = \{x, w \} = \{z,y \} = \text{ etc.}
        \end{equation*}
\end{definition}



\begin{definition}
Empty Set \\
\begin{tcolorbox}
    The solution set containing no elements is called the {\bf empty set}. For example, the set of all rational numbers that are solutions of the equation $x^2=-2$ is the empty set since this equation has no solution that are rational numbers. We can symbolically represent the empty set as $\{ \}$ or $\emptyset$ \\
\end{tcolorbox}
\end{definition}

\begin{definition}
Roster Method of Describing Sets \\

    \begin{tcolorbox}
        We describe a set by listing the names of its elements, separated by commas, with the full list enclosed in braces. Thus $A = \{1,2,3,4\}$ or $B = \{cos(x), 2x+y^2 \}$ are sets consisting of four and two elements, respectively, described by the \textbf{rooster method}. Note that $2 \in A$ and $cos(x) \in B$ but $2x+y^2 \notin A$ and $2 \notin B$. Two important facts are:
        \begin{enumerate}
            \item The order in which the elements are listed is irrelevant
            \item An object should be listed only once in the roster, since listing it more than once does not change the set
        \end{enumerate}
    
        As an example, the set $\{1,1,2\}$ is the same as the set $\{1,2\}$, which in turn, is the same as $\{2,1\}$
    \end{tcolorbox}

\end{definition}


\begin{definition}
Set Building Method of Describing Sets \\
    
    \begin{tcolorbox}
    We describe a set in terms of one or more properties to be satisfied by obects in the set, and by those objects only. Such a description is formulated in so-called {\bf Set Building Notation}, that is in the form 
    	\begin{center}
    		$A = \{ x | x \text{satisfies some property or properties} \}$
    	\end{center}
    which reads "A is the set of all objects x such that x satisfies ..." Typical representation of a set by the set building method is:
    	\begin{center}
    		$C = \{ x | (x \in \bb{N})\wedge(x \leqq 100) \}$
    	\end{center}	
    \end{tcolorbox}
    
    An important connect between the roster method and the set building notation. In a  number of mathematical situations {\it solving a problem} means essentially to convert a description of a set by the set building method into a rooster method description. In this context, we often refer to the rooster representation as the {\bf solution set} of the original problem.  
\end{definition}




\begin{example}
Set Builder Notation \cite[Chap.2, P.C.2.15]{ted} \\

Each of the following sets is defined using rooster method: \\
    
    \begin{enumerate}
        \item $A = \{1, 5, 9, 13, \cdots \}$
        \item $B = \{\sqrt{2},(\sqrt{2})^3, (\sqrt{2})^5, \cdots \}$
        \item $C = \{\cdots, -8, -6, -4, -2, 0 \}$ 
        \item $D = \{1, 3, 9, 27,  \cdots \}$
    \end{enumerate}
    
    
    Determine four more elements of each set and use set builder notation to describe each set. \\
    \begin{enumerate}
        \item $A = \{1, 5, 9, 13,17, 21, 21, 25, 29 \cdots \} = \{ x \in \mathbb{Z^+}| 4x + 1\}$ 
        \item $B = \{\sqrt{2},(\sqrt{2})^3, (\sqrt{2})^5, (\sqrt{2})^7, (\sqrt{2})^9, (\sqrt{2})^11, \cdots  \} = \{ x \in \mathbb{Z^+}|  (\sqrt{2})^{2n + 1} \}$ 
        \item $C = \{\cdots, -16, -14, -12, -10, -8, -6, -4, -2, 0 \} = \{ x \in \mathbb{Z^-}| 4x \}$ 
        \item $D = \{1, 3, 9, 27, 81, 243, 729, 2187  \cdots \} = \{ x \in \mathbb{Z^+}| 4^x \}$ 
    \end{enumerate}
\end{example}



\begin{definition}
Solution Set \\

    \begin{tcolorbox}
        The {\bf solution set} (truth set) of an open sentence with on variable is the collection of objects in the universal set that can be substituted for the variable to make the propositional function (predicate) a true statement. 
    \end{tcolorbox}
    
    For instance, if the universal set is $\bb{R}$, then the solution set of the equation $x^2 - 3x - 10 = 0$ is $[-2, 5]$ 
\end{definition}
    


\begin{definition}
The Number Systems \\

The number systems are are special sets: \\

    \begin{tcolorbox}
    \begin{itemize}
        \item Natural Number System: $\bb{N} = \{ 1, 2, 3, \cdots \}$ of all positive integers 
    	
    	\item Integer Number System: $\bb{Z} = \{0, \pm 1, \pm 2, \pm 3, \cdots \} = {\bb{N}}^- \vee \{0\} \vee \bb{N}$ of all integers
    	
    	\item Rational Number System: $\mathbb{Q} = \{ \frac{m}{n} | (m,n \in \bb{Z}) \wedge (n \neq 0)\}$ is the set of all rational number (quotients of integers) 
    	
    	\item Real Number System: $\bb{R} = \mathbb{Q} \vee {\mathbb{Q}}^c$ is the set $\{ 1, 2, 3, \cdots \}$ of all positive integers 
    	
    	\item  Complex Number System: $\bb{C} = \{ a+bi | (a,b \in \bb{R})\wedge i=\sqrt{-1}$
    \end{itemize}
    \end{tcolorbox}
\end{definition}


\begin{definition}
Intervals \\

A set $I$, all of whose elements are real numbers is called an {\bf interval} if and only if whenover $a$ and $b$ are elements of $I$ and $c$ is a real number within $a<c<b$ then $c \in I$ \\

The nine types of intervals are described by the following terminology and notation, in which $a$ and $b$ denote real numbers: \\
\begin{tcolorbox}
    \begin{enumerate}
        \item $\{x \in \bb{R} | a \leqq x \leqq b \}$, a closed and bounded interval, denoted $[a,b]$ 
        
        \item $\{x \in \bb{R} | a < x < b \}$, a open and bounded interval, denoted $(a,b)$ 
        
        \item $\{x \in \bb{R} | a \leqq x < b \}$, a closed-open and bounded interval, denoted $[a,b)$ 
        
        \item $\{x \in \bb{R} | a < x \leqq b \}$, a open-closed and bounded interval, denoted $(a,b]$ 
        
        \item $\{x \in \bb{R} | a \leqq x \}$, a closed and unbounded above interval, denoted $[a, \infty)$ 
        
        \item $\{x \in \bb{R} | a < x \}$, a open and unbounded above interval, denoted $(a, \infty)$ 
        
        \item $\{x \in \bb{R} | x \leqq b \}$, a closed and unbounded below interval, denoted $(- \infty ,b]$ 
        
        \item $\{x \in \bb{R} | x < b \}$, a open and unbounded below interval, denoted $(- \infty ,b)$
        
        \item $\bb{R}$ itself is open and unbounded interval, denoted $(- \infty, \infty)$ \\
    \end{enumerate}
\end{tcolorbox}
\end{definition}






\newpage
\section{Arithmetic of Sets}

\begin{definition}
Set Theoretic Equality \\
    \begin{tcolorbox}
        Two sets, $A$ and $B$, are {\bf equal} when they have precisely the same elements. In this case, we write $A = B$. When the sets $A$ and $B$ are not equal, we write $A \neq B$. 
    \end{tcolorbox}
\end{definition}
    

\begin{definition}
Subset, Containment, and the Superset \\
    
    \begin{tcolorbox}
        The set $A$ is a {\bf subset} of a set $B$ provided that each element of $A$ is an  element of $B$. In this case, we write $A \subseteq B$ and also say that $A$ is {\bf contained} in $B$. When $A$ is not a subset of $B$, we write $A \nsubseteq B$. Finally, we define $B$ is a \textbf{Superset} of $A$ to mean $A \supseteq B$ 
    \end{tcolorbox}
    
    Here are two examples of important subset relationships: 
        \begin{enumerate}
            \item Number System: $\bb{N} \subseteq \bb{Z} \subseteq \mathbb{Q} \subseteq \bb{R} \subseteq \bb{C}$ 
            
            \item Functional Subspace: $\{\text{Polynomial Functions}\} \subseteq \{\text{Differentiable Functions}\} \subseteq \{\text{Continuous Functions}\} \subseteq \{\text{Integrable Functions}\} \subseteq \{\text{Real-Valued Functions}\}$
        \end{enumerate}
    
    The subset relations, like set theoretic equality, enjoy the reflexive and transitive properties. Put more directly, every set is a subset of itself and for any sets $A$, $B$, and $C$:
        \begin{center}
            If $A \subseteq B$, $B \subseteq C$, then $A \subseteq C$
        \end{center}
    The subset relation is not symmetric. \\
    
    Since \textit{elementhood} and \textit{subset} are both relationships of containment, it is important not to confuse the two. For instance, the set $A = \{ 1, 2 , 3 \}$ contains $\{ 2, 3 \}$ and contains $2$, but in different senses. 
    
    In general, the subset relation is described with the use of a universal quantifier since $A \subseteq B$ means that for each element $x$ of $U$, if $x \in A$, then $x \in B$, so when we negate this, we use an existential quantifier  as follows: 
    
    \begin{align*}
    A \subseteq B & \equiv (\forall x \in U) [(x \in A) \to (x \in B)] \nonumber \\
    A \nsubseteq B & \equiv \neg (\forall x \in U) [(x \in A) \to (x \in B)] \nonumber\\
           & \equiv (\exists x \in U) \neg [(x \in A) \to (x \in B)] \nonumber\\
           & \equiv (\exists x \in U) [(x \in A) \wedge (x \notin B)] \nonumber\\
    \end{align*}
    so we see that $A \nsubseteq B$ means  that there exists an $x$ in $U$ such that $x \in A$ and $x \notin B$. \\
    
    Notice that if $A = \emptyset$, the the conditional statement $ \emptyset \subseteq B = (\forall x \in U) [(x \in \emptyset) \to (x \in B)]$ must be true since the hypothesis will always be false, and that  $B \subseteq B = (\forall x \in U) [(x \in B) \to (x \in B)]$

\end{definition}


\begin{definition}
Proper Subset \\

    \begin{tcolorbox}
        Let $A$ and $B$ be sets. We say that $A$ is a \textbf{Proper Subset} of $B$, denoted $A \subset B$, if and only if $A \subseteq B$ but $A \neq B$. We write $A \nsubseteq B$ to symbolize the statement that $A$ is not a proper subset of $B$ (which could mean that either $A \nsubseteq B$ or $A = B$)
    \end{tcolorbox}
    \end{definition}
    
    
\begin{definition}
Power Set \\
    
    \begin{tcolorbox}
    Let $A$ be a set. We denote by $\mathbb{P}(A)$, the \textbf{power set} of $A$, the set of all subsets of $A$
    
    Let $A$ be a subset of a universal set $U$, then the set whose members are all the subsets of $A$ is called the \textbf{Power Set} of $A$. We denote the power set of $A$ by $\bb{P}(A)$. Symbolically,
        \begin{center}
            $\mathbb{P}(A) = \{ X \subseteq U | X \subseteq A \}$ 
        \end{center}
    That is, $X \in \bb{P}(A)$ if and only if $X \subseteq A$. 
    \end{tcolorbox}

\end{definition}


\begin{definition}
Cardinality of Finite Set \\

    \begin{tcolorbox}
        The number of elements in a finite set is called the \textbf{Cardinality} of the set, denoted card$(A)$ or $|A|$
    \end{tcolorbox}
    For $A$ is a finite subset some universal set. We can define the cardinality of the power set of $A$ as: 
        \begin{equation*}
            card[\bb{P}(A)] = 2^n
        \end{equation*}
        Where $n$ is the number of unique element of the set $A$
    Where $|A|$ is the cardinal number of the finite set $A$, i.e. the number of elements in $A$.
\end{definition}

\newpage
\begin{theorem}
Properties of Subsets \\

    \begin{tcolorbox}
        For any sets $A, B$, and $C$ in any universal set $U$ \\

        \begin{enumerate}
            \item $\emptyset \subseteq A$ %Proof Completed
            
            \item $A \subseteq U$  
            
            \item $A \subseteq A$ \textit{Reflexive Property of Subset Relation} %Proof Completed
            
            \item $A = B \equiv (A \subseteq B) \cap (B \subseteq A)$ \textit{Antisymmetric Property of Subsets} %Proof Completed
            
            \item If $(A \subseteq B) \cap (B \subseteq\ C)$, then $A \subseteq C$ \textit{Transitive Property of Subsets}  %Proof Completed

            \item If $A \subseteq B$ and $C \subseteq D$, then $A \cup C \subseteq B \subseteq B \cup D$ Set Union Preserve Subset
            
            \item $A \subseteq (A \cup B)$ Union is a Superset % Proof Completed
            
            \item $A \subseteq B \equiv A \cup B = B$ Union with a Superset is a Superset  %Proof Completed
            
            \item $(A \cap B) \subseteq A$ Intersection is a Subset % Proof Completed
            
            \item $A \subseteq B \equiv A \cap B = A$ Intersection with a Subset is a Subset %Proof Completed
        \end{enumerate}
    \end{tcolorbox}




    \newpage
    \begin{property}
    .\\
        \begin{tcolorbox}
            For any sets $A, B$, and $C$ in any universal set $U$
                \begin{equation*}
                    \emptyset \subseteq A
                \end{equation*}
        \end{tcolorbox}
    
        \begin{proof}
            Let $A$ be a subset of some universal set. By the definition of subset, for every $x \in U$, if $x \in \emptyset$, then $x \in A$. but the antecedent $x \in \emptyset$ is false for any element $x$, so the conditional is vacuously true for any $x$, regardless of the true value of $x \in A$. Hence, the we conclude that $\emptyset \subseteq A$. 
        \end{proof}
    \end{property}
    
    
    \begin{property}
    Reflexive Property of Subset Relation  \\
        \begin{tcolorbox}
            For any sets $A, B$, and $C$ in any universal set $U$
                \begin{equation*}
                    A \subseteq A
                \end{equation*}
        \end{tcolorbox}
    
        \begin{proof}
            Let $A$ be a subset of some universal set, $U$. By the definition of subsets, for every $x \in U$, if $x \in A$, then $x \in A$. But the predicate $(x \in A) \to (x \in A)$ has the form $p \to p$ for any substitution, since $p \to p$ is a tautology. Hence, we conclude $A \subseteq A$
        \end{proof}
    \end{property}
    
    
    \newpage
    \begin{property}
    Antisymmetric Property of Subsets  \\
        \begin{tcolorbox}
            For any sets $A, B$, and $C$ in any universal set $U$
                \begin{equation*}
                    A = B \equiv (A \subseteq B) \cap (B \subseteq A)
                \end{equation*}
        \end{tcolorbox}
    
        \begin{proof}
            Let $A$ and $B$ be subsets of some universal set, $U$, where $x \in U$. By the set definitions of $A = B$ and $(A \subseteq B) \cap (B \subseteq A$, we see that, 
                \begin{equation*}
                    (\forall x)[(x \in A)\equiv (x \in B)] \equiv (\forall x)[(x \in A \to x \in B) \cap (x \in B \to x \in A)]
                \end{equation*}
            Using the propositional logic tautology $(p \leftrightarrow q) \leftrightarrow (p \to q) \wedge (q \to p)$ and the theorem $(\forall x)[p(x) \wedge q(x)] \leftrightarrow (\forall x)[p(x)] \wedge (\forall x)[q(x)]$ from predicate logic, we observe that
                \begin{align*}
                    (\forall x)[(x \in A)\equiv (x \in B)] & \equiv (\forall x)[(x \in A \to x \in B) \cap (x \in B \to x \in A)] \\
                    & \equiv (\forall x)[(x \in A \to x \in B)] \cap (\forall x)[(x \in B \to x \in A)] \\
                \end{align*}
            Hence, we conclude that $$ A = B \equiv (A \subseteq B) \cap (B \subseteq A) $$
        \end{proof}
    \end{property}
    
    
    
    \newpage
    \begin{property}
    Transitive Property of Subsets  \\
        \begin{tcolorbox}
            For any sets $A, B$, and $C$ in any universal set $U$
                \begin{equation*}
                    (A \subseteq B) \cap (B \subseteq\ C)\text{, then } A \subseteq C
                \end{equation*}
        \end{tcolorbox}
    
        \begin{proof}
            Let $A$ and $B$ be subsets of some universal set, $U$, where $x \in U$. By the set definitions of $(A \subseteq B) \cap (B \subseteq\ C)$ and $A \subseteq C$, we see that, 
                \begin{equation*}
                    (\forall x)[(A \subseteq B) \cap (B \subseteq\ C)] \to (\forall x)[A \subseteq C]
                \end{equation*}
            Let $x \in (A \subseteq B) \cap (B \subseteq\ C)$. This mean that $x \in A \subseteq B$ and $x \in B \subseteq C$. Since $x \in U$, suppose $x \in A$. Using logical principle \textbf{modus ponens} \footnotemark if $x \in A$ and $x \in A \subseteq B$, then $x \in B$. And $x \in B$ and $x \in B \subseteq C$ implies $x \in C$. Thus $x \in A$ is sufficient for $x \in C$. Therefore, we conclude $$ (A \subseteq B) \cap (B \subseteq\ C)\text{, then } A \subseteq C $$.
                \footnotetext{\textbf{Modus Ponens} $[p \wedge (p \to q)] \to q$}
        \end{proof}
    \end{property}
    
    
    
    
    
    \newpage
    \begin{property}
    Set Union Preserve Subset \\
        \begin{tcolorbox}
            For any set $A, B, C$ and $D$ in any universal set $U$
                \begin{equation*}
                    \text{If } A \subseteq B \text{and } C \subseteq D \text{, then } A \cup C \subseteq B \subseteq B \cup D
                \end{equation*}
        \end{tcolorbox}
    
        \begin{proof}
            Let  $A, B, C$ and $D$ be subsets of some universal set. And let $A \subseteq B$ and $C \subseteq D$. By the definition of subsets, we known that if $x \in A$, then $x \in B$ and if $x \in C$, then $x \in D$. \\
            We now invoke the \textit{Constructive Dilemma} \footnotemark. Using the fact that $x \in A$ implies $x \in B$ and $x \in C$ imples $x \in D$, applying the dilemma yields that if $x \in A \cup C$, then $x \in B \cup D$. And from the definition of subset, we conclude that if $A \subseteq B$ and $C \subseteq D$, then $A \cup C \subseteq B \cup D$
        
                \footnotetext{\textbf{Constructive Dilemma} of propostional logic can take any of the following three equivalent forms: 
                    \begin{itemize}
                        \item $p \to q$, $r \to s$ $\vdash p \vee r \to q \vee s$
                        
                        \item $\vdash [(p \vee r) \wedge (p \to q) \wedge (r \to s)] \to (q \vee s)$
                        
                        \item $(p \to q)\wedge(r \to s) \wedge (p \vee r) \vdash (q \vee s)$
                    \end{itemize}}
        \end{proof}
    
    \smallskip
    
    \begin{corollary}
    Set Union Preserve Subset\\
        \begin{tcolorbox}
            For any set $A, B$ and $C$ in any universal set $U$
                \begin{equation*}
                    \text{If } A \subseteq B \text{, then } A \cup C \subseteq B \subseteq B \cup C
                \end{equation*}
        \end{tcolorbox}
    
        \begin{proof}
            Let  $A, B$ and $C$ be subsets of some universal set and let $A \subseteq B$. By the definition of subsets, we known that if $x \in A$, then $x \in B$. Now, suppose we have $x \in C$. By the reflexive property of subsets, $x \in C \subseteq C$. This means that if $x \in C$ then $x \in C$. \\ 
            We now invoke the \textit{Constructive Dilemma}. Using the fact that $x \in A$ implies $x \in B$ and $x \in C$ imples $x \in C$, applying the dilemma yields that if $x \in A \cup C$, then $x \in B \cup C$. And from the definition of subset, we conclude that if $A \subseteq B$, then $A \cup C \subseteq B \cup C$
        \end{proof}
    \end{corollary}
    
    \end{property}
    
    
    \newpage
    \begin{property}
    Union is a Superset  \\
        \begin{tcolorbox}
            For any sets $A, B$, and $C$ in any universal set $U$
                \begin{equation*}
                    A \subseteq (A \cup B)
                \end{equation*}
        \end{tcolorbox}
    
        \begin{proof}
            Let $A$ be a subset of some universal set and $x \in A$. By the law of simplification for logical addition \footnotemark, we see that if $x \in A$, then $x \in A \cup B$. Therefore, by the definition of subsets, $A \in A \cup B$
            
            \footnotetext{The \textbf{Law of Simplification for Logical Addition} (also known as the Rule of Addition) is a valid deduction sequent in propositional logic. \\
            \begin{flushleft} Proof Rule \end{flushleft}
            The \textbf{Rule of Addition} can be symbolised by the conditional statements
                \begin{itemize}
                    \item If we can conclude $A$, then we may infer $A \vee B$
                    \item If we can conclude $B$, then we may infer $A \vee B$
                \end{itemize}
                
            \begin{flushleft} Sequent Rule \end{flushleft}
            The \textbf{Rule of Addition} can be symbolized by the sequents    
                \begin{itemize}
                    \item $A \vdash A \vee B$
                    \item $B \vdash A \vee B$
                \end{itemize}
            Note: A sequent is an expression in the form $A_1, A_2,, \cdots A_n \vdash B$. Where, $A_1, A_2,, \cdots A_n$ are premises (any number of them), and $B$ the conclusion (only one) of an argument.}
        \end{proof}
    \end{property}
    
    
    \newpage
    \begin{property}
    Union with a Superset is a Superset  \\
        \begin{tcolorbox}
            For any sets $A, B$, and $C$ in any universal set $U$
                \begin{equation*}
                    A \subseteq B \equiv A \cup B = B
                \end{equation*}
        \end{tcolorbox}
    
        \begin{proof}
        Let $A$ be a subset of some universal set, $U$. By the antisymmetric property of subsets, we will prove this set equality $A \subseteq B \equiv A \cup B = B$ by division of cases: \\
            
            
            \begin{flushleft} \textbf{Case 1: $A \cup B = B \to A \subseteq B$} \end{flushleft}
                We will proceed to prove this implication using division of cases: 
                
                \begin{flushleft} \textit{Case 1 - Subcase 1: $A \cup B \subseteq B \to A \subseteq B$} \end{flushleft}
                    We see using the definition of subsets that if $x \in A \cup B$, then $x \in B$. Since $x \in A \cup B$, we know that $x \in A \subseteq (A \cup B)$, so if $x \in A$, then $x \in A \cup B$. Hence, by the transitive property of subsets, since $x \in A \subset A \cup B$ and $x \in A \cup B \subseteq B$, it follows that $x \in A \subseteq B$. This completes the proof that $$A \cup B \subseteq B \to A \subseteq B$$
                    
                \begin{flushleft} \textit{Case 1 - Subcase 2: $B \subseteq A \cup B \to A \subseteq B$} \end{flushleft} 
                    For some $x \in U$, using set algebra, the proposition becomes
                        \begin{align*}
                            [x \in B \subseteq A \cup B] & \to [x \in A \subseteq B] \\
                            [x \in B \to A \cup B & \to [x \in A \subseteq B]  & \text{Def. Subset} \\
                            [x \in A \cup B \cup \overline{B}] & \to [x \in A \subseteq B]  & \text{Logical Equivalence} \\
                            [x \in A \cup (B \cup \overline{B})] & \to [x \in A \subseteq B]  & \text{By Associativity} \\
                            [x \in A \cup U] & \to [x \in A \subseteq B]  & \text{By Associativity} \\   
                            [x \in U] & \to [x \in A \subseteq B]  \\ 
                        \end{align*}
                    Since $x \in U$, it must be that $x \in A \subseteq B$. This completes the proof that $$ B \subseteq A \cup B \to A \subseteq B $$
                    
                    
            \begin{flushleft} \textbf{Case 2: $A \subseteq B \to A \cup B = B$} \end{flushleft}
                We will proceed to prove this implication using division of cases:   
                \begin{flushleft} \textit{Case 2 - sub.case 1: $A \subseteq B \to A \cup B \subseteq B$} \end{flushleft}
                    We will proceed to show that this statement is trivially true. Let $x \in A \cup B \subseteq B$. This means that $x \in A \cup B \subseteq B$ if $x \in A \subseteq B$. Since $x \in A \cup B$. we know that $x \in A$ or $x \in B$. Now, we will proceed by division of cases:
                        \begin{flushleft} \text{Case 2 - sub.case 1 - part 1: $y \in A$} \end{flushleft}
                            Let $x \in A$ and $x \in A \subseteq B$. By \textit{modus ponens}, since $x \in A$ and $x \in A \subseteq B$, it follow that $x \in B$. Hence, its trivially true that $x \in A \cup B$ is sufficient for $x \in B$
                    
                        \begin{flushleft} \text{Case 2 - sub.case 1 - part 2: $y \in B$} \end{flushleft}
                            Let $x \in B$ and $x \in A \subseteq B$. So, we are assured it's trivially true that if $x \in A \subseteq B$, then $x \in B$   \\
                    This completes the proof that $$ A \subseteq B \to A \cup B \subseteq B $$
                    
                \begin{flushleft} \textit{Case 2 - sub.case 2: $A \subseteq B \to B \subseteq A \cup B$} \end{flushleft}
                    We will proceed to show that this statement is trivially true. Let $x \in B \subseteq A \cup B$. This means that $x \in B$ implies $x \in A \cup B$. Now, by the \textit{Simplification Law of Logical Addition}, we know that $x \in A \cup B$ is necessary for $x \in B$. This completes the proof that $$ A \subseteq B \to B \subseteq A \cup B $$ \\
            Since, we prove the division of cases exhausively, we conclude that $$ A \subseteq B \equiv A \cup B = B $$
        \end{proof}
    \end{property}
    
    
    
    \newpage
    \begin{property}
    Intersection is a Subset  \\
        \begin{tcolorbox}
            For any sets $A, B$, and $C$ in any universal set $U$
                \begin{equation*}
                    (A \cap B) \subseteq A
                \end{equation*}
        \end{tcolorbox}
    
        \begin{proof}
            Let $A$ and $B$ be subsets of some universal set. By the definition of intersection, we see that if $x \in A \cap B$, then $x \in A$ and $x \in B$. Now, using the rule for simplification for logical multiplication \footnotemark, we see that $x \in A$. Therefore, if $x \in A \cap B$ then $x \in A$; using the definition of subsets, conclude that $A \cap B \in A$.
            
            \footnotetext{The \textbf{Law of Simplification for Logical Multiplication} (also known as the Rule of Multiplication) is a valid deduction sequent in propositional logic. \\
            \begin{flushleft} Proof Rule \end{flushleft}
            The \textbf{Rule of Multiplciation} can be symbolised by the conditional statements
                \begin{itemize}
                    \item If we can conclude $A \wedge B$, then we may infer $A$
                    \item If we can conclude $A \wedge B$, then we may infer $B$
                \end{itemize}
                
            \begin{flushleft} Sequent Rule \end{flushleft}
            The \textbf{Rule of Multiplication} can be symbolized by the sequents    
                \begin{itemize}
                    \item $A \wedge B \vdash A$
                    \item $B \wedge B \vdash B$
                \end{itemize}}
        \end{proof}
    \end{property}
    
    
    \newpage
    \begin{property}
    Intersection with a Subset is a Subset  \\
        \begin{tcolorbox}
            For any sets $A, B$, and $C$ in any universal set $U$
                \begin{equation*}
                    A \subseteq B \equiv A \cap B = A
                \end{equation*}
        \end{tcolorbox}
    
        \begin{proof}
            Let $A$ and $B$ be a subset of some universal set, $U$. By the antisymmetric property of subsets, we will prove this set equality by division of cases:
            
            \begin{flushleft} \textbf{Case 1: $A \cap B = A \subseteq (A \subseteq B)$} \end{flushleft}
                
                We will proceed to prove that the implication, if $A \cap B = A$, then $A \subseteq B$ is trivally true. Let $x \in A \subseteq B$. This means that if $x \in A$ then $x \in B$. Since $xx \in A$ and $x \in B$, it must be that $x \in A \cap B$. It follows that $x \in A \cap B$ s sufficient for $x \in A \cap B \subseteq B$. \\
                
                Using the definition of subsets, we see that $x \in A \cap B \subseteq B$ means $x \in B$ if $x \in A \cap B$. Hence, whenever $x \in A$, then $x \in B$. This completes the proves that $$ A \subseteq B \subseteq A \cap B = A $$
            
            \begin{flushleft} \textbf{Case 2: $(A \subseteq B) \subseteq A \cap B = A$} \end{flushleft} 
                We will proceed to prove = the implication, $A \subseteq B$ imples $A \cap B = A$, by division of cases:
            
                    \begin{flushleft} \textbf{Case 2 - Subcase 1: $(A \subseteq B) \subseteq A \cap B \subseteq A$} \end{flushleft}     
                        We will proceed to prove that the proposition $A \subseteq B) \subseteq A \cap B \subseteq A$ is trivially true. This means that if $x in A \subseteq B$, then $x \in A \cap B \subseteq A$. By the Law of Simplification of Logical Multiplication, we see that $x \in A \cap B$ imples $x \in A$. This completes the proof that $$ A \subseteq B) \subseteq A \cap B \subseteq A $$  
       
                    \begin{flushleft} \textbf{Case 2 - Subcase 2: $(A \subseteq B) \subseteq A \subseteq A \cap B $} \end{flushleft}     
                        Let $x \in A \subseteq B$. By the definition of subsets, $x \in A$ imples $x \in B$. Since the element $x$ is in both sets, it must be that $x \in A \cap B$. Consequently $x \in A \subseteq A \cap B$ is necessary for $x \in A \subseteq B$. This completes the proof that $$ (A \subseteq B) \subseteq A \subseteq A \cap B  $$ \\        
            Therefore, having proved both all cases exhaustively, we conclude that $$ A \subseteq B \equiv A \cap B = A $$
        \end{proof}
    \end{property}
\end{theorem}



\newpage
\begin{theorem}
Properties of Set Theoretic Equality \\

    \begin{tcolorbox}
        For all sets $A, B$, and $C$ in any universal set $U$ \\
    
        \begin{enumerate}
            \item $A = A$ \textit{Reflexive Property of Equality} %Proof Completed
            \item If $A = B$, then $B = A$ \textit{Symmetric Property of Equality} 
            \item If $(A = B) \cap (B = C)$, then $A = C$ \textit{Transitive Property of Equality} 
        \end{enumerate}
    \end{tcolorbox}



    \begin{property}
    Reflexive Property of Equality  \\
        \begin{tcolorbox}
            For any set $A$ in any universal set $U$
                \begin{equation*}
                    A = A
                \end{equation*}
        \end{tcolorbox}
        
        \begin{proof}
             Let $A$ be a subset of some universal set, $U$. By the antisymmetric property of subsets, we will prove this set equality by division of cases 
             
             
             \begin{flushleft} \textbf{Case 1 \& 2: $A \subseteq A$ and $A \subseteq A$ } \end{flushleft}
                Let $x \in U$ such that $x \in A \subseteq A$. By the definition of subsets, for every $x \in U$, if $x \in A$, then $x \in A$. But the predicate $(x \in A) \to (x \in A)$ has the form $p \to p$ for any substitution, since $p \to p$ is a tautology. Hence, this proves $A \subseteq A$. \\
                
            Hence, we have proved all cases of set equality, consequently, $A = A$
        \end{proof}
    \end{property}
\end{theorem}








\newpage
\section{Algebra of Sets}
\subsection{Union and Intersection}
\begin{definition}
Intersection of Sets \\

\begin{figure}[H]
    \centering
        \def \setA{ (0,0) circle (1cm) }
        \def \setB{ (1.5,0) circle (1cm) }
        \def \myrectangle{ (-2, -1.5) rectangle (3.5, 1.5) }
            \begin{center}
                \begin{tikzpicture}
                    \draw \myrectangle node[below left]{$U$};
                    
                    \begin{scope} % start of clip scope
                        \clip \setA ;
                        \fill[gray] \setB ;
                    \end{scope} % end of clip scope
                    
                    \draw \setA node[left] {$A$};
                    \draw \setB node[right] {$B$};
                    \end{tikzpicture}
            \end{center}
    \caption{$A \cap B$}
    \label{fig:AcapB}
\end{figure}

\begin{tcolorbox}
    Let $A$ and $B$ be subsets of some universal set $U$. \\
    The \textbf{Intersection} of $A$ and $B$, denoted $A \cap B$ and read "$A$ intescts $B$", is the set of all elements that are both in $A$ and $B$. That is,
        \begin{equation*}
            A \cap B = \{x \in U | x \in A \text{ and } x \in B \}
        \end{equation*}
\end{tcolorbox}
\end{definition}



\begin{definition}
Intersection of Collection Sets \\

\begin{tcolorbox}
    Let $\bb{C}$ be a family of sets. \\
    The \textbf{Intersection over of $\bb{C}$}, denoted $\bigcap_{X \in \bb{C}}{X}$, is defined as the set of of all elements that are in all of the sets in $\bb{C}$. That is,  
        \begin{equation*}
            \bigcap_{X \in \bb{C}}{X} = \{x \in U | x \in X, \text{ for all } X \in \bb{C} \}
        \end{equation*}
\end{tcolorbox}
\end{definition}



\begin{definition}
Intersection of Indexed Family Sets \footnotemark \\
 
    \begin{tcolorbox}
    Let $A = \{A_1, A_2, A_3, \cdots, A_n  \}$ be a family of $n$ sets, where $n \in \bb{N}$ \\ 
    
    The \textbf{Intersection over of the family of $n$ sets} , denoted $A_1 \cap A_2 \cap A_3 \cap \cdots \cap A_n$ or $\bigcap_{i=1}^{n}{A_i}$, is defined as defined as the set of of all elements that are in all of the $n$ sets. That is,  
        \begin{equation*}
            \bigcap_{i=1}^{n}{A_i} = \{x \in U | x \in A_i, \text{ for all } j \text{ with } 1 \leqq i \leqq n \}
        \end{equation*}
    \end{tcolorbox}
        \footnotetext{We can also extend this idea to define the intersection of a family of sets consisting of infinitely many sets,
            \begin{equation*}
                \bigcap_{i=1}^{\infty}{B_i} = \{ x \in U | x \in A_i, \text{ for all } j \text{ with } i \geqq 1 \}
            \end{equation*}}
    
We can generalize our definition of indexing by using an indexing set \footnotemark as follows:

        \footnotetext{
            Let $\lambda$ be a nonempty set and suppose that for each $\alpha \in \lambda$, there is a corresponding set $A_{\alpha}$. The family of sets $\{A_{\alpha} | \alpha \in \lambda \}$ is called an \textbf{indexed family of sets} indexed by $\lambda$. Each $\alpha \in \lambda$ is called an \textbf{index} and $\lambda$ is called an \textbf{indexing set}. }

    \begin{tcolorbox}
    Let $\lambda$ be a nonempty indexing set and let $\mathcal{A} = \{ A_{\alpha} | \alpha \in \lambda \}$ be an indexed family of sets. \\ 
    The \textbf{Intersection over of the indexed family $\mathcal{A}$}, denoted $\bigcap_{\alpha \in \lambda}{A_{\alpha}}$, is defined as defined as the set of of all elements that are in all of the sets $A_{\alpha}$, for each $\alpha \in \lambda$. That is,  
        \begin{equation*}
            \bigcap_{\alpha \in \lambda}{A_{\alpha}} = \{x \in U | \text{ for all } \alpha \in \lambda, x \in A_{\alpha} \}
        \end{equation*}
    \end{tcolorbox}
\end{definition}


\newpage
\begin{definition}
Union of Sets \\

\begin{figure}[H]
    \centering
        \def \setA{ (0,0) circle (1cm) }
        \def \setB{ (1.5,0) circle (1cm) }
        \def \myrectangle{ (-2, -1.5) rectangle (3.5, 1.5) }
            \begin{center}
                \begin{tikzpicture}
                    \draw \myrectangle node[below left]{$U$};
                    
                    \begin{scope} % start of clip scope
                        \fill[gray] \setA ;
                        \fill[gray] \setB ;
                    \end{scope} % end of clip scope
                    
                    \draw \setA node[left] {$A$};
                    \draw \setB node[right] {$B$};
                    \end{tikzpicture}
            \end{center}
    \caption{$A \cup B$}
    \label{fig:AcupB}
\end{figure}

\begin{tcolorbox}
    Let $A$ and $B$ be subsets of some universal set $U$. \\

    The \textbf{Union} of $A$ and $B$, denoted $A \cup B$ and read "$A$ union $B$", is the set  of all elements that are in $A$ or in $B$. That is,
     \begin{equation*}
            A \cup B = \{x \in U | x \in A \text{ or } x \in B \}
        \end{equation*}
\end{tcolorbox}

\end{definition}

\begin{definition}
Union of Collection Sets \\

\begin{tcolorbox}
    Let $\bb{C}$ be a family of sets. \\
    The \textbf{Union over of $\bb{C}$}, denoted $\bigcup_{X \in \bb{C}}{X}$, is defined as the set of of all elements that are in at least one of the sets in $\bb{C}$. That is,  
        \begin{equation*}
            \bigcup_{X \in \bb{C}}{X} = \{x \in U | x \in X, \text{ for some } X \in \bb{C} \}
        \end{equation*}
\end{tcolorbox}
\end{definition}



\begin{definition}
Union of Indexed Family Sets \footnotemark \\
    \begin{tcolorbox}
    Let $A = \{A_1, A_2, A_3, \cdots, A_n  \}$ be a family of $n$ sets, where $n \in \bb{N}$ \\ 
    
    The \textbf{Union over of the family of $n$ sets} , denoted $A_1 \cup A_2 \cup A_3 \cup \cdots \cup A_n$ or $\bigcup_{i=1}^{n}{A_i}$, is defined as defined as the set of of all elements that are in at least one of the $n$ sets. That is,  
        \begin{equation*}
            \bigcup_{i=1}^{n}{A_i} = \{x \in U | x \in A_i, \text{ for some } j \text{ with } 1 \leqq i \leqq n \}
        \end{equation*}
\end{tcolorbox}
\footnotetext{We can also extend this idea to define the union of a family of sets consisting of infinitely many sets,
    \begin{equation*}
        \bigcup_{i=1}^{\infty}{B_i} = \{ x \in U | x \in A_i, \text{ for some } j \text{ with } i \geqq 1 \}
    \end{equation*}}
We can generalize our definition of indexing by using an indexing set as follows:

    \begin{tcolorbox}
    Let $\lambda$ be a nonempty indexing set and let $\mathcal{A} = \{ A_{\alpha} | \alpha \in \lambda \}$ be an indexed family of sets. \\ 
    The \textbf{Union over of the indexed family $\mathcal{A}$}, denoted $\bigcup_{\alpha \in \lambda}{A_{\alpha}}$, is defined as defined as the set of of all elements that are in at least one of the sets $A_{\alpha}$, for each $\alpha \in \lambda$. That is,  
        \begin{equation*}
            \bigcup_{\alpha \in \lambda}{A_{\alpha}} = \{x \in U | \text{ there exists } \alpha \in \lambda, x \in A_{\alpha} \}
        \end{equation*}
    \end{tcolorbox}
\end{definition}

\newpage
\begin{theorem}
Properties of Union and Intersection \\

    \begin{tcolorbox}
        For all sets $A, B$, and $C$ in any universal set $U$ \\
    
        \begin{enumerate}
            \item $A \cup \emptyset = A$ \textit{Identity for Union I} %Proof Completed
            \item $A \cup U = U$ \textit{Identity for Union II} %Proof Completed
            \item $A \cap U = A$ \textit{Identity for Intersection I} %Proof Completed
            \item $A \cap \emptyset = \emptyset$ \textit{Identity for Intersection II}  % Proof Completed           
            \item $A \cup A = A$ \textit{Idempotent law for union} % Proof Completed
            \item $A \cap A = A$ \textit{Idempotent law for intersection} % Proof Completed
            
            \item $A \cup B = B \cup A$ \textit{Commutative law of union} % Proof Completed
            \item $A \cap B = B \cap A$ \textit{Commutative law for intersection} % Proof Completed
            
            \item $A \cup (B \cup C) = (A \cup B) \cup C$ \textit{Associative law for union} % Proof Completed
            \item $A \cap (B \cap C) = (A \cap B) \cap C$ \textit{Associative law for intersection} % Proof Completed
            
            \item $A \cup (B \cap C) = (A \cup B) \cap (A \cup C)$ \textit{Distributive law: Union over intersection} % Proof completed
            \item $A \cap (B \cup C) = (A \cap B) \cup (A \cap C)$ \textit{Distributive Law: Intersection over union} % Proof Completed
        \end{enumerate}
    \end{tcolorbox}


    \newpage
    \begin{property}
    Identity for Union I \\
        \begin{tcolorbox}
            For any set $A$ in any universal set $U$
                \begin{equation*}
                    A \cup \emptyset = A
                \end{equation*}
        \end{tcolorbox}
    
        \begin{proof}
            Let $A$ be a subset of some univeral set, $U$. By the antisymmetric property of subsets, we will prove this set equality $A \cup \emptyset = A$ by division of cases:
            
            \begin{flushleft} \textbf{Case 1: $A \cup \emptyset \subset A$} \end{flushleft}
                Let $x \in A \cup \emptyset $. This mean that $x \in A$ or $x \in \emptyset$. Since the set $\emptyset$ contains no elements, we must have $x \in A$; therefore, we conclude that $A \cup \emptyset \subset A$.
            
            \begin{flushleft} \textbf{Case 2: $A \subseteq A \cup \emptyset$} \end{flushleft} 
                Let $y \in A$. This mean that if  $y \in A$ then, $y \in A$ or $y \in \emptyset$. This implication  is trivially true because we already showed in \textit{case 1} that $y \in A \cup \emptyset$. Hence, we have proved that $A \subseteq A \cup \emptyset$. \\
            
            To sum up, we have proven both of the directions of the set equality, we have proven that $A \cup \emptyset = A$. 
        \end{proof}
    \end{property}
    
    
    
    \begin{property}
    Property 2: Identity for Union II \\
        \begin{tcolorbox}
            For any set $A$ in any universal set $U$
                \begin{equation*}
                    A \cup U = U
                \end{equation*}
        \end{tcolorbox}
    
        \begin{proof}
            Let $A$ be a subset of some univeral set, $U$. By the antisymmetric property of subsets, we will prove this set equality $A \cup U = U$ by division of cases:
            
            \begin{flushleft} \textbf{Case 1: $A \cup U \subset U$} \end{flushleft}
                Let $x \in A \cup U $. This mean that if $x \in A$ or $x \in U$, then $x \in U$. Since $A \subseteq U$, it follows that $x \in A$ and $x \in U$; therefore, we conclude that $A \cup U \subset U$.
            
            \begin{flushleft} \textbf{Case 2: $U \subseteq A \cup U$} \end{flushleft} 
                Let $y \in U$. This mean that if  $y \in U$ then, $y \in A$ or $y \in U$. This implication  is trivially true because we already showed in \textit{case 1} that $y \in A \cup U$. Hence, we have proved that $U \subseteq A \cup U$. \\
            
            To sum up, we have proven both of the directions of the set equality, we have proven that $A \cup U = U$. 
        \end{proof}
    \end{property}
    
    
    
    \newpage
    \begin{property}
    Identity for Intersection I \\
        \begin{tcolorbox}
            For any set $A$ in any universal set $U$
                \begin{equation*}
                    A \cap U = A
                \end{equation*}
        \end{tcolorbox}
    
        \begin{proof}
            Let $A$ be a subset of some univeral set, $U$. By the antisymmetric property of subsets, we will prove this set equality $A \cap U = A$ by division of cases:
            
            \begin{flushleft} \textbf{Case 1: $A \cap U \subset A$} \end{flushleft}
                Let $x \in A \cap U $. This mean that if $x \in A$ and $x \in U$, then $x \in A$. Since $A \subseteq U$, it follows that $x \in A$ and $x \in U$; therefore, we conclude that $A \cap U \subset U$.
            
            \begin{flushleft} \textbf{Case 2: $A \subseteq A \cap U$} \end{flushleft} 
                Let $y \in A$. This mean that if  $y \in A$ then, $y \in A$ and $y \in U$. This implication  is trivially true because we already showed in \textit{case 1} that $y \in A \cap U$. Hence, we have proved that $A \subseteq A \cap U$. \\
            
            To sum up, we have proven both of the directions of the set equality, we have proven that $A \cap U = A$. 
        \end{proof}
    \end{property}
    
    
    \begin{property}
    Identity for Intersection II \\
        \begin{tcolorbox}
            For any set $A$ in any universal set $U$
                \begin{equation*}
                    A \cap \emptyset = \emptyset
                \end{equation*}
        \end{tcolorbox}
    
        \begin{proof}
            Let $A$ be a subset of some univeral set, $U$. By the antisymmetric property of subsets, we will prove this set equality $A \cap \emptyset = \emptyset$ by division of cases:
            
            \begin{flushleft} \textbf{Case 1: $A \cap \emptyset \subset \emptyset$} \end{flushleft}
                Let $x \in A \cap \emptyset $. This mean that $x \in A$ and $x \in \emptyset$. Since the set $\emptyset$ contains no elements, we must have $x \in \emptyset$; therefore, we conclude that $A \cap \emptyset \subset \emptyset$.
            
            \begin{flushleft} \textbf{Case 2: $\emptyset \subseteq A \cap \emptyset$} \end{flushleft} 
                Let $y \in \emptyset$. This mean that if  $y \in \emptyset$ then, $y \in A$ and $y \in \emptyset$. This implication  is trivially true because we already showed in \textit{case 1} that $y \in A \cap \emptyset$. Hence, we have proved that $A \subseteq A \cap \emptyset$. \\
            
            To sum up, we have proven both of the directions of the set equality, we have proven that $A \cap \emptyset = \emptyset$. 
        \end{proof}
    \end{property}
    
    
    
    \newpage
    \begin{property}
    dempotent law for Union \\
        \begin{tcolorbox}
            For any set $A$ in any universal set $U$
                \begin{equation*}
                    A \cup A = A
                \end{equation*}
        \end{tcolorbox}
    
        \begin{proof}
            Let $A$ be a subset of some univeral set, $U$. By the antisymmetric property of subsets, we will prove this set equality $A \cup A = A$ by division of cases:
            
            \begin{flushleft} \textbf{Case 1: $A \cup A \subset A$} \end{flushleft}
                Let $x \in A \cup A$. This mean that either $x \in A$ or $x \in A$, we must have $x \in A$; therefore, we conclude that $A \cup A \subset A$.
            
            \begin{flushleft} \textbf{Case 2: $A \subseteq A \cup A$} \end{flushleft} 
                Let $y \in A$. This mean that if  $y \in A$ then, $y \in A$ or $y \in A$. This implication  is trivially true because we already showed in \textit{case 1} that $y \in A \cup A$. Hence, we have proved that $A \subseteq A \cup A$. \\
            
            To sum up, we have proven both of the directions of the set equality, we have proven that $A \cup A = A$. 
        \end{proof}
    \end{property}
    
    
    
    \begin{property}
    Idempotent law for Intersection \\
        \begin{tcolorbox}
            For any set $A$ in any universal set $U$
                \begin{equation*}
                    A \cap A = A
                \end{equation*}
        \end{tcolorbox}
    
        \begin{proof}
            Let $A$ be a subset of some univeral set, $U$. By the antisymmetric property of subsets, we will prove this set equality $A \cap A = A$ by division of cases:
            
            \begin{flushleft} \textbf{Case 1: $A \cap A \subset A$} \end{flushleft}
                Let $x \in A \cap A $. This mean that if $x \in A$ and $x \in A$, then $x \in A$. Since $A \subseteq U$, it follows that $x \in A$ and $x \in A$; therefore, we conclude that $A \cap A \subset A$.
            
            \begin{flushleft} \textbf{Case 2: $A \subseteq A \cap A$} \end{flushleft} 
                Let $y \in A$. This mean that if  $y \in A$ then, $y \in A$ and $y \in A$. This implication  is trivially true because we already showed in \textit{case 1} that $y \in A \cap A$. Hence, we have proved that $A \subseteq A \cap A$. \\
            
            To sum up, we have proven both of the directions of the set equality, we have proven that $A \cap A = A$. 
        \end{proof}
    \end{property}
    
    
    
    \newpage
    \begin{property}
    Commutative law of Union \\
        \begin{tcolorbox}
            For any set $A$ in any universal set $U$
                \begin{equation*}
                    A \cup B = B \cup A
                \end{equation*}
        \end{tcolorbox}
    
        \begin{proof}
            Let $A$ be a subset of some univeral set, $U$. By the antisymmetric property of subsets, we will prove this set equality $A \cup B = B \cup A$ by division of cases:
            
            \begin{flushleft} \textbf{Case 1: $A \cup B \subset B \cup A$} \end{flushleft}
                Let $x \in A \cup B$. This mean if $x \in A \cup B$, then $x \in A$ or $x \in B$. By the definition of union, we know that if $x \in A \cup B$, then $x \in B$ or $x \in A$; therefore, we conclude that $A \cup B \subset B \cup A$.
            
            \begin{flushleft} \textbf{Case 2: $B \cup A \subseteq A \cup B$} \end{flushleft} 
                Let $y \in B \cup A$. This mean if $x \in B \cup A$, then $x \in B$ or $x \in A$. By the definition of union, we know that if $x \in B \cup A$, then $x \in A$ or $x \in B$; therefore, we conclude that $B \cup A \subset A \cup B$.
            
            To sum up, we have proven both of the directions of the set equality, we have proven that $A \cup B = B \cup A$. 
        \end{proof}
    \end{property}
    
    
    \begin{property}
    Commutative law of Intersection \\
        \begin{tcolorbox}
            For any set $A$ in any universal set $U$
                \begin{equation*}
                    A \cap B = B \cap A
                \end{equation*}
        \end{tcolorbox}
    
        \begin{proof}
            Let $A$ be a subset of some univeral set, $U$. By the antisymmetric property of subsets, we will prove this set equality $A \cap B = B \cap A$ by division of cases:
            
            \begin{flushleft} \textbf{Case 1: $A \cap B \subset B \cap A$} \end{flushleft}
                Let $x \in A \cap B$. This mean if $x \in A \cap B$, then $x \in A$ or $x \in B$. By the definition of intersection, we know that if $x \in A \cap B$, then $x \in B$ or $x \in A$; therefore, we conclude that $A \cap B \subset B \cap A$.
            
            \begin{flushleft} \textbf{Case 2: $B \cap A \subseteq A \cap B$} \end{flushleft} 
                Let $y \in B \cap A$. This mean if $x \in B \cap A$, then $x \in B$ or $x \in A$. By the definition of union, we know that if $x \in B \cap A$, then $x \in A$ or $x \in B$; therefore, we conclude that $B \cap A \subset A \cap B$.
            
            To sum up, we have proven both of the directions of the set equality, we have proven that $A \cap B = B \cap A$. 
        \end{proof}
    \end{property}
     
    
    \newpage
    \begin{property}
    Associative law of Union \\
        \begin{tcolorbox}
            For any set $A$ in any universal set $U$
                \begin{equation*}
                    A \cup (B \cup C) = (A \cup B) \cup C
                \end{equation*}
        \end{tcolorbox}
    
        \begin{proof}
            Let $A$ be a subset of some univeral set, $U$. By the antisymmetric property of subsets, we will prove this set equality $A \cup B = B \cup A$ by division of cases:
            
            \begin{flushleft} \textbf{Case 1: $A \cup (B \cup C) \subset (A \cup B) \cup C$} \end{flushleft}
                Let $x \in A \cup (B \cup C)$. This mean if $x \in A \cup (B \cup C)$, then $x \in A$ or $x \in (B \cup C)$, where $x \in (B \cup C)$ implies that $x \in B$ or $x \in C$. So, we know that $x \in A$ or $x \in B$ or $x \in C$, it follows that $x \in (A \cup B)$ or $x \in C$. Given that $x \in A \cup (B \cup C)$ implies $x \in (A \cup B) \cup C$, we conclude that $A \cup (B \cup C) \subset (A \cup B) \cup C$.      
            
            \begin{flushleft} \textbf{Case 2: $(A \cup B) \cup C \subset A \cup (B \cup C)$} \end{flushleft} 
                Let $x \in (A \cup B) \cup C$. This mean if $x \in (A \cup B) \cup C$, then $x \in (A \cup B)$ or $x \in C$, where $x \in (A \cup B)$ implies that $x \in A$ or $x \in B$. So, we know that $x \in A$ or $x \in B$ or $x \in C$, it follows that $x \in A$ or $x \in (B \cup C)$. Given that $x \in (A \cup B) \cup C$ implies $x \in A \cup (B \cup C)$, we conclude that $(A \cup B) \cup C \subset A \cup (B \cup C)$. 
            
            To sum up, we have proven both of the directions of the set equality, we have proven that $A \cup (B \cup C) = (A \cup B) \cup C$. 
        \end{proof}
    \end{property}
    
    
    \newpage
    \begin{property}
    Associative law of Intersection \\
        \begin{tcolorbox}
            For any set $A$ in any universal set $U$
                \begin{equation*}
                    A \cap (B \cap C) = (A \cap B) \cap C
                \end{equation*}
        \end{tcolorbox}
    
        \begin{proof}
            Let $A$ be a subset of some univeral set, $U$. By the antisymmetric property of subsets, we will prove this set equality $A \cap B = B \cap A$ by division of cases:
            
            \begin{flushleft} \textbf{Case 1: $A \cap (B \cap C) \subset (A \cap B) \cap C$} \end{flushleft}
                Let $x \in A \cap (B \cap C)$. This mean if $x \in A \cap (B \cap C)$, then $x \in A$ or $x \in (B \cap C)$, where $x \in (B \cap C)$ implies that $x \in B$ or $x \in C$. So, we know that $x \in A$ or $x \in B$ or $x \in C$, it follows that $x \in (A \cap B)$ or $x \in C$. Given that $x \in A \cap (B \cap C)$ implies $x \in (A \cap B) \cap C$, we conclude that $A \cap (B \cap C) \subset (A \cap B) \cap C$.      
            
            \begin{flushleft} \textbf{Case 2: $(A \cap B) \cap C \subset A \cap (B \cap C)$} \end{flushleft} 
                Let $x \in (A \cap B) \cap C$. This mean if $x \in (A \cap B) \cap C$, then $x \in (A \cap B)$ or $x \in C$, where $x \in (A \cap B)$ implies that $x \in A$ or $x \in B$. So, we know that $x \in A$ or $x \in B$ or $x \in C$, it follows that $x \in A$ or $x \in (B \cap C)$. Given that $x \in (A \cap B) \cap C$ implies $x \in A \cap (B \cap C)$, we conclude that $(A \cap B) \cap C \subset A \cap (B \cap C)$. 
            
            To sum up, we have proven both of the directions of the set equality, we have proven that $A \cap (B \cap C) = (A \cap B) \cap C$. 
        \end{proof}
    \end{property}
    
    
    
    
    \newpage
    \begin{property}
    Distributive law: Union over Intersection \\
        \begin{tcolorbox}
            For any set $A$ in any universal set $U$
                \begin{equation*}
                    A \cup (B \cap C) = (A \cup B) \cap (A \cup C)
                \end{equation*}
        \end{tcolorbox}
    
        \begin{proof}
            Let $A$ be a subset of some univeral set, $U$. By the antisymmetric property of subsets, we will prove this set equality $A \cup (B \cap C) = (A \cup B) \cap (A \cup C)$ by division of cases: \\
            
            \begin{flushleft} \textbf{Case 1: $A \cup (B \cap C) \subset (A \cup B) \cap (A \cup C)$} \end{flushleft}
                Let $x \in A \cup (B \cap C)$. This mean that if $x \in A \cup (B \cap C)$, then $x \in A$ or $x \in B \cap C$. We will show that this conditional system is trivially true by division of sub-cases. \\
                
                    \begin{flushleft} \textbf{Case 1 - Sub-Case 1: $x \in A$} \end{flushleft}
                        Let $x \in A$. This means that if $x \in A$ then $x \in A \cup B$ and $x \in A \cup C$, so it follows that $x \in (A \cup B) \cap (A \cup C)$ \\
                    
                    \begin{flushleft} \textbf{Case 1 - Sub-Case 2: $x \in (B \cap C)$} \end{flushleft}  
                        Let $x \in (B \cap C)$. This mean that if $x \in (B \cap C)$ then, $x \in B$ and $x \in C$.  But $x \in B$ implies $x \in (A \cup B)$. Similarly, $x \in C$ implies $x \in (A \cup C)$. Since $x$ is an element of both sets, we can conclude that $x \in (A \cup B) \cap (A \cup C)$. \\
                
                We see in both sub cases that $x \in A \cup (B \cap C)$ implies $x \in (A \cup B) \cap (A \cup C)$, consequently, we proved that $A \cup (B \cap C) \subset (A \cup B) \cap (A \cup C)$. \\
                
            \begin{flushleft} \textbf{Case 2: $(A \cup B) \cap (A \cup C) \subset A \cup (B \cap C)$} \end{flushleft} 
                Let $y \in (A \cup B) \cap (A \cup C)$. This mean that if $y \in (A \cup B) \cap (A \cup C)$, then $y \in (A \cup B)$ and $y \in (A \cup C)$. We will show that this conditional statement is trivially true by division of sub-cases. \\
                
                    \begin{flushleft} \textbf{Case 2 - Sub-Case 1: $y \in A$} \end{flushleft}
                        Let $y \in A$. This means that if $y \in A$ then $y \in A \cup B$ and $y \in A \cup C$, so it follows that $y \in (A \cup C) \cap (A \cup C)$ \\
                    
                    \begin{flushleft} \textbf{Case 2 - Sub-Case 2: $y \notin A$} \end{flushleft}  
                        Let $y \notin A$. Recall, we also assume that $y \in (A \cup B) \cap (A \cup C)$. This means if $y \notin A$ and $y \in (A \cup B)$, then $y \in B$. Similarly, if $y \notin A$ and $y \in (A \cup C)$, then $y \in C$. Thus, it must be $y \in B \cap C$. Hence, if $y \notin A$, then $y \in A \cup (B \cap C)$. \\
                
                We see in both sub cases that $y \in (A \cup B) \cap (A \cup C)$ implies $y \in A \cup (B \cap C)$, consequently, we proved that $(A \cup B) \cap (A \cup C) \subset A \cup (B \cap C)$. \\   
            To sum up, we have proven both of the directions of the set equality, we have proven that $A \cup (B \cap C) = (A \cup B) \cap (A \cup C)$. 
        \end{proof}
    \end{property}
    
    
    \newpage
    \begin{property}
    Distributive law: Union over Intersection \\
        \begin{tcolorbox}
            For any sets $A$ and $B$ in any universal set $U$
                \begin{equation*}
                    A \cup (B \cap C) = (A \cup B) \cap (A \cup C)
                \end{equation*}
        \end{tcolorbox}
    
        \begin{proof}
            Let $x \in U$, using set algebra, the proposition becomes: 
                \begin{align*}
                    x \in A \cup (B \cap C) & \equiv (x \in A) \cup (x \in B \cap C) \\
                    & \equiv (x \in A) \cup [(x \in B) \cap (x \in C)] \\
                    & \equiv [(x \in A) \cup (x \in B)] \cap [(x \in A) \cup (x \in C)] \\
                    & \equiv [x \in A \cup B] \cap [x \in A \cup C] \\
                    & \equiv x \in [(A \cup B) \cap (A \cup C)] \\
                \end{align*}
            Using the tautology, $p \wedge (q \vee r) = (p \wedge q) \vee (p \wedge r)$, we see that $x \in A \cup (B \cap C)$ is sufficient for $x \in [(A \cup B) \cap (A \cup C)]$. This completes the proof that $$ A \cup (B \cap C) = (A \cup B) \cap (A \cup C) $$
        \end{proof}
    \end{property}
    
    
    \newpage
    \begin{property}
    Distributive law: Intersection over Union \\
        \begin{tcolorbox}
            For any set $A$ in any universal set $U$
                \begin{equation*}
                    A \cap (B \cup C) = (A \cap B) \cup (A \cap C)
                \end{equation*}
        \end{tcolorbox}
    
        \begin{proof}
            Let $A$ be a subset of some univeral set, $U$. By the antisymmetric property of subsets, we will prove this set equality $A \cap (B \cup C) = (A \cap B) \cup (A \cap C)$ by division of cases: \\
            
            \begin{flushleft} \textbf{Case 1: $A \cap (B \cup C) \subset (A \cap B) \cup (A \cap C)$} \end{flushleft}
                Let $x \in A \cap (B \cup C)$. This mean that if $x \in A \cap (B \cup C)$, then $x \in A$ and $x \in B \cup C$. We will show that this conditional system is trivially true by division of sub-cases. \\
                
                    \begin{flushleft} \textbf{Case 1 - Sub-Case 1: $x \in A$} \end{flushleft}
                        Let $x \in A$. This means that if $x \in A$ then $x \in A \cap B$ or $x \in A \cap C$, so it follows that $x \in (A \cap B) \cup (A \cap C)$ \\
                    
                    \begin{flushleft} \textbf{Case 1 - Sub-Case 2: $x \in (B \cup C)$} \end{flushleft}  
                        Let $x \in (B \cup C)$. This mean that if $x \in (B \cup C)$ then, $x \in B$ or $x \in C$.  But $x \in B$ implies $x \in (A \cap B)$. Similarly, $x \in C$ implies $x \in (A \cap C)$. Since $x$ is an element of either sets, we can conclude that $x \in (A \cap B) \cup (A \cap C)$. \\
                
                We see in both sub cases that $x \in A \cap (B \cup C)$ implies $x \in (A \cap B) \cup (A \cap C)$, consequently, we proved that $A \cap (B \cup C) \subset (A \cap B) \cup (A \cap C)$. \\
                
            \begin{flushleft} \textbf{Case 2: $(A \cap B) \cup (A \cap C) \subset A \cap (B \cup C)$} \end{flushleft} 
                Let $y \in (A \cap B) \cup (A \cap C)$. This mean that if $y \in (A \cap B) \cup (A \cap C)$, then $y \in (A \cap B)$ or $y \in (A \cap C)$. We will show that this conditional statement is trivially true by division of sub-cases. \\
                
                    \begin{flushleft} \textbf{Case 2 - Sub-Case 1: $y \in A$} \end{flushleft}
                        Let $y \in A$. This means that if $y \in A$ then $y \in A \cap B$ or $y \in A \cap C$, so it follows that $y \in (A \cap C) \cup (A \cap C)$ \\
                    
                    \begin{flushleft} \textbf{Case 2 - Sub-Case 2: $y \notin A$} \end{flushleft}  
                        Let $y \notin A$. Recall, we also assume that $y \in (A \cap B) \cup (A \cap C)$. This means if $y \notin A$ and $y \in (A \cap B)$, then $y \in B$. Similarly, if $y \notin A$ and $y \in (A \cap C)$, then $y \in C$. Thus, it must be $y \in B \cup C$. Hence, if $y \notin A$, then $y \in A \cap (B \cup C)$. \\
                
                We see in both sub cases that $y \in (A \cap B) \cup (A \cap C)$ implies $y \in A \cap (B \cup C)$, consequently, we proved that $(A \cap B) \cup (A \cap C) \subset A \cap (B \cup C)$. \\   
            To sum up, we have proven both of the directions of the set equality, we have proven that $A \cap (B \cup C) = (A \cap B) \cup (A \cap C)$. 
        \end{proof}
    \end{property}
    
    
    \newpage
    \begin{property}
    Distributive law: Intersection over Union \\
        \begin{tcolorbox}
            For any sets $A$ and $B$ in any universal set $U$
                \begin{equation*}
                    A \cap (B \cup C) = (A \cap B) \cup (A \cap C)
                \end{equation*}
        \end{tcolorbox}
    
        \begin{proof}
            Let $x \in U$, using set algebra, the proposition becomes: 
                \begin{align*}
                    x \in A \cap (B \cup C) & \equiv (x \in A) \cap (x \in B \cup C) \\
                    & \equiv (x \in A) \cap [(x \in B) \cup (x \in C)] \\
                    & \equiv [(x \in A) \cap (x \in B)] \cup [(x \in A) \cap (x \in C)] \\
                    & \equiv [x \in A \cap B] \cup [x \in A \cap C] \\
                    & \equiv x \in [(A \cap B) \cup (A \cap C)] \\
                \end{align*}
            Using the tautology, $p \vee (q \wedge r) = (p \vee q) \wedge (p \vee r)$, we see that $x \in A \cap (B \cup C)$ is sufficient for $x \in [(A \cap B) \cup (A \cap C)]$. This completes the proof that $$ A \cap (B \cup C) = (A \cap B) \cup (A \cap C) $$
        \end{proof}
    \end{property}

\end{theorem}








\newpage
\begin{theorem}
Properties for Collection of Union and Intersection \\

For all sets $A$ and $B$ in any universal set $U$. Let $\lambda$ be a nonempty indexing set and $\cc{A} = \{ A_{\alpha} | \alpha \in \lambda \}$ be an indexed family of sets, then 
    
    \begin{enumerate}
        \item For each $\beta \in \lambda$, $\bigcap_{\alpha \in \lambda}{A_{\alpha}} \subseteq A_{\beta}$ Intersection is a Subset %Proof Completed
        
        \item For each $\beta \in \lambda$, $A_{\beta} \subseteq \bigcup_{\alpha \in \lambda}{A_{\alpha}}$ Union is a Superset %Proof Completed
        
        \item $B \cap \left (\bigcap_{\alpha \in \lambda}{A_{\alpha}} \right ) = \bigcap_{\alpha \in \lambda}{(B \cap A_{\alpha})}$ Distribution Law: Intersection over Intersection %Proof Completed
        
        \item $B \cap \left (\bigcup_{\alpha \in \lambda}{A_{\alpha}} \right ) = \bigcup_{\alpha \in \lambda}{(B \cap A_{\alpha})}$ Distribution Law: Intersection over Union %Proof Completed
        
        \item $B \cup \left ( \bigcup_{\alpha \in \lambda}{A_{\alpha}} \right ) = \bigcup_{\alpha \in \lambda}{(B \cup A_{\alpha})}$ Distribution Law: Union over Union %Proof Completed
        
        \item $B \cup \left ( \bigcap_{\alpha \in \lambda}{A_{\alpha}} \right ) = \bigcap_{\alpha \in \lambda}{(B \cup A_{\alpha})}$ Distribution Law: Union over Intersection %Proof Completed
    \end{enumerate}



    \newpage
    \begin{property}
    Intersection is a Subset \\
        \begin{tcolorbox}
            For all sets $A$ and $B$ in any universal set $U$. Let $\lambda$ be a nonempty indexing set and $\cc{A} = \{ A_{\alpha} | \alpha \in \lambda \}$ be an indexed family of sets. For each $\beta \in \lambda$, 
                \begin{equation*}
                    \bigcap_{\alpha \in \lambda}{A_{\alpha}} \subseteq A_{\beta}
                \end{equation*}
        \end{tcolorbox}
    
        \begin{proof}
            Let $x \in U$ such that $x \in \bigcap_{\alpha \in \lambda}{A_{\alpha}}$. Suppose $\beta \in \lambda$. This means that $x \in A_\beta$, for each $\beta \in \lambda$. By Law of Simplification of Logical Multiplication, we see that if $x \in \bigcap_{\alpha \in \lambda}{A_{\alpha}}$ then $x \in A_{\beta}$. Therefore, for each $\beta \in \lambda$, $$ \bigcap_{\alpha \in \lambda}{A_{\alpha}} \subseteq A_{\beta} $$ 
        \end{proof}
    \end{property}
    
    
    \begin{property}
    Union is a Superset \\
        \begin{tcolorbox}
            For all sets $A$ and $B$ in any universal set $U$. Let $\lambda$ be a nonempty indexing set and $\cc{A} = \{ A_{\alpha} | \alpha \in \lambda \}$ be an indexed family of sets. For each $\beta \in \lambda$, 
                \begin{equation*}
                    A_{\beta} \subseteq \bigcup_{\alpha \in \lambda}{A_{\alpha}}
                \end{equation*}
        \end{tcolorbox}
    
        \begin{proof}
            Let $x \in U$ such that $x \in A_{\beta}$. Suppose $x \in \bigcup_{\alpha \in \lambda}{A_{\alpha}}$. This means that $x \in A_\alpha$, for each $\alpha \in \lambda$. By Law of Simplification of Logical Addition, we see thatt if $x \in A_\beta$ then $x \in \bigcup_{\alpha \in \lambda}{A_{\alpha}}$. Therefore, for each $\beta \in \lambda$, $$ A_{\beta} \subseteq \bigcup_{\alpha \in \lambda}{A_{\alpha}} $$ 
        \end{proof}
    \end{property}
    
    
    
    
    \newpage
    \begin{property}
    Distribution Law: Intersection over Intersection \\
        \begin{tcolorbox}
            For all sets $A$ and $B$ in any universal set $U$. Let $\lambda$ be a nonempty indexing set and $\cc{A} = \{ A_{\alpha} | \alpha \in \lambda \}$ be an indexed family of sets, then
                \begin{equation*}
                    B \cap \left ( \bigcap_{\alpha \in \lambda}{A_{\alpha}} \right ) = \bigcap_{\alpha \in \lambda}{(B \cap A_{\alpha})}
                \end{equation*}
        \end{tcolorbox}
    
        \begin{proof}
            Let $B$ be a subset of some universal set, $U$. By the antisymmetric property of subsets, we will prove this set equality by division of cases: \\
            
            \begin{flushleft} \textbf{Case 1: $B \cap \left ( \bigcap_{\alpha \in \lambda}{A_{\alpha}} \right ) \subseteq \bigcap_{\alpha \in \lambda}{(B \cap A_{\alpha})}$} \end{flushleft}
                Let $x \in U$ such that $x \in B \cap \left ( \bigcap_{\alpha \in \lambda}{A_{\alpha}} \right )$. This means that $x \in B$ and $x \in \left ( \bigcap_{\alpha \in \lambda}{A_{\alpha}} \right )$. Since $x \in \left ( \bigcap_{\alpha \in \lambda}{A_{\alpha}} \right )$, there exists $\beta \in A_{\beta}$, where $\beta \in \lambda$. Since $x \in \left ( \bigcap_{\alpha \in \lambda}{A_{\alpha}} \right )$, it must be that, $x \in B \cap A_{\beta}$. It follows that $x \in B \cap \left ( \bigcap_{\alpha \in \lambda}{A_{\alpha}} \right )$ implies $x \in \bigcap_{\alpha \in \lambda}{(B \cap A_{\alpha})}$. This proves the case that $$B \cap \left ( \bigcap_{\alpha \in \lambda}{A_{\alpha}} \right ) \subseteq \bigcap_{\alpha \in \lambda}{(B \cap A_{\alpha})}$$. \\
                
            \begin{flushleft} \textbf{Case 2: $\bigcap_{\alpha \in \lambda}{(B \cap A_{\alpha})} \subseteq B \cap \left ( \bigcap_{\alpha \in \lambda}{A_{\alpha}} \right ) )$} \end{flushleft} 
                Let $y \in U$ and $y \in \bigcap_{\alpha \in \lambda}{(B \cap A_{\alpha})}$. There exists $\omega \in A_{\omega}$, where $\omega \in \lambda$ such that $y \in B \cap A_{\omega}$. This means that $y \in B$ and $y \in A_{\omega}$. Since $y \in A_{\omega}$, it must be that $y \in \bigcap_{\alpha \in \lambda}{A_{\alpha}}$. It follows that $\bigcap_{\alpha \in \lambda}{(B \cap A_{\alpha})}$ implies $y \in B \cap \left ( \bigcap_{\alpha \in \lambda}{A_{\alpha}} \right )$. This proves the case that $$\bigcap_{\alpha \in \lambda}{(B \cap A_{\alpha})} \subseteq B \cap \left ( \bigcap_{\alpha \in \lambda}{A_{\alpha}} \right ) )$$. \\
                
            To sum up, we have proven both of the directions of the set equality, we have proven that $$B \cup \left ( \bigcup_{\alpha \in \lambda}{A_{\alpha}} \right ) = \bigcup_{\alpha \in \lambda}{(B \cup A_{\alpha})}$$. 
        \end{proof}
    \end{property}
    
    
    
    \newpage
    \begin{property}
    Distribution Law: Intersection over Union \\
        \begin{tcolorbox}
            For all sets $A$ and $B$ in any universal set $U$. Let $\lambda$ be a nonempty indexing set and $\cc{A} = \{ A_{\alpha} | \alpha \in \lambda \}$ be an indexed family of sets, then
                \begin{equation*}
                    B \cap \left ( \bigcup_{\alpha \in \lambda}{A_{\alpha}} \right ) = \bigcup_{\alpha \in \lambda}{(B \cap A_{\alpha})}
                \end{equation*}
        \end{tcolorbox}
    
        \begin{proof}
            Let $B$ be a subset of some universal set, $U$. By the antisymmetric property of subsets, we will prove this set equality by division of cases: \\
            
            \begin{flushleft} \textbf{Case 1: $B \cap \left ( \bigcup_{\alpha \in \lambda}{A_{\alpha}} \right ) \subseteq \bigcup_{\alpha \in \lambda}{(B \cap A_{\alpha})}$} \end{flushleft}
                Let $x \in U$ such that $x \in B \cap \left ( \bigcup_{\alpha \in \lambda}{A_{\alpha}} \right )$. This means that $x \in B$ and $x \in \left ( \bigcup_{\alpha \in \lambda}{A_{\alpha}} \right )$. Since $x \in \left ( \bigcup_{\alpha \in \lambda}{A_{\alpha}} \right )$, there exists $\beta \in A_{\beta}$, where $\beta \in \lambda$. Since $x \in \left ( \bigcup_{\alpha \in \lambda}{A_{\alpha}} \right )$, it must be that, $x \in B \cap A_{\beta}$. It follows that $x \in B \cap \left ( \bigcup_{\alpha \in \lambda}{A_{\alpha}} \right )$ implies $x \in \bigcup_{\alpha \in \lambda}{(B \cap A_{\alpha})}$. This proves the case that $$B \cap \left ( \bigcup_{\alpha \in \lambda}{A_{\alpha}} \right ) \subseteq \bigcup_{\alpha \in \lambda}{(B \cap A_{\alpha})}$$. \\
                
            \begin{flushleft} \textbf{Case 2: $\bigcup_{\alpha \in \lambda}{(B \cap A_{\alpha})} \subseteq B \cap \left ( \bigcup_{\alpha \in \lambda}{A_{\alpha}} \right )$} \end{flushleft} 
                Let $y \in U$ and $y \in \bigcup_{\alpha \in \lambda}{(B \cap A_{\alpha})}$. There exists $\omega \in A_{\omega}$, where $\omega \in \lambda$ such that $y \in B \cap A_{\omega}$. This means that $y \in B$ and $y \in A_{\omega}$. Since $y \in A_{\omega}$, it must be that $y \in \bigcup_{\alpha \in \lambda}{A_{\alpha}}$. It follows that $\bigcup_{\alpha \in \lambda}{(B \cap A_{\alpha})}$ implies $y \in B \cap \left ( \bigcup_{\alpha \in \lambda}{A_{\alpha}} \right )$. This proves the case that $$\bigcup_{\alpha \in \lambda}{(B \cap A_{\alpha})} \subseteq B \cap \left ( \bigcup_{\alpha \in \lambda}{A_{\alpha}} \right )$$. \\
                
            To sum up, we have proven both of the directions of the set equality, we have proven that $$B \cap \left ( \bigcup_{\alpha \in \lambda}{A_{\alpha}} \right ) = \bigcup_{\alpha \in \lambda}{(B \cap A_{\alpha})}$$. 
        \end{proof}
    \end{property}
    
    \newpage
    \begin{property}
    Distribution Law: Union over Union \\
        \begin{tcolorbox}
            For all sets $A$ and $B$ in any universal set $U$. Let $\lambda$ be a nonempty indexing set and $\cc{A} = \{ A_{\alpha} | \alpha \in \lambda \}$ be an indexed family of sets, then
                \begin{equation*}
                    B \cup \left ( \bigcup_{\alpha \in \lambda}{A_{\alpha}} \right ) = \bigcup_{\alpha \in \lambda}{(B \cup A_{\alpha})}
                \end{equation*}
        \end{tcolorbox}
    
        \begin{proof}
            Let $B$ be a subset of some universal set, $U$. By the antisymmetric property of subsets, we will prove this set equality by division of cases: \\
            
            \begin{flushleft} \textbf{Case 1: $B \cup \left ( \bigcup_{\alpha \in \lambda}{A_{\alpha}} \right ) \subseteq \bigcup_{\alpha \in \lambda}{(B \cup A_{\alpha})}$} \end{flushleft}
                Let $x \in U$ such that $x \in B \cup \left ( \bigcup_{\alpha \in \lambda}{A_{\alpha}} \right )$. This means that $x \in B$ or $x \in \left ( \bigcup_{\alpha \in \lambda}{A_{\alpha}} \right )$. Since $x \in \left ( \bigcup_{\alpha \in \lambda}{A_{\alpha}} \right )$, there exists $\beta \in A_{\beta}$, where $\beta \in \lambda$. Since $x \in \left ( \bigcup_{\alpha \in \lambda}{A_{\alpha}} \right )$, it must be that, $x \in B \cup A_{\beta}$. It follows that $x \in B \cup \left ( \bigcup_{\alpha \in \lambda}{A_{\alpha}} \right )$ implies $x \in \bigcup_{\alpha \in \lambda}{(B \cup A_{\alpha})}$. This proves the case that $$B \cup \left ( \bigcup_{\alpha \in \lambda}{A_{\alpha}} \right ) \subseteq \bigcup_{\alpha \in \lambda}{(B \cup A_{\alpha})}$$. \\
                
            \begin{flushleft} \textbf{Case 2: $\bigcup_{\alpha \in \lambda}{(B \cup A_{\alpha})} \subseteq B \cup \left ( \bigcup_{\alpha \in \lambda}{A_{\alpha}} \right )$} \end{flushleft} 
                Let $y \in U$ and $y \in \bigcup_{\alpha \in \lambda}{(B \cup A_{\alpha})}$. There exists $\omega \in A_{\omega}$, where $\omega \in \lambda$ such that $y \in B \cup A_{\omega}$. This means that $y \in B$ or $y \in A_{\omega}$. Since $y \in A_{\omega}$, it must be that $y \in \bigcup_{\alpha \in \lambda}{A_{\alpha}}$. It follows that $\bigcup_{\alpha \in \lambda}{(B \cup A_{\alpha})}$ implies $y \in B \cup \left ( \bigcup_{\alpha \in \lambda}{A_{\alpha}} \right )$. This proves the case that $$\bigcup_{\alpha \in \lambda}{(B \cup A_{\alpha})} \subseteq B \cup \left ( \bigcup_{\alpha \in \lambda}{A_{\alpha}} \right )$$. \\
                
            To sum up, we have proven both of the directions of the set equality, we have proven that $$B \cup \left ( \bigcup_{\alpha \in \lambda}{A_{\alpha}} \right ) = \bigcup_{\alpha \in \lambda}{(B \cup A_{\alpha})}$$. 
        \end{proof}
    \end{property}
    
    \newpage
    \begin{property}
    Distribution Law: Union over Intersection \\
        \begin{tcolorbox}
            For all sets $A$ and $B$ in any universal set $U$. Let $\lambda$ be a nonempty indexing set and $\cc{A} = \{ A_{\alpha} | \alpha \in \lambda \}$ be an indexed family of sets, then
                \begin{equation*}
                    B \cup \left ( \bigcap_{\alpha \in \lambda}{A_{\alpha}} \right ) = \bigcap_{\alpha \in \lambda}{(B \cup A_{\alpha})}
                \end{equation*}
        \end{tcolorbox}
    
        \begin{proof}
            Let $B$ be a subset of some universal set, $U$. By the antisymmetric property of subsets, we will prove this set equality by division of cases: \\
            
            \begin{flushleft} \textbf{Case 1: $B \cup \left ( \bigcap_{\alpha \in \lambda}{A_{\alpha}} \right ) \subseteq \bigcap_{\alpha \in \lambda}{(B \cup A_{\alpha})}$} \end{flushleft}
                Let $x \in U$ and $x \in B \cup \left ( \bigcap_{\alpha \in \lambda}{A_{\alpha}} \right )$. This means that $x \in B$ or $x \in \left ( \bigcap_{\alpha \in \lambda}{A_{\alpha}} \right )$. Since $x \in \left ( \bigcap_{\alpha \in \lambda}{A_{\alpha}} \right )$, there exists $\beta \in A_{\beta}$, where $\beta \in \lambda$. Since $x \in \left ( \bigcap_{\alpha \in \lambda}{A_{\alpha}} \right )$, it must be that $x \in B \cup A_{\alpha})$. It follows that, $x \in B \cup \left ( \bigcap_{\alpha \in \lambda}{A_{\alpha}} \right )$ implies $x \in \bigcap_{\alpha \in \lambda}{(B \cup A_{\alpha})}$. This proves the case that $$B \cup \left ( \bigcap_{\alpha \in \lambda}{A_{\alpha}} \right ) \subseteq \bigcap_{\alpha \in \lambda}{(B \cup A_{\alpha})}$$. \\
                
            \begin{flushleft} \textbf{Case 2: $\bigcap_{\alpha \in \lambda}{(B \cup A_{\alpha})} \subseteq B \cup \left ( \bigcap_{\alpha \in \lambda}{A_{\alpha}} \right )$} \end{flushleft} 
                Let $y \in U$ and $y \in \bigcap_{\alpha \in \lambda}{(B \cup A_{\alpha})}$. There exists $\omega \in A_{\omega}$, where $\omega \in \lambda$ such that $y \in B \cup A_{\omega}$. This means that $y \in B$ and $y \in A_{\omega}$. Since $y \in B \cup A_{\omega}$, it must be that $\bigcap_{\alpha \in \lambda}{A_{\alpha}}$. It follows that $y \in \bigcap_{\alpha \in \lambda}{(B \cup A_{\alpha})}$ implies $y \in B \cup \left ( \bigcap_{\alpha \in \lambda}{A_{\alpha}} \right )$. This proves the case that $$\bigcap_{\alpha \in \lambda}{(B \cup A_{\alpha})} \subseteq B \cup \left ( \bigcap_{\alpha \in \lambda}{A_{\alpha}} \right )$$. \\
                
            To sum up, we have proven both of the directions of the set equality, we have proven that $$B \cap \left ( \bigcup_{\alpha \in \lambda}{A_{\alpha}} \right ) = \bigcup_{\alpha \in \lambda}{(B \cap A_{\alpha})}$$. 
        \end{proof}
    \end{property}

\end{theorem}








\newpage
\subsection{Set Complement, the Disjoint Set and Partitions of Sets}

\begin{definition}
Set Complement \\


\begin{figure}[H]
    \centering
        \def \setA{ (0,0) circle (1cm) }

        \def \myrectangle{ (-1.5, -1.5) rectangle (1.5, 1.5) }
            \begin{center}
                \begin{tikzpicture}
                    \draw \myrectangle node[below left]{$U$};
                    
                    \begin{scope} % start of clip scope
                        \fill[gray] \setA ;

                    \end{scope} % end of clip scope
                    
                    \draw \setA node[left] {$A$};

                    \end{tikzpicture}
            \end{center}
    \caption{$\overline{A}$}
    \label{fig:Acomp}
\end{figure}


\begin{tcolorbox}
    Let A be a subset of a universal set $U$. We define the \textbf{Complement} of $A$, denoted $A^c$
    \begin{equation*}
            \overline{A} = \{x \in U | x \notin A \}
        \end{equation*}
\end{tcolorbox}
\end{definition}


\begin{definition}
Disjoint Sets

\begin{tcolorbox}
    Let $A$ and $B$ be subsets of the universal set $U$. The set $A$ and $B$ are said to be \textbf{disjoint} provided that $A \cap B = \emptyset$
\end{tcolorbox}
\end{definition}


\begin{definition}
Pairwise Disjoint Families of Sets

\begin{tcolorbox}
    Let $\lambda$ be a nonempty indexing set, and let $\mathcal{A} = \{ A_{\alpha} | \alpha \in \mathcal{A} \}$ be an indexed family of sets. \\
    We say that $\mathcal{A}$ is \textbf{pairwise disjoint} provided that for all $\alpha$ and $\beta$ in $\lambda$,
        \begin{center}
            If $A_{\alpha} \neq A_{\beta}$, then $A_{\alpha} \cap A_{\beta} = \emptyset$
        \end{center}
\end{tcolorbox}
\end{definition}

\begin{definition}
Partition Sets

\begin{tcolorbox}
    Let $A$ be a non empty set \\
    A collection $\lambda$ of pairwise disjoint nonempty subsets of $A$ with the added property that every element of $A$ belongs to some subset in $\lambda$. \\
    
    Such a collection is called a \textbf{partition} \footnotemark of $A$. A \textbf{partition} of $A$ can also be defined as a collection $\lambda$ of nonempty subsets of $A$ such that every element of $A$ belongs to exactly one subset of $\lambda$. Furthermore a partition of $A$ can be defined as a collection $\lambda$ of subsets of $A$ satisfying the three properties: 
        
        \footnotetext{For example, the set $\bb{Z}$ of integers can be partitioned into the set of even integers and the set of odd integers. The set $\bb{R}$ of real numbers can be partitioned into the set of $\bb{R}^+$ of positive real numbers, the set of $\bb{R}^-$ of negative real numbers, and the set $\{ 0 \}$ consisting of the number $0$. In addition the set $\bb{R}$ can be paritioned into the set of $\mathbb{Q}$ rational numbers and irrational numbers.}
        
        \begin{itemize}
            \item $\alpha \neq \emptyset$ for every set $\alpha \in \lambda$
            \item for every two sets, $\alpha, \beta, \in \lambda$, either $\alpha = \beta$ or $\alpha \cap \beta = \emptyset$
            \item $\bigcup_{\alpha \in \lambda}{A_{\alpha}} = A$
        \end{itemize}
\end{tcolorbox}

\end{definition}

\newpage
\begin{theorem}
Properties for Set Complement and of Disjoint Sets \\

For all sets $A, B$, and $C$ in any universal set $U$ \\

    \begin{enumerate}
        \item $\overline{\overline{A}} = A$ \textit{Law of double complement}
        
        \item $A \cup \overline{A} = U$
        
        \item $A \cap \overline{A} = \emptyset$
        
        \item $\overline{U} = \emptyset$
        
        \item $\overline{\emptyset} = U$ 
        
        \item $A = (A \cup B) \cap (A \cup \overline{B})$
        
        \item $A = (A \cap B) \cup (A \cap \overline{B})$
        
        \item $U = (A \cap B) \cup (\overline{A} \cap B) \cup (A \cap \overline{B}) \cup (\overline{A} \cap \overline{B}$
        
        \item $\overline{A \cap B} = \overline{A} \cup \overline{B}$ \textit{De Morgan Law over Intersection}
        
        \item $\overline{A \cup B} = \overline{A} \cap \overline{B}$ \textit{De Morgan Law over Union}
        
        \item $A \subseteq (A \cup B) \cap (A \cup \overline{B})$ %Proof Completed
        
        \item $A \subseteq B \to A \cap \overline{B} = \emptyset$
    \end{enumerate}



    \newpage
    \begin{property}
    . \\
        \begin{tcolorbox}
            For any sets $A$ and $B$ in any universal set $U$
                \begin{equation*}
                    A \cap \overline{A} = \emptyset
                \end{equation*}
        \end{tcolorbox}
    
        \begin{proof}
            Let $A$ be a subset of some universal set, $U$. By the antisymmetric property of subsets, we will prove this set equality by division of cases:
            
            \begin{flushleft} \textbf{Case 1: $\emptyset \subseteq A \cap \overline{A}$} \end{flushleft}
                By the definition of subsets, $x \in \emptyset$ imples $x \in A \cap \overline{A}$. We will proceed to prove that this proposition is vacuously false. Since $x \in \emptyset$ is false, we can conclude that $x \in A \cap \overline{A}$. This completes the proof that $$ \emptyset \subseteq A \cap \overline{A} $$
    
            
            \begin{flushleft} \textbf{Case 2: $A \subseteq A \cup \emptyset$} \end{flushleft} 
                By the definition of subsets, $x \in A \cap \overline{A}$ implies $x \in \emptyset$. We will prove this case by proving that the negation of the proposition which is $x \in A \cap \overline{A}$ and $x \notin \emptyset$. Now, suppose $x \in A \cap \overline{A}$. This means $x \in A$ and $x \in \overline{A}$. So $x \in A$ and $x \notin A$. This is a contradiction since the element $x$ cannot be both in and not in the set $A$. Hence, proposition cannot be false. This completes the proof that $$ A \subseteq A \cup \emptyset $$ \\
            To sum up, we have proven both of the directions of the set equality, we have proven that $$A \cap \overline{A} = \emptyset$$ 
        \end{proof}
    \end{property}
    
    
    
    \newpage
    \begin{property}
    . \\
    
        \begin{tcolorbox}
            For all sets $A$ and $B$ in any universal set $U$ \\
                \begin{equation*}
                    A \subseteq (A \cup B) \cap (A \cup \overline{B})
                \end{equation*}
        \end{tcolorbox}
        
        \begin{proof}
            Let $A$ and $B$ be subsets of some universal set $U$ such that $x \in U$. Let $x \in A$. By the definition of subsets, for all $x$, if $x \in A$, then $x \in (A \cup B) \cap (A \cup \overline{B})$. Since $x \in A$, then $x \in A \cup B$. Also that $x \in A$ implies $x \in A \cup \overline{B}$. Hence, $x \in (A \cup B) \cap (A \cup \overline{B}$, if $x \in A$. Therefore, we conclude that $$ A \subseteq (A \cup B) \cap (A \cup \overline{B}) $$.          
        \end{proof}
    \end{property}
    
    
    \newpage
    \begin{property}
    . \\
        \begin{tcolorbox}
            For any sets $A$ and $B$ in any universal set $U$
                \begin{equation*}
                    A \subseteq B \to A \cap \overline{B} = \emptyset
                \end{equation*}
        \end{tcolorbox}
    
        \begin{proof}
            Let $A$ and $B$ be subsets of some universal set, $U$, where $x \in A \subseteq B$. By the antisymmetric property of subsets, we will prove this proposition is trivially true by division of cases:
            
            \begin{flushleft} \textbf{Case 1: $\emptyset \subseteq A \cap \overline{B}$} \end{flushleft}
                By the definition of subsets, $x \in \emptyset$ imples $x \in A \cap \overline{B}$. We will proceed to prove that this proposition is vacuously false. Since $x \in \emptyset$ is false, we can conclude that $x \in A \cap \overline{B}$. This completes the proof that $$ \emptyset \subseteq A \cap \overline{B} $$
    
            
            \begin{flushleft} \textbf{Case 2: $A \subseteq A \cup \emptyset$} \end{flushleft} 
                By the definition of subsets, $x \in A \cap \overline{A}$ implies $x \in \emptyset$. We will prove this case by proving that the negation of the proposition which is $x \in A \cap \overline{B}$ and $x \notin \emptyset$. Now, suppose $x \in A \subseteq B$, by the definition of subsets $x \in A$ implies $x \in B$. Also, let $x \in A \cap \overline{B}$. This means that $x \in A$ and $x \in \overline{B}$. Since $x \in \overline{B}$, this means that $x \notin B$. Notice that $x \in B$ and $x \notin B$. This is a contradiction since the element $x$ cannot be both in and not in the set $B$. Hence, proposition cannot be false. This completes the proof that $$ A \subseteq \overline{B} \cup \emptyset $$ \\
            To sum up, we have proven all the trivial cases, we have proven that $$ A \subseteq B \to A \cap \overline{B} = \emptyset $$ 
        \end{proof}
    \end{property}
\end{theorem}





\newpage
\begin{theorem}
Properties for Collection of Set Complement and of Disjoint Sets \\

For all sets $X, Y$, and $Z$ in any universal set $U$. Let $\lambda$ be a nonempty indexing set and let $\cc{A} = \{ A_{\alpha} | \alpha \in \lambda \}$ be an indexed family of sets, then 
    
    \begin{enumerate}
        \item $\overline{\bigcap_{\alpha \in \lambda}{A_{\alpha}}} = \bigcup_{\alpha \in \lambda}{ \overline{A_{\alpha}}}$
        
        \item $\overline{\bigcup_{\alpha \in \lambda}{A_{\alpha}}} = \bigcap_{\alpha \in \lambda}{ \overline{A_{\alpha}}}$
    \end{enumerate}
\end{theorem}



\newpage
\subsection{Set Theoretic Difference}

\begin{definition}
Set Theoretic Difference \\
    \begin{tcolorbox}
        Let $A$ and $B$ be subsets of some universal set $U$. The \textbf{Set Theoretic Difference} \footnotemark of $A$ and $B$, or \textbf{Relative Complement} of $B$ with respect to $A$, denoted $A \setminus B$ and read "$A$ minus $B$ or "the complement of $B$ with respect to $A$" is the set of all elements in $A$ that is not in $B$. That is, \\
         \begin{equation*}
                A \setminus B = \{x \in U | x \in A \text{ and } x \notin B \}
            \end{equation*}
    \end{tcolorbox}
        \footnotetext{Notice, \textit{complement} is a special case of \textit{difference}. Note also that  we need not know $U$ in order to compute  $A \setminus B$}

\end{definition}



\begin{theorem}
Properties of Set Theoretic Differences \\

For all sets $X, Y$, and $Z$ in any universal set $U$ \\

    \begin{enumerate}
        \item $X \setminus Y = X \cap \overline{Y}$
        \item $X \setminus \emptyset = X$
        \item $\emptyset \setminus Y = \emptyset$
        \item $A \setminus U = \emptyset$
        \item $\overline{X} \setminus \overline{Y} = \overline{Y} \setminus \overline{X}$
        \item $X \cup (Y \setminus X) = X \cup Y$    
        \item $(X \setminus Y) \setminus Z = (X \setminus Z) \setminus (Y \setminus Z)$
        \item $X \cup (Y \triangle Z) = (X \cup Y) \triangle (X \cup Z)$ \textit{Distributive Law: Intersection over symmetric difference}
        \item $X \setminus (Y \cup Z) = (X \setminus Y) \cap (X \setminus Z)$ \textit{DeMorgan Law over Union}
        \item $X \setminus (Y \cap Z) = (X \setminus Y) \cup (X \setminus Z)$ \textit{DeMorgan Law over Intersection}
    \end{enumerate}

    \begin{property}    
        \begin{tcolorbox}
            For every two two sets $A$ and $B$
                \begin{equation*}
                    A \setminus B = A \cap \overline{B}
                \end{equation*}
        \end{tcolorbox}


        \begin{proof}
            Let $A$ and $B$ be subsets of some univeral set. We will prove this set equality $A \setminus B = A \cap \overline{B}$ by proving the following two conditional statements:
                \begin{itemize}
                    \item $A \setminus B \subseteq A \cap \overline{B}$
                    \item $A \cap \overline{B} \subseteq A \setminus B$
                \end{itemize}
            
            \textbf{Forward direction: $A \setminus B \subseteq A \cap \overline{B}$ } \\ 
                Let $x \in A \setminus B$. This mean that $x \in A$ and $x \notin B$. Since $x \notin B$, it follows that $x \in \overline{B}$. This proves that $x \in A$ and $x \in \overline{B}$. This means that $x \in A \cap B$, and hence we have proved that $A \setminus B \subseteq A \cap \overline{B}$.
            
            \textbf{Backwards direction: $A \cap \overline{B} \subseteq A \setminus B$} \\
                Let $y \in A \cap \overline{B}$. This means that $y \in A$ and $y \in \overline{B}$. Since $y \in \overline{B}$, this mean that $y \notin B$. This proves that $y \in A$ and $y \notin B$. This means that $A \setminus B$, and hence wehave proved that $A \cap \overline{B} \subseteq A \setminus B$
            
            Since, we have proven both of the directions of the set equality, we have proven that $A \setminus B = A \cap \overline{B}$. 
        \end{proof}
    \end{property}
\end{theorem}


\newpage
\subsection{Symmetric Difference}


\begin{definition}
Symmetric Difference \\

    \begin{tcolorbox}
        Let $A$ and $B$ be subsets of some universal set $U$. We define the  \textbf{symmetric difference} of $A$ and $B$, denoted $A \triangle B$, by the rule:
         \begin{equation*}
                A \triangle B = \{x \in U | (A \setminus B) \cup (B \setminus A) \}
            \end{equation*}
    \end{tcolorbox}    
        
    To be in $A \triangle B$, an object must lie in either in $A \setminus B$ or $B \setminus A$ (or both? Which objects are in both $A \setminus B$ and $B \setminus A$ ?), that is, either in $A$ but not in $B$, or in $B$ but not in $A$. Stated differently, the elements of $A \triangle B$ are objects in one or the other of the set $A$ and $B$, but not in both. 
\end{definition}


\begin{theorem}
Properties of Symmetric Difference \\

For all sets $X, Y$, and $Z$ in any universal set $U$ \\

    \begin{enumerate}
        \item $X \triangle Y = Y \triangle X$ \textit{Commutativity of symmetric difference}
        \item $X \triangle (Y \triangle Z) = (X \triangle Y) \triangle Z$ \textit{Associatity of symmetric differnce}
        \item $X \triangle X = \emptyset$
        \item $X \triangle U = \overline{X}$
        \item $X \triangle \emptyset = X$
        \item $X \triangle Y = (X \cup Y) \setminus (X \cap Y)$
    \end{enumerate}
\end{theorem}


\newpage
\subsection{Cartesian Product}

\begin{definition}
The Set $\{ \{ x \}  \}$ \\

We have mentioned before that the elements of a set may themselves be sets. For example, let us denote $A$, $B$, $C$, $D$, $E$ to be the families living on a particular street in some  given city. Then $\{ A, B, C, D, E \}$ is the set whose elements are the families in question; each if these families in turn a set, a set of persons. The set $\{ A, B, C, D, E \}$ should not be confused with the set 
    \begin{equation*}
        A \cup B \cup C \cup D \cup E
    \end{equation*}
which is defined by 
    \begin{equation*}
        \{ x | x \in A \cup x \in B \cup x \in C \cup x \in D \cup x \in E \}
    \end{equation*}
The elements of the set $A \cup B \cup C \cup D \cup E$ are persons, whereas the elements of the set $\{ A, B, C, D, E \}$ are the family units, and therefore  \textit{the elements of the set $A \cup B \cup C \cup D \cup E$ are not the elements of the set $\{ A, B, C, D, E \}$}. On the other hand, each element of $\{ A, B, C, D, E \}$ considered as a subset of $A \cup B \cup C \cup D \cup E$.. \\

We may sum up the foregoing discussion by remarking that the symbol 
    \begin{equation*}
        \{ \{ x \} \}
    \end{equation*}
denotes the set containing the single element $\{x\}$ and  that $\{x\}$ itself is a set containing the single element $x$. 
\end{definition}


\newpage
\begin{definition}

Before defining the term "\textit{ordered pair}" we require a theorem concerning the set $\{ \{ x \}, \{ x, y \} \}$

\begin{tcolorbox}
    \begin{theorem}
        $\{ \{ x \}, \{ x, y \} \} =  \{ \{ z \}, \{ z, w \} \}$ if and only if $x=z$ and $y=w$
    \end{theorem}
\end{tcolorbox}

\begin{proof}
    Let $x, y, z, w$ be elements of some universal set. By the antisymmetric property of subsets, we will proceed to prove this set equality by the following division of cases: \\
    
    \begin{flushleft} \textbf{Case 1: $\{ \{ x \}, \{ x, y \} \} =  \{ \{ z \}, \{ z, w \} \} \subseteq (x=z) \cap (y=w)$} \end{flushleft}
    
        We assume that $\{ \{ x \}, \{ x, y \} \} =  \{ \{ z \}, \{ z, w \} \}$. Notice, from our assumption we can derive the following series of logically equivalent set theoretic equalities: 
            \begin{align}
                \{x\} = \{x\} \cap \{x,y\} \label{ste1} \\
                \{z\} = \{z\} \cap \{z,w\} \label{ste2} \\
                \{x,y\} = \{x\} \cup \{x,y\} \label{ste3} \\
                \{z,w\} =  \{z\} \cup \{z,w\} \label{ste4} \\
            \end{align}
        Now, adding equation (\ref{ste1}) and equation (\ref{ste2}) yields:
            \begin{align*}
                \{x\} & = \{x\} \cap \{x,y\} = \{z\} \cap \{z,w\} = \{z\} \\
            \end{align*}
        Since $\{x \} = \{ z \}$, we conclude that $x=y$. Similarly, adding equation (\ref{ste3}) and equation (\ref{ste4}) yields:
            \begin{align*}
                \{x,y\} & = \{x\} \cup \{x,y\} = \{z\} \cup \{z,w\} = \{z,w\}
            \end{align*}
        
        Now, using the facts that $\{x,y\} = \{z,w\}$ and  $x=z$ we will proceed by cases to show that $y=w$ \\
        
            \textit{Case 1: $y \neq z$} \\
                Using the assumptions that $y \neq z$ and $\{x,y\} = \{z,w\}$, we know that $y \in \{z,w \}$. So it follows that $y = w$ \\ 
            
            \textit{Case 1: $y = z$} \\
                Using the assumptions that $y \neq z$ and $\{x,y\} = \{z,w\}$, we the following relationship: 
                    \begin{equation*}
                          y = x = \{y \} = \{x\} = \{x,y\} = \{z,w\} \{w\} = \{w\} = z = w
                    \end{equation*}
                Since $y = x = z = w$, we conclude that $y = w  \\
                $
    \begin{flushleft} \textbf{Case 2: $(x=z) \cap (y=w) \subseteq \{ \{ x \}, \{ x, y \} \} =  \{ \{ z \}, \{ z, w \} \}$} \end{flushleft}
    
        We assume that $x=y$ and $y=w$. We will use a direct proof to show that $\{ \{ x \}, \{ x, y \} \} =  \{ \{ z \}, \{ z, w \} \}$. Since $x=z$ and $y=w$, it follows that $x=z=y=w$. We know that $\{x \} = \{z \}$ and $\{x,y \} = \{z,w \}$. Hence, $\{ \{ x \}, \{ x, y \} \} =  \{ \{ z \}, \{ z, w \} \}$. Consequently, if $x=z$ and $y=w$, then $\{ \{ x \}, \{ x, y \} \} =  \{ \{ z \}, \{ z, w \} \}$.
    Since, we have now proved both of the directions of the set equivalency, we have proved $\{ \{ x \}, \{ x, y \} \} =  \{ \{ z \}, \{ z, w \} \}$ if and only if $x=z$ and $y=w$
\end{proof}
\end{definition}


\newpage
\begin{definition}
Ordered Pairs \\

    \begin{tcolorbox}
        The set $\{ \{ x \}, \{ x, y \} \}$ is an \textbf{ordered pair} $x,y$ and is denoted by "$(x,y)$". Thus
        \begin{equation*}
            (x,y) = \{ \{x\}, \{x,y \} \}
        \end{equation*}
    \end{tcolorbox}
    The element $x$ is the \textit{initial component}, the element $y$ is the \textit{final component} of the ordered pair $(x,y)$ \footnotemark
    
    \footnotetext{Notice that if $(x,y)$ is an ordered pair, neither $x$ nor $y$ are elements of $(x,y)$, but $x$ and $y$ are \textbf{components} of $(x,y)$. The elements of $(x,y)$ are $\{x\}$ and $\{x,y\}$. The distinction must be carefully drawn. If $x$ and $y$ are distinct objects, then there are two distinct ordered pairs whose components are $x$ and $y$, namely $(x,y)$ and $(y,x)$ but there is only one set whose elements are precisely $x$ and $y$, namely $\{ x,y \}$, which is the same set as $\{y,x\}$}
    
    Ordered pairs resembles notationally two-elements sets but differ in two important respects:
        \begin{itemize}
            \item The notation $(a,b)$, read "ordered pairs $a$ comma $b$" differs from the set $\{a,b\}$ in that the order in which the elements are listed makes a difference. The ordered pairs $(a,b)$ and $(b,a)$ are different, or unequal, unless $a=b$
            
            \item The same element may be used twice in an ordered pair. That is $(a,a)$ is commonly used mathematical symbol, but $\{a,a\}$ is not (i.e., the latter is always expressed $\{a\}$)
        \end{itemize}
\end{definition}




\begin{definition}
Cartesian Product \\

    \begin{tcolorbox}
        Given sets $A$ and $B$, we define the \textbf{Cartesian Product}, denoted $A \times B$ and read "$A$ cross $B$" by the rule:
            \begin{equation*}
                A \times B = \{ (a,b) | a \in A, b \in B \}
            \end{equation*}
    \end{tcolorbox}
    Thus $A \times B$ consists of all possible distinct ordered pairs whose first elements come from $A$ and whose second element comes from $B$. An object $x$ is an element of $A \times B$ if and only if these exists $a \in A$ and $b \in B$ such that $x = (a,b)$. Note that there is noting in the definition of cardinal products that prevents $A$ and $B$ from being the same set.
    \end{definition}

\begin{definition}
Cartesian Product Notation \\

The notion of couple can be extended as follows, If $a,b,c$ are three objects, we define a new object $(a,b,c)$ by 
    \begin{equation*}
        (a,b,c) = ((a,b), c)
    \end{equation*}
Such an object is called a \textbf{triple}. Two triples, $(a,b,c)$ and $(d,e,f)$, are identical if and only if $(a,b)=(d,e)$ and $c=f$; hence, if and only if $a=d$, $b=e$, and $c=f$. \\

If $a, b, c$ and $d$ are four objects, we define a new object $(a,b,c,d)$ by 
    \begin{equation*}
        (a,b,c,d) = ((a,b,c), d)
    \end{equation*}
Such an object is called a \textbf{quadruple}. Two quadruples, $(a,b,c,d)$ and $(e,f,g,h)$, are identical if and only if $(a,b,c)=(e,f,g)$ and $d=h$; hence, if and only if $a=e$, $b=f$, $c=g$ and $d=h$. \\

Now let $x_1, x_2, x_3, \cdots, x_n$ be $n(n>1)$ objects. Suppose that \footnotemark $(x_1, x_2, x_3, \cdots, x_{n-1})$ as already defined, we then define a new object $(x_1, x_2, x_3, \cdots, x_n)$ by
\footnotetext{For $n=1$ we simply write $x$ instead of $(x)$}

    \begin{equation*}
        (x_1, x_2, x_3, \cdots, x_n) = (((x_1, x_2, x_3, \cdots, x_{n-1}), x_n)
    \end{equation*}
Such an object is usually called an \textbf{n-tuple} (hence, a triple is a 3-tuple; a quadruple is a 4-tuple). As in the case of triples or quadruples, we show that two n-tuples $(x_1, x_2, x_3, \cdots, x_n)$ and $(y_1, y_2, y_3, \cdots, y_n)$ are identical if and only if
    \begin{equation*}
        x_1=y_2, x_2=y_2, x_3=y_3, \cdots x_n=y_n 
    \end{equation*}
Now, let $A_1, A_2, A_3, \cdots A_n$ be $n$ sets. In the same way as we define $A \times B$, we define $A_1 \times A_2 \times A_3 \times \cdots A_n$. Thus, we define $A_1 \times A_2 \times A_3 \times \cdots A_n$ by 
     \begin{equation*}
        z \in A_1 \times A_2 \times A_3 \times \cdots A_n
    \end{equation*}
if and only if $z=(z_1, z_2, z_3, \cdots z_n)$ with 
    \begin{equation*}
        z_1 \in A_1, z_2 \in A_2, z_3 \in A_3, \cdots z_n \in A_n
    \end{equation*}
The set $A_1 \times A_2 \times A_3 \times \cdots A_n$ is called the \textbf{Cartesian Product}, or simply the \textit{product} of the sets $A_1, A_2, A_3, \cdots A_n$. \\


\end{definition}


\newpage
\begin{theorem}
Properties of Cartesian Product \\

For all sets $A, B, C$, and $C$ in any universal set $U$ \\

    \begin{enumerate}
        \item $A \times \emptyset = \emptyset \times A = \emptyset$
        
        \item $A \times B \neq B \times A$
        
        \item $A \times (B \cap C) = (A \times B) \cap (A \times C)$ %Proof Completed
        
        \item $A \times (B \cup C) = (A \times B) \cup (A \times C)$        
        
        \item $(A \cap B) \times C = (A \times C) \cap (B \times C)$        
        
        \item $(A \cup B) \times C = (A \times C) \cup (B \times C)$ 
        
        \item $A \times (B \setminus C) = (A \times B) \setminus (A \times C)$            
        
        \item $(A \times B) \cap (A \times C) = (A \cap B) \times (B \cap C)$
        
        \item $(A \times B) \cup (A \times B) = (A \cup C) \times (B \cup C)$
        
        \item $(A \setminus B) \times C = (A \times C) \setminus (B \times C)$   
        
        \item If $A \subseteq A$, then $A \times B \subseteq A \times B$
        
        \item If $B \subseteq B$, then $A \times B \subseteq A \times B$
        
        \item $A = B \equiv A \times B = B \times A$ %Proof Completed
    \end{enumerate}


    \newpage
    \begin{property}
    . \\
        \begin{tcolorbox}
            For any set $A$ in any universal set $U$
                \begin{equation*}
                    A \times \emptyset = \emptyset \times A = \emptyset
                \end{equation*}
        \end{tcolorbox}
    
        \begin{proof}
            We will prove this result by proving the negation of the proposition which is $$ A \times \emptyset \neq \emptyset \times A \neq \emptyset $$
            Let $\omega \in U$. Using def. Cartesian Product, there exists an element $\omega$ such that $\omega = (x,y)$ with $x \in A$ and $y \in \emptyset$ if and only if $\omega \in A \times \emptyset$. Since $y \in \emptyset$ is falsehood. This is a contradiction becuase $y$ cannot be contained in an empty set. Hence, the proposition cannot be false. This establishes that $A \times \emptyset = \emptyset$. Similary, we can show that $\emptyset \times A = \emptyset$. Therefore, we completed the proof that $$ A \times \emptyset = \emptyset \times A = \emptyset $$
            
        \end{proof}
    \end{property}
    
    
    
    
    \newpage
    \begin{property}
    . \\
        \begin{tcolorbox}
            For any set $A, B$ and $C$ in any universal set $U$
                \begin{equation*}
                    A \times (B \cup C) = (A \times B) \cup (A \times C)
                \end{equation*}
        \end{tcolorbox}
    
        \begin{proof}
            Let $A, B$ and $C$ be subsets of some universal set, $U$. By the antisymmetric property of subsets, we will prove this set equality by division of cases: \\
            
            \begin{flushleft} \textbf{Case 1: $A \times (B \cup C) \subseteq (A \times B) \cup (A \times C)$} \end{flushleft}
                Let $\alpha \in U$, where $\alpha \in A \times (B \cup C)$. By the definition of Cartesian Product, there exists an element $\alpha$, such that $\alpha \in A \times (B \cup C)$ if and only if $\alpha = (x,y)$ with $x \in A$ and $y \in B \cup C$. Since $y \in B \cup C$, this means that $y \in B$ or $y \in C$. Now, proceeding by division of subcases:
    
                
                    \begin{flushleft} \textbf{Case 1 - Sub-Case 1: $x \in B$} \end{flushleft}
                        Let $x \in A$ and $y \in B$. By definition of Cartesian Product, since $\alpha = (x,y)$, it follows that $\alpha \in A \times B$. Using the Simplification Law of Logical Addition, we see that if $\alpha \in A \times B$, then $\alpha \in (A \times B) 
                        \cup (A \times C)$. Hence, we have established that $\alpha \in (A \times B) 
                        \cup (A \times C)$.  
                    
                    \begin{flushleft} \textbf{Case 1 - Sub-Case 2: $x \in C$} \end{flushleft}  
                        Let $x \in A$ and $y \in C$. By definition of Cartesian Product, since $\alpha = (x,y)$, it follows that $\alpha \in A \times C$. Using the Simplification Law of Logical Addition, we see that if $\alpha \in A \times C$, then $\alpha \in (A \times B) 
                        \cup (A \times C)$. Hence, we have established that $\alpha \in (A \times B) 
                        \cup (A \times C)$. \\
                        
                We see in both sub cases that $A \times (B \cup C)$ implies $\alpha \in (A \times B) \cup (A \times C)$, consequently, we proved that $$A \times (B \cup C) \subseteq (A \times B) \cup (A \times C)$$
                
            \begin{flushleft} \textbf{Case 2: $(A \times B) \cup (A \times C) \subseteq A \times (B \cup C)$} \end{flushleft} 
                Let $\omega \in U$, where $\omega \in (A \times B) \cup (A \times C)$. This means that $\omega \in (A \times B$ or $\omega \in A \times C$. Now, proceeding by division of subcases:
                
                
                By the definition of Cartesian Product, there exists an element $\alpha$, such that $\alpha \in A \times (B \cup C)$ if and only if $\alpha = (x,y)$ with $x \in A$ and $y \in B \cup C$. Since $y \in B \cup C$, this means that $y \in B$ or $y \in C$. Now, proceeding by division of subcases:
                
                    \begin{flushleft} \textbf{Case 2 - Sub-Case 1: $y \in (A \times B)$} \end{flushleft}
                        Let $\omega \in (A \times B)$. By definition of Cartesian Product, there exists an element $\omega$ such that $\omega \in A \times B$ if and only if $\omega = (s,t)$, with $s \in A$ and $t \in B$. Using the Simplification Law of Logical Addition, we see that $t \in B \cup C$ if $t \in B$. Now, it follows that $\omega = (s,t)$ with $x \in A$ and $t \in (B \cup C)$. Hence we have shown that $(A \times B) \cup (A \times C)$. 
                    
                    \begin{flushleft} \textbf{Case 2 - Sub-Case 2: $y \in (A \times C)$} \end{flushleft}  
                        Let $\omega \in (A \times C)$. By definition of Cartesian Product, there exists an element $\omega$ such that $\omega \in A \times C$ if and only if $\omega = (s,t)$, with $s \in A$ and $t \in C$. Using the Simplification Law of Logical Addition, we see that $t \in A \cup C$ if $t \in C$. Now, it follows that $\omega = (s,t)$ with $x \in A$ and $t \in (B \cup C)$. Hence we have shown that $(A \times B) \cup (A \times C)$. 
                
                We see in both sub cases that $\omega \in (A \times B) \cup (A \times C)$ implies $\omega \in A \times (B \cup C)$, consequently, we proved that $$(A \times B) \cup (A \times C) \subseteq A \times (B \cup C)$$ \\ 
            To sum up, we have proven both of the directions of the set equality, we have proven that $A \times (B \cup C) = (A \times B) \cup (A \times C)$. 
        \end{proof}
    \end{property}
    
    
    
    
    \newpage
    \begin{property}
    . \\
        \begin{tcolorbox}
            For any set $A, B$ and $C$ in any universal set $U$
                \begin{equation*}
                    A = B \equiv A \times B = B \times A
                \end{equation*}
        \end{tcolorbox}
    
        \begin{proof}
            Let $A$ and $B$ be subsets of some universal set, $U$. By the antisymmetric property of subsets, we will prove this set equality by division of cases:
            
            \begin{flushleft} \textbf{Case 1: $A \times B = B \times A \subseteq A = B$} \end{flushleft}
                Let $\alpha \in U$, where $\alpha \in (A \times B = B \times A \subseteq A = B)$. This means that if $\alpha \in A \times B = B \times A$, then $\alpha \in A = B$. Since $\alpha \in A = B$, this means that $\alpha in A \subseteq B$ and $\alpha \in B \subseteq B$. We will proceed to prove this propistion is trivially true by division of case:
    
                
                    \begin{flushleft} \textbf{Case 1 - Sub-Case 1: $x \in A \subseteq B$} \end{flushleft}
                        Let $\alpha \in A \subseteq B$. By def. Cartesian Product, let $\alpha = (x,y)$ with $x \in A$ and $y \in B$. We know that $\alpha \in A \times B$. Now let $\alpha \in (A \times B = B \times A)$, we say that $\alpha \in B \times A$. This imples $x \in B$. Hence, we have established that $x \ B$ if $x \in A$.
                    
                    \begin{flushleft} \textbf{Case 1 - Sub-Case 2: $x \in B \subseteq A$} \end{flushleft}  
                        Let $\alpha \in B \subseteq A$. By def. Cartesian Product, let $\alpha = (x,y)$ with $x \in B$ and $y \in A$. We know that $\alpha \in B \times A$. Now let $\alpha \in (A \times B = B \times A)$, we say that $\alpha \in A \times B$. This implies $x \in A$. Hence, we have established that $x \ A$ if $x \in B$. \\
                We see that $\alpha \in A \subseteq B$ and $\alpha \in B \subseteq A$, consequently, we have trivially proven that $$ A \times B = B \times A \subseteq A = B$$
                
            \begin{flushleft} \textbf{Case 2: $A=B \subseteq A \times B = B \times A$} \end{flushleft} 
                Let $\omega \in U$, where $\omega \in A = B \subseteq A \times B = B \times A$. This means that if $\omega \in A=B$, then $\omega \in A \times B = B \times A$. Since $\omega \in A = B$, it follows that $\omega \in A \subseteq B$ and $\omega \in B \subseteq A$, proceed by cases:
                
                    \begin{flushleft} \textbf{Case 2 - Sub-Case 1: $\omega \in A \subseteq B$} \end{flushleft}
                        Let $\omega \in A \subseteq B$. By def. Cartesian Product, suppose $\omega = (s,t)$ then $s \in A$ and $t \in B$, we know that $\omega \in A \times B$. Thus $\omega \in A \subseteq B$ implies $\omega \in A \times B$
                    
                    \begin{flushleft} \textbf{Case 2 - Sub-Case 2: $y \in B \subseteq A$} \end{flushleft}  
                        Let $\omega \in B \subseteq A$. By def. Cartesian Product, suppose $\omega = (s,t)$ then $s \in B$ and $t \in A$, we know that $\omega \in B \times A$. Thus $\omega \in B \subseteq A$ implies $\omega \in B \times A$. \\
                
                Hence, it follows that if $\omega \in A \subseteq B \to A \times B$ and $\omega \in B \subseteq A \to B \times A$, then $\omega \in A=B \subseteq A \times B = B \times A$. Thus, we have directly proved that $$ A=B \subseteq A \times B = B \times A $$\\  
            To sum up, we have proven both of the directions of the set equality, we have proven that $$ A = B \equiv A \times B = B \times A $$
        \end{proof}
    \end{property}
\end{theorem}


\newpage
\section{Proof Structures: Set Theory}

\begin{definition}
Choose-an-Element Strategy \\

    The \textbf{Choose-an-Element} strategy is used when we encounter a universal quantifer of the form: 
        \begin{center}
            For each element with a given property, something happens
        \end{center}
    Since most statement with a universal quantifier can be expressed in the form of a conditional statement, this statement coulf have the following equaivalent form: 
        \begin{center}
            If an element has a given property, then something happens
        \end{center}
    For instance, consider the proposition:
        \begin{center}
            Let $S$ be the set of all integers that are multiples of $6$, and let $T$ be the set of all even integers, then $S$ is a set of $T$
        \end{center}
    We can rewrite the proposition in the form:
        \begin{center}
            Each element of $S$ is an element of $T$ or more precisely, If $x \in S$, then $x \in T$
        \end{center}
    
    In this case, the "element" is an integer, the "given property" is that it is an element of $S$, and the "something that happens" is that the element is also an element of $T$. One way to approach this is to create a list of all elements with the given property and verify that for each one, the "something happens." When the list is short this is a reasonable approach. However, as in the case, when the list is infinite, this apporach is not practical. \\
    
    We overcome this difficulty by using the \textbf{Choose-an-Element} strategy, where we choose an arbitrary element with the given property. So in this case, we choose an element $x$ that is a multiple of $6$. We cannot use a specific multiple of $6$ (such as $6, 24, \cdots$) but rather the only thing we can assume is thta the integer satisfies the property that it is a multiple of $6$. This is the key part of this strategy.
        \begin{center}
            \textit{Whenever we choose an arbitrary element with a given property, we are not selecting a specific element. Rather, the onyl thing we can assume about the element is the given property.}
        \end{center}
    
    It is important to realize that once we have the arbitrary element, we have additional tools to prove or disprove the proposition. We will proceed to formally prove the proposition to illustrate the strategy. 
        \newpage
        \begin{tcolorbox}
            \begin{theorem}
                Let $S$ be the set of all integers that are multiples of $6$, and let $T$ be the set of all even integers. Then $S$ is a subset of $T$.
            \end{theorem}
        \end{tcolorbox}
        
        \begin{proof}
            Let $S$ be the set of all integers that are multiples of $6$, and let $T$ be the set of all even integers. We will show that $S$ is the a subset of $T$ by showing that if an integer $x$ us an element of $S$, then it is also an element of $T$. \\
            
            Let $x \in S$. This means that $x$ is a multiple of $6$ therefore, there exists an integer $m$ such that $x=6m$. Since the factors of $6 = 2 \m 3$, this equation can be written in the form $x = 2(3m)$. \\
            By he closure properties of integers, $3m$ is an integer. Hence, $x$ must be even. Therefore, we have shown that if $x$ is an element of $S$ then $x$ is an element of $T$, and hence that $S \subseteq T$
        \end{proof}
\end{definition}

\begin{theorem}
FACT 7: Some Important Statements of Implications \\

For all sets $X, Y$, and $Z$ in any universal set $U$ \\

    \begin{enumerate}
        \item If $(X \subseteq Y) \wedge (X \subseteq Z)$, then $X \subseteq (Y \cap Z)$
        \item If $(X \subseteq Z) \wedge (Z \subseteq Z)$, then $(X \cup Y) \subseteq Z)$
        \item If $X \subseteq Y$, then $Y = X \cup (Y \setminus X)$
        \item If $X \subseteq Z$, then $X \cup (Y \cap Z) = (X \cup Y) \cap Z$
        \item If $(X \cap Y = X \cap Z) \wedge (X \cup Y = X \cup Z)$, then $Y = Z$
        \item If $(X \cap Y = X \cap Z) \wedge (\overline{X} \cap Y = \overline{X} \cap Z)$, then $Y = Z$
        \item If $(X \cup Y = X \cup Z) \wedge (\overline{X} \cup Y = \overline{X} \cup Z)$, then $Y = Z$
        \item If $X \cap Y = \emptyset$, then $X \triangle Y = X \cup Y$
        \item If $(X \times Y = X \times Z) \wedge (X \neq \emptyset)$, then $Y = Z$
        \item If $(X \times Y = Y \times X) \wedge (X \neq \emptyset) \wedge (Y \neq \emptyset)$, then $X = Y$
        \item If $Y \times Z = \emptyset$, then $(Y = \emptyset) \vee (Z = \emptyset)$
    \end{enumerate}
\end{theorem}


\newpage 
\begin{definition}
Proving Set Equality \\

    One way to prove that two sets are equal using the following theorem. 
        \begin{theorem}
            Let $A$ and $B$ be subsets if some universal set $U$. Then $A = B$ if and only if $A \subseteq B$ and $B \subseteq A$
        \end{theorem}
    And prove each of the two sets is a subset of the other set. To illustrate this technique see the sample proof below. 
    
        \begin{tcolorbox}
            \begin{theorem}
                Let $A$ and $B$ be subsets of some univeral set. Then $A \setminus (A \setminus B) = A \cap B$
            \end{theorem}
        \end{tcolorbox}
        
        \begin{proof}
            Let $A$ and $B$ be subsets of some univeral set. We will prove this set equality $A \setminus (A \setminus B) = A \cap B$ by proving that $A \setminus (A \setminus B) \subseteq A \cap B$ and $A \cap B \subseteq A \setminus (A \setminus B)$. \\
            
            \begin{flushleft} \textbf{Case 1: $A \setminus (A \setminus B) \subseteq A \cap B$} \end{flushleft}
                Let $x \in A \setminus (A \setminus B) $. This means that $x \in A$ and $x \notin (A \setminus B)$. We know that an element is in $A \setminus B$ if and only if it is in $A$ and not in $B$. Since $x \notin (A \setminus B)$, we conclude that $x \notin A$ or $x \in B$. However, we also know that $x \in A$ and so we conclude that $x \in B$. This proves that $x \in A$ and $x \in B$. This mean that $x \in A \cap B$, and hence we have proved that $A \setminus (A \setminus B) \subseteq A \cap B$. \\
            
            \begin{flushleft} \textbf{Case 2: $A \cap B \subseteq A \setminus (A \setminus B)$} \end{flushleft} 
                Let $y \in A \cap B$. This mean that $y \in A$ and $y \in B$. We note note that $y \in A \setminus B$ if and only if $y \in A$ and $y \notin B$ and hence, $y \notin A \setminus B$ if and only if $y \notin A$ or $y \in B$. Since we have proved that $y \in B$, we conclude that $y \notin A \setminus B$ and hence, we hace established that $y \in A$ and $y \notin A \setminus B$. This proves that if $y \in A \cap B$, then $y \in A \setminus (A \setminus B)$ and hence, $A \cap B \subseteq A \setminus (A \setminus B)$. \\
            
            Since, we have proven both of the directions of the set equality, we have proven that $A \setminus (A \setminus B) = A \cap B$. 
        \end{proof}
\end{definition}

\newpage
\begin{theorem}
Some Important Equivalence Statements \\

For all sets $X, Y$, and $Z$ in any universal set $U$ \\

    \begin{enumerate}
        \item $X \subseteq $ if and only if $\overline{Y} \subseteq \overline{X}$
        \item $X \subseteq Y $ if and only if $(X \cup Y) = Y$
        \item $X \subseteq Y $ if and only if $ (X \cap Y) = X$
        \item $X \subseteq Y $ if and only if $ X \setminus Y = \emptyset$
        \item $X \subseteq Y $ if and only if $ X \cap \overline{Y} = \emptyset$
        \item $X \subseteq Y $ if and only if $ \overline{X} \cup Y = U$
    \end{enumerate}
\end{theorem}




\newpage
\begin{definition}
Proving Disjoint Sets \\

    Proving a set is empty is a case where a proof by contrapositve or contradiction could be reasonable approaches. This illustrate this fact, let us assume the proposition we are trying to prove is the following form: 
        \begin{center}
            If $P$, the $T = \emptyset$
        \end{center}
    If we choose to prove the contrapostive or use a proof by contradiction, we will assume that  $T \neq \emptyset$. These methods can be outlined as folows:
        \begin{itemize}
            \item The contrapositive of "If $P$, the $T = \emptyset$" is "If $T \neq \emptyset$, then $neg P$." So in this case, we would assume $T \neq \emptyset$ and try to prove $\neg P$.
            \item Using a proof by contradiction, wee would assume $P$ and assume that $T \neq \emptyset$. From this two assumptions, we would attempt to derive a contradiction. 
        \end{itemize}
    One advantage of these methods is that when we assume that $T \neq \emptyset$, we know that there exists an element in $T$. We can then use the that element in the rest of the proof. To illustrate this technique see the sample proof below. 
    
        \newpage
        \begin{tcolorbox}
            \begin{theorem}
                Let $A$ and $B$ be subsets of some universal set. Then $A \subseteq B$ if and only if $A \cap \overline{B} = \emptyset$
            \end{theorem}
        \end{tcolorbox}
        
        \begin{proof}
             Let $A$ and $B$ be subsets of some univeral set. We will prove this bi-conditional statement, $A \subseteq B$ if and only if $A \cap \overline{B} = \emptyset$, by proving the following two conditional statements:
                \begin{itemize}
                    \item If $A \subseteq B$, then $A \cap \overline{B} = \emptyset$
                    \item If $A \cap \overline{B} = \emptyset$, then $A \subseteq B$
                \end{itemize}
            
            \textit{For the forwards direction} \\
                We will first prove that if $A \subseteq B$, then $A \cap \overline{B} = \emptyset$, by proving its contapositive. That is, we will prove
                    \begin{center}
                        If $A \cap \overline{B} \neq \emptyset$, then $A \nsubseteq B$
                    \end{center}
                
                So we assume that $A \cap \overline{B} \neq \emptyset$. We will prove that $A \nsubseteq B$ by proving that there must exist an element $x4$ such that $x \in A$ and $x \notin B$. \\
                
                Since $A \cap \overline{B} \neq \emptyset$, there exists an element $x$ that is in $A \cap \overline{B}$. This mean that $x \in A$ and $x \in \overline{B}$. Now, the fact that $x \in \overline{B}$ mean that $x \notin B$. Hence we conclude that $x \in A$ and $x \notin B$. This means that $A \nsubseteq B$, and hence, we have proved that if $A \cap \overline{B} \neq \emptyset$, then $A \nsubseteq B$, and therefore, we have proved that if $A \subseteq B$, then $A \cap \overline{B} = \emptyset$. 
            
            \textit{For the backwards direction} \\ 
                We will prove that if $A \cap \overline{B} = \emptyset$, then $A \subseteq B$, by proving its contradiction. That is, we will prove
                    \begin{center}
                        $A \cap \overline{B} = \emptyset$ and $A \nsubseteq B$
                    \end{center}
                So we assume that $A \cap \overline{B} = \emptyset$ and $A \nsubseteq B$. Since $x\in A$, this implies that $x \notin B$. The fact that $x \notin B$ means that $x \in \overline{B}$. Hence, the set containing $x \in A$ and $x \in \in \o\overline{B}$ is a non-empty set. This is a contradiction since $A \cap \overline{B} = \emptyset$. Hence, the proposition cannot be false. Consequently, if $A \cap \overline{B} = \emptyset$, then $A \subseteq B$
                
                Since, we have proven both of the directions of the set equality, we have proven that $A \subseteq B$ if and only if $A \cap \overline{B} = \emptyset$.          
        \end{proof}
\end{definition}


\newpage
\begin{definition}
Proving a Chain of Equivalent Statements \\

    When we have a list of three statements $P$, $Q$, and $R$ such that each statement in the list is equivalent to the other two statements in the list, we say that the three statements are equivalent. This means that each of the statements in the list implies each of the other statements in the list. \\
    The basic idea to prove a sequence of conditional statements so that there is an unbroken chain of conditional statements from each statement to every other statement. To illustrate this technique see the sample proof below. 

        \begin{tcolorbox}
            \begin{theorem}
                Let $A$ and $B$ be subsets of some universal set. The following are equivalent:  
                    \begin{enumerate}
                        \item $A \subseteq B$
                        \item $A \cap \overline{B} = \emptyset$
                        \item $\overline{A} \cup B = U$
                    \end{enumerate}
            \end{theorem}
        \end{tcolorbox}
        
        \begin{proof}
             Let $A$ and $B$ be subsets of some univeral set. We will proceed by cases according to the logically equivalnecy of the proposition: \\  
             
             To prove that these are equivalent conditions, we will prove that the following conditional statements: \\
             
             
             \textit{Case 1: $A \subseteq B$, then $A \cap \overline{B} = \emptyset$} \\
                 We will first prove that if $A \subseteq B$, then $A \cap \overline{B} = \emptyset$, by proving its contapositive. That is, we will prove
                    \begin{center}
                        If $A \cap \overline{B} \neq \emptyset$, then $A \nsubseteq B$
                    \end{center}
                
                So we assume that $A \cap \overline{B} \neq \emptyset$. We will prove that $A \nsubseteq B$ by proving that there must exist an element $x4$ such that $x \in A$ and $x \notin B$. \\
                
                Since $A \cap \overline{B} \neq \emptyset$, there exists an element $x$ that is in $A \cap \overline{B}$. This mean that $x \in A$ and $x \in \overline{B}$. Now, the fact that $x \in \overline{B}$ mean that $x \notin B$. Hence we conclude that $x \in A$ and $x \notin B$. This means that $A \nsubseteq B$, and hence, we have proved that if $A \cap \overline{B} \neq \emptyset$, then $A \nsubseteq B$, and therefore, we have proved that if $A \subseteq B$, then $A \cap \overline{B} = \emptyset$. \\
             
            \textit{Case 2: If $A \cap \overline{B} = \emptyset$, then $\overline{A} \cup B = U$} \\
                We will secondly prove that if $A \cap \overline{B} = \emptyset$, then $\overline{A} \cup B = U$, using set algebra. We assume that $A \cap \overline{B} = \emptyset$ and use the fact that $\overline{\emptyset} = U$. We see that 
                    \begin{equation*}
                        \overline{A \cap \overline{B}} = \overline{\emptyset}
                    \end{equation*}
                Then, using one of DeMorgan Laws, we obtain
                    \begin{align*}
                        U & = \overline{A} \cup \overline{\overline{B}}
                            & = \overline{A} \cup B
                    \end{align*}
                Hence, this completes the proof that if $A \cap \overline{B} = \emptyset$, then $\overline{A} \cup B = U$ \\
                    
            \textit{Case 3: If $\overline{A} \cup B = U$, then $A \subseteq B$} \\
                We will thirdly prove that if $\overline{A} \cup B = U$, then $A \subseteq B$. We assume that $\overline{A} \cup B = U$ and since will prove that $A \subseteq B$ by proving that every element of $A$ must be in set $B$. \\
                
                So let $x \in A$. Then we know that $x \notin \overline{A}$. However $x \in U$ and since $\overline{A} \cup B = U$, we can conclude that $x \in \overline{A} \cup B$. Since $x \in \overline{A}$, we conclude that $x \in B$. This proves that if $\overline{A} \cup B = U$, then $A \subseteq B$ \\
             
             Since we have proved that all three conditional statements, we have proved that the three conditional statements are equivalent. 
        \end{proof}
\end{definition}

\newpage
\begin{theorem}
Some Important Statements of Implications \\

For all sets $X, Y$, and $Z$ in any universal set $U$ \\

    \begin{enumerate}
        \item If $(X \subseteq Y) \wedge (X \subseteq Z)$, then $X \subseteq (Y \cap Z)$
        \item If $(X \subseteq Z) \wedge (Z \subseteq Z)$, then $(X \cup Y) \subseteq Z)$
        \item If $X \subseteq Y$, then $Y = X \cup (Y \setminus X)$
        \item If $X \subseteq Z$, then $X \cup (Y \cap Z) = (X \cup Y) \cap Z$
        \item If $(X \cap Y = X \cap Z) \wedge (X \cup Y = X \cup Z)$, then $Y = Z$
        \item If $(X \cap Y = X \cap Z) \wedge (\overline{X} \cap Y = \overline{X} \cap Z)$, then $Y = Z$
        \item If $(X \cup Y = X \cup Z) \wedge (\overline{X} \cup Y = \overline{X} \cup Z)$, then $Y = Z$
        \item If $X \cap Y = \emptyset$, then $X \triangle Y = X \cup Y$
        \item If $(X \times Y = X \times Z) \wedge (X \neq \emptyset)$, then $Y = Z$
        \item If $(X \times Y = Y \times X) \wedge (X \neq \emptyset) \wedge (Y \neq \emptyset)$, then $X = Y$
        \item If $Y \times Z = \emptyset$, then $(Y = \emptyset) \vee (Z = \emptyset)$
    \end{enumerate}
\end{theorem}




\newpage.
\section{Summary of Set Theory Theorems}

\begin{definition}
FACT 1: Properties of Set Equality and of Subsets \\

    For all sets $X, Y$, and $Z$ in any universal set $U$ \\
    
    \begin{enumerate}
        \item $X=X$ \textit{Reflexive Property of Equality}
        \item $X \subseteq X$ \textit{Reflexive Property of Subset Relation}
        \item If $X=Y$, then $Y=X$ \textit{Summetric Property of Equality} 
        \item $X=Y \equiv (X \subseteq Y) \wedge (Y \subseteq X)$ \textit{Antisymmetric Property of Subsets}
        \item If $(X=Y) \wedge (Y=Z)$, then $X=Z$ \textit{Transitive Property of Equality} 
        \item If $(X \subseteq Y) \wedge (Y \subseteq\ Z)$, then $X \subseteq Z$ \textit{Transitive Property of Subsets} 
        \item $\emptyset \subseteq X$
        \item $X \subseteq U$  
    \end{enumerate}
\end{definition}



\begin{definition}
FACT 2A: Properties for Union and of Intersection \\

For all sets $X, Y$, and $Z$ in any universal set $U$ \\
    
    \begin{enumerate}
        \item $X \cup X = X$ \textit{Idempotent law for union}
        \item $X \cap X = X$ \textit{Idempotent law for intersection}
        \item $X \cap \emptyset = X$ \textit{Identity for union}
        \item $X \cap \emptyset = \emptyset$
        \item $X \cup U = U$
        \item $X \cup Y = Y \cup X$ \textit{Commutative law of union}
        \item $X \cap Y = Y \cap X$ \textit{Cummulative law for intersection}
        \item $X \cup (Y \cup Z) = (X \cup Y) \cup Z$ \textit{Associative law for union}
        \item $X \cap (Y \cap Z) = (X \cap Y) \cap Z$ \textit{Associative law for intersection}
        \item $X \cup (Y \cap Z) = (X \cup Y) \cap (X \cup Z)$ \textit{Distributive law: Union over intersetion}
        \item $X \cap (Y \cup Z) = (X \cap Y) \cup (X \cap Z)$ \textit{Distributive Law: Intersection over union}
        \item $X \subseteq (X \cup Y)$
        \item $(X \cap Y) \subseteq X$
    \end{enumerate}
\end{definition}


\begin{definition}
Fact 2B: Properties for Collection of Union and Intersection \\

For all sets $X, Y$, and $Z$ in any universal set $U$ \\
Let $\lambda$ be a nonempty indexing set \\
Let $\mathcal{A} = \{ A_{\alpha} | \alpha \in \lambda \}$ be an indexed family of sets, then 
    
    \begin{enumerate}
        \item For each $\beta \in \lambda$, $\bigcap_{\alpha \in \lambda}{A_{\alpha}} \subseteq A_{\beta}$ 
        \item For each $\beta \in \lambda$, $A_{\beta} \subseteq \bigcup_{\alpha \in \lambda}{A_{\alpha}}$ 
        \item $\overline{\bigcap_{\alpha \in \lambda}{A_{\alpha}}} = \bigcup_{\alpha \in \lambda}{ \overline{A_{\alpha}}}$
        \item $\overline{\bigcup_{\alpha \in \lambda}{A_{\alpha}}} = \bigcap_{\alpha \in \lambda}{ \overline{A_{\alpha}}}$
        \item $B \cap \left (\bigcup_{\alpha \in \lambda}{A_{\alpha}} \right ) = \bigcup_{\alpha \in \lambda}{(B \cap A_{\alpha})}$
        \item $B \cup \left ( \bigcap_{\alpha \in \lambda}{A_{\alpha}} \right ) = \bigcap_{\alpha \in \lambda}{(B \cup A_{\alpha})}$
    \end{enumerate}
\end{definition}



\begin{definition}
FACT 3: Properties for Set Complement and of Disjoint \\

For all sets $X, Y$, and $Z$ in any universal set $U$ \\

    \begin{enumerate}
        \item $\overline{\overline{X}} = X$ \textit{Law of double complement}
        \item $X \cup \overline{X} = U$
        \item $X \cap \overline{X} = \emptyset$
        \item $\overline{U} = \emptyset$
        \item $\overline{\emptyset} = U$ 
        \item $X = (X \cup Y) \cap (X \cup \overline{Y})$
        \item $X = (X \cap Y) \cup (X \cap \overline{Y})$
        \item $U = (X \cap Y) \cup (\overline{X} \cap Y) \cup (X \cap \overline{Y}) \cup (\overline{X} \cap \overline{Y}$
        \item $\overline{X \cap Y} = \overline{X} \cup \overline{Y}$ \textit{DeMorgan Law over Intersection}
        \item $\overline{X \cup Y} = \overline{X} \cap \overline{Y}$ \textit{DeMorgan Law over Union}
    \end{enumerate}
\end{definition}





\begin{definition}
FACT 4: Properties of Set Theoretic Differences \\

For all sets $X, Y$, and $Z$ in any universal set $U$ \\

    \begin{enumerate}
        \item $X \setminus Y = X \cap \overline{Y}$
        \item $X \setminus \emptyset = X$
        \item $\emptyset \setminus Y = \emptyset$
        \item $A \setminus U = \emptyset$
        \item $\overline{X} \setminus \overline{Y} = \overline{Y} \setminus \overline{X}$
        \item $X \cup (Y \setminus X) = X \cup Y$    
        \item $(X \setminus Y) \setminus Z = (X \setminus Z) \setminus (Y \setminus Z)$
        \item $X \cup (Y \triangle Z) = (X \cup Y) \triangle (X \cup Z)$ \textit{Distributive Law: Intersection over symmetric difference}
        \item $X \setminus (Y \cup Z) = (X \setminus Y) \cap (X \setminus Z)$ \textit{DeMorgan Law over Union}
        \item $X \setminus (Y \cap Z) = (X \setminus Y) \cup (X \setminus Z)$ \textit{DeMorgan Law over Intersection}
    \end{enumerate}
\end{definition}



\begin{definition}
FACT 5: Properties of Symmetric Difference \\

For all sets $X, Y$, and $Z$ in any universal set $U$ \\

    \begin{enumerate}
        \item $X \triangle Y = Y \triangle X$ \textit{Commutativity of symmetric difference}
        \item $X \triangle (Y \triangle Z) = (X \triangle Y) \triangle Z$ \textit{Associatity of symmetric differnce}
        \item $X \triangle X = \emptyset$
        \item $X \triangle U = \overline{X}$
        \item $X \triangle \emptyset = X$
        \item $X \triangle Y = (X \cup Y) \setminus (X \cap Y)$
    \end{enumerate}
\end{definition}

\begin{definition}
Fact 6: Properties of Cartesian Product \\

For all sets $A, B, C, X$, and $Y$ in any universal set $U$ \\

    \begin{enumerate}
        \item $A \times \emptyset = \emptyset \times B = \emptyset$
        \item $A \times (B \cap C) = (A \times B) \cap (A \times C)$
        \item $A \times (B \cup C) = (A \times B) \cap (A \times C)$        
        \item $(A \cap B) \times C = (A \times C) \cap (B \times C)$        
        \item $(A \cup B) \times C = (A \times C) \cup (B \times C)$ 
        \item $A \times (B \setminus C) = (A \times B) \setminus (A \times C)$            
        \item $(A \setminus B) \times C = (A \times C) \setminus (B \times C)$   
        \item If $X \subseteq A$, then $X \times B \subseteq A \times B$
        \item If $Y \subseteq B$, then $A \times Y \subseteq A \times B$
        \item $(X \cup Y) \times Z = (X \times Z) \cup (Y \times Z)$
        \item $(X \cap Y) \times Z = (X \times Z) \cap (Y \times Z)$
        \item $(X \setminus Y) \times Z = (X \times Z) \setminus (Y \times Z)$
    \end{enumerate}
\end{definition}



\begin{definition}
FACT 7: Some Important Statements of Equivalence \\

For all sets $X, Y$, and $Z$ in any universal set $U$ \\

    \begin{enumerate}
        \item $X \subseteq $ if and only if $\overline{Y} \subseteq \overline{X}$
        \item $X \subseteq Y $ if and only if $(X \cup Y) = Y$
        \item $X \subseteq Y $ if and only if $ (X \cap Y) = X$
        \item $X \subseteq Y $ if and only if $ X \setminus Y = \emptyset$
        \item $X \subseteq Y $ if and only if $ X \cap \overline{Y} = \emptyset$
        \item $X \subseteq Y $ if and only if $ \overline{X} \cup Y = U$
    \end{enumerate}
\end{definition}



\begin{definition}
FACT 8: Some Important Statements of Implications \\

For all sets $X, Y$, and $Z$ in any universal set $U$ \\

    \begin{enumerate}
        \item If $(X \subseteq Y) \wedge (X \subseteq Z)$, then $X \subseteq (Y \cap Z)$
        \item If $(X \subseteq Z) \wedge (Z \subseteq Z)$, then $(X \cup Y) \subseteq Z)$
        \item If $X \subseteq Y$, then $Y = X \cup (Y \setminus X)$
        \item If $X \subseteq Z$, then $X \cup (Y \cap Z) = (X \cup Y) \cap Z$
        \item If $(X \cap Y = X \cap Z) \wedge (X \cup Y = X \cup Z)$, then $Y = Z$
        \item If $(X \cap Y = X \cap Z) \wedge (\overline{X} \cap Y = \overline{X} \cap Z)$, then $Y = Z$
        \item If $(X \cup Y = X \cup Z) \wedge (\overline{X} \cup Y = \overline{X} \cup Z)$, then $Y = Z$
        \item If $X \cap Y = \emptyset$, then $X \triangle Y = X \cup Y$
        \item If $(X \times Y = X \times Z) \wedge (X \neq \emptyset)$, then $Y = Z$
        \item If $(X \times Y = Y \times X) \wedge (X \neq \emptyset) \wedge (Y \neq \emptyset)$, then $X = Y$
        \item If $Y \times Z = \emptyset$, then $(Y = \emptyset) \vee (Z = \emptyset)$
    \end{enumerate}
\end{definition}





\newpage
\section{Proof Repository: Set Theory}

\begin{example}
Source: \cite[Chap.6, S.6.1, Result 6.6]{gray} \\ 

Prove that for every two sets $A$ and $B$ that $A \setminus B = A \cap \overline{B}$
    \begin{tcolorbox}
        \begin{theorem}
            For every two two sets $A$ and $B$
                \begin{equation*}
                    A \setminus B = A \cap \overline{B}
                \end{equation*}
        \end{theorem}
    \end{tcolorbox}


    \begin{proof}
        Let $A$ and $B$ be subsets of some univeral set. We will prove this set equality $A \setminus B = A \cap \overline{B}$ by proving the following two conditional statements:
            \begin{itemize}
                \item $A \setminus B \subseteq A \cap \overline{B}$
                \item $A \cap \overline{B} \subseteq A \setminus B$
            \end{itemize}
        
        \textbf{Forward direction: $A \setminus B \subseteq A \cap \overline{B}$ } \\ 
            Let $x \in A \setminus B$. This mean that $x \in A$ and $x \notin B$. Since $x \notin B$, it follows that $x \in \overline{B}$. This proves that $x \in A$ and $x \in \overline{B}$. This means that $x \in A \cap B$, and hence we have proved that $A \setminus B \subseteq A \cap \overline{B}$.
        
        \textbf{Backwards direction: $A \cap \overline{B} \subseteq A \setminus B$} \\
            Let $y \in A \cap \overline{B}$. This means that $y \in A$ and $y \in \overline{B}$. Since $y \in \overline{B}$, this mean that $y \notin B$. This proves that $y \in A$ and $y \notin B$. This means that $A \setminus B$, and hence wehave proved that $A \cap \overline{B} \subseteq A \setminus B$
        
        Since, we have proven both of the directions of the set equality, we have proven that $A \setminus B = A \cap \overline{B}$. 
    \end{proof}

\end{example}



\begin{example}
Source: \cite[Chap.6, S.6.1, Result 6.6]{gray} \\ 


Prove that for every integer $n \geqq 2$, the complement of the union of any $n$ sets equals the interesction of the complement of these sets
    \begin{tcolorbox}
        \begin{theorem}
            If $A_1, A_2, \cdots, A_n$ are $n \geqq 2$ sets, then 
                \begin{equation*}
                    \overline{A_1 \cup A_2 \cup \cdots A_n} = \overline{A_1} \cap \overline{A_2} \cap \cdots \overline{A_n}
                \end{equation*}
        \end{theorem}
    \end{tcolorbox}

    \begin{proof}
        We will use a proof by the First Principle of Mathematical Induction. For each natural number $n$, we let $P(n)$ be
            \begin{equation*}
                \overline{A_1 \cup A_2 \cup \cdots A_n} = \overline{A_1} \cap \overline{A_2} \cap \cdots \overline{A_n}
            \end{equation*}
        We first prove that $P(2)$ is true. Notice that by the De Morgan law that $\overline{A_1 \cup A_2} = \overline{A_1} \cap \overline{A_2}$, which proves that $P(2)$ is true. The basis step has been established 
        
        For the inductive step, we prove that for all $k \in \bb{N}$ with $k \geqq 2$, if $P(k)$, then $P(k+1)$. So let $k$ be a natural number and assume that $P(k)$ is true. That is, we assume that 
            \begin{equation*}
                \overline{A_1 \cup A_2 \cup \cdots A_k} = \overline{A_1} \cap \overline{A_2} \cap \cdots \overline{A_k}
            \end{equation*}
        
        The goal is to prove that $P(k+1)$ is true. That is, it must be proved that  
            \begin{equation*}
                \overline{A_1 \cup A_2 \cup \cdots A_{k+1}} = \overline{A_1} \cap \overline{A_2} \cap \cdots \overline{A_{k+1}}
            \end{equation*}
        
        To do this, let the set $T$ be $T = A_1 \cup A_2 \cup \cdots \cup A_n$, then
            \begin{align*}
                \overline{A_1 \cup A_2 \cup \cdots \cup A_k \cup A_{k+1}} & = \overline{[A_1 \cup A_2 \cup \cdots \cup A_k ] \cup A_{k+1}}
                    & = \overline{T \cup A_{k+1}}
                    & = \overline{T} \cap \overline{A_{k+1}}
            \end{align*}
        By the definition of set $T$ and by the inductive hypothesis we have
            \begin{equation*}
                \overline{T} = \overline{A_1} \cap \overline{A_2} \cap \cdots \cap \overline{A_n}
            \end{equation*}
        Therefore 
            \begin{align*}
                \overline{A_1 \cup A_2 \cup \cdots \cup A_k \cup A_{k+1}} & = \overline{T} \cap \overline{A_{k+1}} 
                    & = \overline{A_1} \cap \overline{A_2} \cap \cdots \cap \overline{A_n} \cap \overline{A_{k+1}} 
            \end{align*}
        
        Hence, the inductive step has been established, and by the First Principle of Mathematical Induction, we have proven that for each natural number $n \geqq 2$, if $A_1, A_2, \cdots, A_n$ are $n \geqq 2$ sets, then 
            \begin{equation*}
                \overline{A_1 \cup A_2 \cup \cdots A_n} = \overline{A_1} \cap \overline{A_2} \cap \cdots \overline{A_n}
            \end{equation*}
    \end{proof}
\end{example}





\newpage
\begin{example}
Source: \cite[Chap.6, S.6.1, Result 6.15]{gray} \\ 

Prove that if $A$ is a finite set of cardinality $n \geqq 0$, then the cardinality of its power set is $2^n$
    \begin{tcolorbox}
        \begin{theorem}
                \begin{center}
                    If $A$ is a finite set of cardinality $n \geqq 0$, then the cardinality of its power set is $\mathcal P \left({A}\right) =2^n$
                \end{center}
        \end{theorem}
    \end{tcolorbox}

    \begin{proof}
        We will use a proof by the Extended First Principle of Mathematical Induction. For each natural number $n$, we let $P(n)$ be
            \begin{center}
                If $A$ is a finite set of cardinality $n \geqq 0$, then the cardinality of its power set is $\mathcal P \left({A}\right) = 2^n$
            \end{center}
        
        We first prove that $P(0)$ is true. Notice that if $A$ is a set with the $|A| = 0$, then $A = \emptyset$. Thus cardinaltiy of its power set $\mathcal P \left({A}\right) = 2^0 = 1$, which proves that $P(0)$ is true. The basis step has been established 
        
        For the inductive step, we prove that for all $k \in \bb{N}$ with $k \geqq 0$, if $P(k)$, then $P(k+1)$. So let $k$ be a natural number and assume that $P(k)$ is true. That is, we assume that 
            \begin{center}
                If $A$ is a finite set of cardinality $k \geqq 0$, then the cardinality of its power set is $\mathcal P \left({A}\right) =2^k$
            \end{center}
        
        The goal is to prove that $P(k+1)$ is true. That is, it must be proved that  
            \begin{center}
                If $A$ is a finite set of cardinality $k+1 \geqq 0$, then the cardinality of its power set is $\mathcal P \left({A}\right) =2^{k+1}$
            \end{center}
        
        Let $A_1 = \{a_1, a_2, \cdots , a_k, a_{k+1} \}$. By the inductive hypothesis, there are $2^k$ subsets of the set $\{a_1, a_2, \cdots , a_k \}0$ that is, there are $2^k$ subsets of the set $A_1$ not containing the element $a_{k+1}$. Now subset of the set $A_1$ that contains the $a_{k+1}$ element can be expressed as 
            \begin{equation*}
                A_1 = A_2 \cup \{a_{k+1} \} \text{, where } A_2 = \{a_1, a_2, \cdots , a_k  \}
            \end{equation*}
        
        Again, using the inductive hypothesis, there are $2^k$ such subsets the set $A_2$. Therefore there are 
            \begin{equation*}
                2^k + 2^k = 2 \m 2^k = 2^{k+1}
            \end{equation*}
        subsets of $A_1$
        
        Hence, the inductive step has been established, and by the extended First Principle of Mathematical Induction, we have proven that if $A$ is a finite set of cardinality $n \geqq 0$, then the cardinality of its power set is $\mathcal P \left({A}\right) =2^n$

    \end{proof}
\end{example}







\newpage
\chapter{EQUIVALENCE RELATIONS \& PARTIAL ORDERINGS}

\section{Relations}

\subsection{Definition of a Relation}

    The concept of relations from a set $A$ to a set $B$ is based on the concept of ordered pair $(x,y)$ and, more specifically, the idea of the Cartesian product $A \times B$ of two sets $A$ and $B$. We recall here the criteria for equality of ordered pairs and for membership of an object in the Cartesian Product of two sets. 
    
        \begin{itemize}
            \item We say that two ordered pairs of objects $(a,b)$ and $(y,z)$ are \textbf{equal}, denoted
            $(a,b)=(y,z) \equiv (a=y) \cap (b=z)$
            \item Given sets $A$ and $B$, we say that an object $x$ is an element of the Cartesian product $A \times B$ if and only if there exists $a \in A$ and $b \in B$ such that $x = (a,b)$
        \end{itemize}
    
    
    \begin{definition}
    Relation \\   
        \begin{tcolorbox}
            Let $A$ and $B$ be sets. A \textbf{relations} from $A \mapsto B$ is any subset $\cc{R}$ of $A \times B$. \\
                \begin{center}
                    If $A = B$, we say that $R$ is a relation on $A$
                \end{center}
        \end{tcolorbox}
    \end{definition}
    
    \begin{remark}
        Since a relation is among other things a set, specific relations may be described by either a rooster method or the the rule method. As was in the case of describing general sets, a rule describing a relation must have the property that, given an ordered pair $(x,y)$, we must be able to determine, by the rule whether or not $(x,y)$ lie in the relation
    \end{remark}
    
    \begin{example}
    Let $\{ 1,2,3 \}$ and $B = \{w,x,y,z \}$. \\
    
    Then $\cc{R}_1 = \{ (1,x), (2,y), (3,z) \}$ and $\cc{R}_2 = \{ (2,w), (2,x), (2,z) \}$ are relations from $A \mapsto B$. While the sets $\cc{R}_3 = \{ (w,1), (w,3) \}$ and $\cc{R}_4 = \{ (w,1), (x,2), (z,1) \}$ are relations from $B \mapsto A$. \\
    
    The entire set $A \times B$ itself is a relation from $A \mapsto B$, as is the empty set $\emptyset$. The relation $R_2$ may be described by the rule method, namely
        \begin{equation*}
            \cc{R}_2 = \{ (2,b) | b \in B \} \text{ or } \cc{R}_2 = \{ (a,b) | a=2 \wedge b \in B \}
        \end{equation*}
    \end{example}
    
    
    \begin{remark}
        A common way of viewing a relation is from a dynamic, rather than a static, point of view. Frequently, we think of a relation not primarily as a set of ordered pairs, but rather, as a relationship where the relationship exists between precisely those pairs of objects that occur together in an ordered pair contained in the relation.
    \end{remark}
    
    

\newpage
\subsection{Properties of Relations}

    The following four properties a relation may or many not possess. These properites are fundamental to the definition of equivalence relation and partial ordering. 
    
    \begin{definition}
        Let $A$ be a set and $\cc{R}$ a relation on $A$. We say that \footnotemark,
            \begin{itemize}
                \item $\cc{R}$ is \textbf{reflexive on $A$} if and only if $x \cc{R} x$, for all $x \in A$
                    \begin{equation*}
                        (\forall x \in A)[(x,x) \in \cc{R}]
                    \end{equation*}
                    
                    \item $\cc{R}$ is \textbf{symmetric} if and only if, for every $x$, $y$ and $z$, if $x \cc{R} y$ then $y \cc{R} x$
                    \begin{equation*}
                        (\forall x \in A)(\forall y \in A)[(x,y) \in \cc{R} \to (y,x) \in \cc{R}]
                    \end{equation*}
                    
                    \item $\cc{R}$ is \textbf{transitive} if and only if, for every $x$, $y$ and $z$ in $A$, if $x \cc{R} y$ and $y \cc{R} z$, then $x \cc{R} z$
                    \begin{equation*}
                        (\forall x \in A)(\forall y \in A)(\forall z \in A)[(x \cc{R} y \wedge y \cc{R} z) \to x \cc{R}z]
                    \end{equation*}
                    
                    \item $\cc{R}$ is \textbf{antisymmetric} if and only if, for every $x$ and $y$ in $A$, if $x \cc{R} y$ and $y \cc{R} x$, then $x=y$  
                        \begin{equation*}
                            (\forall x \in A)(\forall y \in A)[(x \cc{R} y \wedge y \cc{R} x) \to x=y]
                        \end{equation*}
                    
                    \item \textbf{Power Set of a relation} $\cc{P}(A \times B)$ is the set of all relations from sets $A \mapsto B$ 
            \end{itemize}
        \footnotetext{From the point of view of a relation $\cc{R}$ as a set of ordered pairs,
            \begin{itemize}
                \item \textbf{Reflexive Property} means that $\cc{R}$ contains all pairs $(x,x)$ were $x$ ranges over all the elements of $A$. Dynamically, the reflexive property mean \textit{every element of $A$ is related to itself}
                
                \item \textbf{Symmetric Property} means that whenever we "\textit{flip over}" an ordered pair in $\cc{R}$, the resulting ordered pair is also in $\cc{R}$. 
                
                \item \textbf{Antisymmetric Property} means the ordered pair $(y,x)$ we get from flipping over an ordered pair $(x,y)$ in $\cc{R}$ is never in $\cc{R}$, unless $x = y$.
                
                \item \textbf{Transitive Property} may be regarded as a property by which ordered pairs in relation are "\textit{linked together}" to form new ordered pairs in the relation.
            \end{itemize}}
    \end{definition}
    
\newpage
\begin{example}
    Consider the relation \textit{less than} on $\cc{R}$; that is, a pair $(x,y)$ is an element of this relation if and only if $x < y$. 
        \begin{itemize}
            \item This relation is \textit{not reflexive}, since, for  instance, it is false that $5 < 5$. 
            
            \item It is \textit{not symmetric} since, for example $8 < 9$, whereas it is not the case that $9 < 8$. 
            
            \item The relation clearly is \textit{transitive} since $x < y$ and $y < z$, then $x < z$ for any real number $x$, $y$, and $z$.
            
            \item The relation is also \textit{antisymmetric}, although only by means of a logical technicality. The question is whether the statement "for all $x,y \in \cc{R}$, if $x < y$ an $y < x$, then $x=y$" is true. The answer is "yes" since the premise $x < y$ and $y < x$ is false for all $x$ and $y$
        \end{itemize}
\end{example}


\newpage
\subsection{Domain, Range, and Inverse of a Relations}   
    
    \begin{definition} 
        Domain and Range of $\cc{R}$ \\
        
        \begin{tcolorbox}      
            Let $\cc{R}$ be a relation from a set $A$ to a set $B$, i.e. $\cc{R} \subseteq A \times B$. \\
            We define the \textbf{domain of $\cc{R}$} to be the set 
                \begin{equation*}
                    \text{dom } \cc{R} = \{ x \in A \:|\: x \cc{R} y, \text{ for some } y \in B \}
                \end{equation*}
            and the \textbf{range of $\cc{R}$} by the rule 
                \begin{equation*}
                    \text{rng } \cc{R} = \{ y \in B \:|\: x \cc{R} y, \text{ for some } y \in A \}
                \end{equation*}
        \end{tcolorbox}
    \end{definition}

    
    Clearly whenever $\cc{R}$ is a relation from $A$ to $B$, the domain of $\cc{R}$ is subset of $A$, $\text{dom } \cc{R} \subseteq A$ and the range of $\cc{R}$ is a subset of $B$, $\text{rng } \cc{R} \subseteq\ B$. In fact, $\text{dom} \cc{R}$ consists of those elements of $A$ that are first elements of ordered pairs in $R$, where as $\text{rng } \cc{R}$ comprises te elements of $B$ that are second elements of ordered pairs in $\cc{R}$. 

    \begin{example}
        Let $\{ 1,2,3 \}$ and $B = \{w,x,y,z \}$. \\
        
        Consider the follow relations: 
            \begin{enumerate}
                \item For $\cc{R}_1 = \{ (1,x), (2,y), (3,z) \}$, $\text{dom } \cc{R}_1 = \{1, 2, 3 \} = A$ and $\text{rng } \cc{R}_1 = \{x, y, z \} \subseteq B$
                
                \item For $\cc{R}_2 = \{ (2,w), (2,x), (2,z) \}$, $\text{dom } \cc{R}_2 = \{2 \} \subseteq A$ and $\text{rng } \cc{R}_2 = \{w, x, y, z \} = B$
                
                \item For $\cc{R}_3 = \{ (w,1), (w,3) \}$, $\text{dom } \cc{R}_3 = \{w \} \nsubseteq B$ and $\text{rng } \cc{R}_3 = \{1, 3 \} \subseteq A$ 
                
                \item For $\cc{R}_4 = \{ (w,1), (x,2), (z,1) \}$, $\text{dom } \cc{R}_4 = \{w, x, z \} \subseteq B$ and $\text{rng } \cc{R}_4 = \{1 \} \subseteq A$
            \end{enumerate}
                
    \end{example}

    \newpage
    \begin{definition} 
        Inverse of $\cc{R}$ \\
        
        \begin{tcolorbox}      
            Let $\cc{R}$ be a relation from a set $A$ to a set $B$, i.e. $\cc{R} \subseteq A \times B$. \\
            We define the relation \textbf{$\cc{R}$ inverse}, denoted by $\cc{R}^{-1}$, by the rule
                \begin{equation*}
                    \cc{R}^{-1} = \{ (y,x) \:|\: (x,y) \in \cc{R} \}
                \end{equation*}
        \end{tcolorbox} 
    \end{definition}

    Clearly $\cc{R}^{-1}$ is a relation from set $B$ to $A$; it is gotten from $\cc{R}$ by flipping over all the ordered pairs in $\cc{R}$. 

   \begin{theorem} 
        Properties of the Inverse of $\cc{R}$ \\
        
        \begin{tcolorbox}      
            Let $\cc{R}$ be a nonempty relation from $A$ to $B$. then: 
                \begin{enumerate}
                    \item $\text{dom } \cc{R}^{-1} = \text{rng } \cc{R}$
                    
                    \item $\text{rng } \cc{R}^{-1} = \text{dom } \cc{R}$
                    
                    \item $\cc{R} = (\cc{R}^{-1})^{-1}$
                    
                    \item $\cc{R} = \cc{R}^{-1}$ if and only if $A=B$ and $\cc{R}$ is symmetric
                    
                    \item $\cc{R} \cap \cc{R}^{-1} = I_n$ if and only if $A=B$ and $\cc{R}$ is antisymmetric
                \end{enumerate}
        \end{tcolorbox} 
    \end{theorem}


\newpage
\section{Equivalence Relations and Classes}

\subsection{Definition of Equivalence Relations}

    A equivalence relation is a special kind of relation a set; whenever two elements $x$ and $y$ of a set $A$ are related by an equivalence relation on $A$, there is some property that $x$ and $y$ share in common, some point of view from which $x$ and $y$ can be regarede as indistinguishable. 

    \begin{definition}
        Equivalence Relation \\
        
        \begin{tcolorbox}
            Let $A$ be a set and $\cc{E}$ is \textbf{equivalence relation on $A$} if and only $\cc{E}$ is reflexive on $A$, symmetric and transitive. (RST)
        \end{tcolorbox}
    \end{definition}

    \begin{remark}
        Equivalence relations on a set tend to express some measure of "same-ness" among the elements of the set, whether it is truee equality or something weaker (like having the same parity)
    \end{remark}
    
    
    
    \begin{example}
    Equality is an Equivalence Relation \\
    
        \begin{center}
            $\cc{E} = \{ (x,y) | x=y\}$, also denoted $I_A = \{ (x,y) | x=y\}$
        \end{center}
    
        To show that equality is an equivalence relation, we need to confirm that the relation $\cc{E}$ is reflexive on $A$, symmetric and transitive. \\
        
            \textbf{Test 1: Reflexivity } \\
                The reflexive property simply requires that every element of $A$ equal itself, a true statement \\
            \textbf{Test 1: Symmetry } \\
            \   The symmetry property states that if $x=y$, then $y=x$, for any $x,y \in A$, a true statement \\
            \textbf{Test 1: Transitivity } \\
                The transitivity property states that if $x=y$ and $y=z$, then $x=z$ for any element $x$, $y$, and $z$ of $A$, a true statement \\
                
        Since equality adheres to all of the properties, we conclude that equality is an equivalence relation. 
    \end{example}
    
    \begin{remark}
        At the other extreme, the whole Cartesian Product $A \times A$ is an equivalence relation on any set $A$. This is uninteresting fact since any two elements of $A$ are related by to; certainly we will not often want to view all elements of a set $A$ as indistinguishable from one another!\textbf{}
    \end{remark}
    
    
    \begin{example}
    Proof and Disproof of Equivalence Relation \\
        \begin{tcolorbox}
            Show that the relation $\cc{E}$ on $A = \bb{R} - \{0 \}$ defined by $\cc{E} = \{ (x,y) | xy > 0 \}$ is an equivalence relation on $A$, whereas the relation $\cc{N} = \{ (x,y) | xy < 0 \}$ satisfies only symmetry
        \end{tcolorbox}
        
            
        Firstly, \\
            \begin{equation*}
                \cc{E} = \{ (x,y) | xy > 0 \}
            \end{equation*}
        
        To show that $\cc{E}$ is an equivalence relation, we need to confirm that the relation $\cc{E}$ is reflexive on $E$, symmetric and transitive. \\

            \textbf{Test 1: Reflexivity } \\
                Let $x \in A$ \\
                Notice that $(x,x \in E)$ implies that $x^2 > 0$. Hence, the relation $\cc{E}$ is reflexive
            
            \textbf{Test 1: Symmetry } \\
                Suppose $x,y \in \bb{R}$ and $x \cc{E} y$, so thta $xy >0$. Since $xy = yx$, we conclude that $xy > 0$. Hence, the relation $\cc{E}$ is symmetric \\
            
            \textbf{Test 1: Transitivity } \\
                 Suppose $x,y,z \in \bb{R}$, such that if $x \cc{E} y$ and $y \cc{E} x$, then $x \cc{E} z$. It follows that we will have to examine the following proposition to test $\cc{C}$ for transitivity:
                    \begin{center}
                        If $xy > 0$ and $yz > 0$, then $xz > 0$
                    \end{center}
                We will proceed by divison into the cases $x>0$ and $x<0$ to prove the propostiion above. \\
                
                \textbf{Case 1: $x>0$} \\
                    We assume that $x>0$, $xy > 0$, and $yz >0$. \\
                    Since $x >0 $ and $xy > 0$, it follows that $y>0$. Using the fact that $y>0$ and $yz > 0$, we know that $z > 0$. \\
                    Since $x > 0$ and $z > 0$, we conclude that $xz > 0$. \\
                
                \textbf{Case 2: $x<0$} \\
                    We assume that $x < 0$, $xy > 0$, and $yz >0$. \\
                    Since $x < 0 $ and $xy > 0$, it follows that $y < 0$. Usinf the fact that $y < 0$ and $yz > 0$,we know that $z < 0$. \\
                    Since $x < 0$ and $z < 0$, we conclude that $xz > 0$. \\
                
                Since we proved that in both cases $xz > 0$, we conclude that the realtion $\cc{E}$ is transitive. \\
                
            Given that the relation $\cc{E}$ adhere to all three properties, we conclude that $\cc{E}$ is an equivalence relation. \\
            
            
        Secondly, \\
            \begin{equation*}
                \cc{N} = \{ (x,y) | xy < 0 \}
            \end{equation*}
            
        To show that $\cc{N}$ is an equivalence relation, we need to confirm that the relation $\cc{N}$ is reflexive on $E$, symmetric and transitive. \\
        
            \textbf{Test 1: Reflexivity } \\
                We will proceedvia a counterexample to show that $\cc{N}$ is not reflexive. Using the artrarly element $5 \in A$. We see hat $5 \m 5 > 0$, so $(5,5) \notin \cc{N}$. We conclude that $\cc{N}$ is not reflexive  \\

            \textbf{Test 1: Symmetry } \\
                Suppose $x,y \in \bb{R}$ and $x \cc{N} y$, so thta $xy >0$. Since $xy = yx$, we conclude that $xy > 0$. Hence, the relation $\cc{N}$ is symmetric \\
            
            \textbf{Test 1: Transitivity } \\
                 Suppose $x,y,z \in \bb{R}$, such that if $x \cc{E} y$ and $y \cc{E} x$, then $x \cc{E} z$. It follows that we will have to examine the folowing propositon to test $\cc{C}$ for transitivity:
                    \begin{center}
                        If $xy > 0$ and $yz > 0$, then $xz > 0$
                    \end{center}
                
                We will proceed via a counter example to show that $\cc{N}$ is not transitive. Consider the arbitrary elements $x = 5 \in A$, $y = - 5 \in A$ and $z = 7 \in A$. Since $(5,-5) \in \cc{N}$ and $(-5, 7) \in \cc{N}$, but $(5,7) \notin \cc{N}$, we conclude that $\cc{N}$ is not transitive. \\
                 
            Given that the relation $\cc{N}$ fails to adhere to all three properties, we conclude that $\cc{N}$ is not an equivalence relation. \\
    \end{example}


    \newpage
    \begin{example}
        Relation involving Congruence \\
        
        \begin{tcolorbox}
            \begin{theorem}
                Let $n \in \bb{N}$. The relation $\equiv (\text{ mod } n)$ on the set $\bb{Z}$ is reflexive, symmetric, and transitive.
            \end{theorem}
        \end{tcolorbox}
    
    
        \begin{proof}
             To show that $\equiv (\text{ mod } n)$ is an equivalence relation, we need to confirm that the relation is reflexive, symmetric and transitive. \\
             
            \textbf{Test 1: Reflexivity } \\    
                Using a direct proof, take any integer $x \in \bb{Z}$, and observe that $n|0$, so $n|(x-x)$. By definition of congruence modulo $n$, we have $x \equiv x (\text{ mod } n)$. This shows $x \equiv x (\text{ mod } n)$ for every $x \in \bb{Z}$, so we conclude that $\equiv (\text{ mod } n)$ is reflexive \\
    
            \textbf{Test 2: Symmetry } \\
                For this, we must show for all $x$, $y \in \bb{Z}$, the condition $x \equiv y (\text{ mod } n)$ implies that $y \equiv x (\text{ mod } n)$. Using a direct proof. \\
                Suppose $x \equiv y (\text{ mod } n)$. Thus $n |(x-y)$, by the definition of congruence modulo $n$. Then, 
                    \begin{equation*}
                        x-y = na         
                    \end{equation*}
                for some $a \in \bb{Z}$, by the definition of divisibility. Multiplying both sides by $-1$ gives
                    \begin{equation*}
                        y-x = n(-a)
                    \end{equation*}
                Therefore, $n |(y-x)$, and this means $y \equiv x (\text{ mod } n)$. We've shown that $x \equiv y (\text{ mod } n)$ implies that $y \equiv x (\text{ mod } n)$, and this means $\equiv (\text{ mod } n)$ is symmetric. \\
            
            \textbf{Test 3: Transitivity } \\    
                For this, we must show that if $x \equiv y (\text{ mod } n)$ and $y \equiv z (\text{ mod } n)$, then $x \equiv z (\text{ mod } n)$. Using a direct proof. \\
                
                Suppose $x \equiv y (\text{ mod } n)$ and $y \equiv z (\text{ mod } n)$. This means $n | (x-y)$ and $n | (y-z)$. Therefore, there are integers $a$ and $b$ for which 
                    \begin{equation*}
                        x-y=na        
                    \end{equation*}
                and 
                    \begin{equation*}
                        y-z=nb
                    \end{equation*}
                Adding these two equations, we obtain 
                    \begin{align*}
                        x-y + y-z   & = na + nb
                        x - z       & = n(a+b) 
                    \end{align*}
                So, $n |(x-z)$, hence $x \equiv z (\text{ mod } n)$. This completes the proof that  $\equiv (\text{ mod } n)$ is transitive. \\
                
        Given that the relation $\equiv (\text{ mod } n)$ adheres to all three properties, we conclude that $\equiv (\text{ mod } n)$ is an equivalence relation.
        \end{proof}
    \end{example}




    \newpage
    \begin{example}
        Relation involving Congruence \\
        
        \begin{tcolorbox}
            Show that the relation "congruence modulo $5$" is an equivalence relation on $\bb{Z}$, where
                \begin{equation*}
                    \cc{R} = \{ m,n \in \bb{Z} | m \equiv n \text{ mod } 5 \}
                \end{equation*}
            
        \end{tcolorbox}
    
    To show that $\cc{R}$ is an equivalence relation, we need to confirm that the relation $\cc{R}$ is reflexive on $m$, symmetric and transitive. \\
            
            \textbf{Test 1: Reflexivity } \\    
                If $m \in \bb{Z}$, then $m=m \text{ mod } 5$, since $m-m = 0$ and $5 | 0$, we conclude that $\cc{R}$ is reflexive \\
    
            \textbf{Test 1: Symmetry } \\
                Assume $m,n \in \bb{Z}$ and $m \equiv n \text{ mod } 5$; thus $5$ divides $m-n$. Since $n-m = -(m-n)$, and since $5$ divides $n-m$, so that $m \equiv n \text{ mod } 5$, we conclude that $\cc{R}$ is symmetric \\
            
            \textbf{Test 1: Transitivity } \\    
                Suppose $m, n, p \in \bb{Z}$, that $5 | (m-n)$ and $5 | (n-p)$. We must prove that $5 | (m-p)$. Since $m-p = (m-n) + (n-p)$, we conclude that $\cc{R}$ is transitive \\
        
        Given that the relation $\cc{R}$ adheres to all three properties, we conclude that $\cc{R}$ is an equivalence relation
    
    \end{example}

    \begin{remark}
        The last example illustrates that two integers $m$ and $n$ are congruent modulo $5$ if and only if both yield the same remainder upon division by $5$, in accordance with the conditions of the division algorithm for $\bb{Z}$
    \end{remark}


\newpage
\subsection{Definition of Equivalence Class}

    \begin{remark}
        It's time to introduce an important definition. Whenever you have an equivalence relation $\cc{R}$ on a set $A$, it divides $A$ into subsets called \textit{equivalence classes}
    \end{remark}
    
    \begin{tcolorbox}
        \begin{definition}
            Equivalence Class \\
            
            Suppose $\cc{R}$ is an equivalence relation on $A$. Given any element $a \in A$, the \textbf{equivalence class} containing $a$ is the subset $\{ x \in A | x \cc{R} a$ of $A$ consisting of all the elements of $A$ that relate to $a$. This set is denoted $[a]$. Thus, the equivalence class containing $a$ is the set $[a] = \{ x \in A | x \cc{R} a$
        \end{definition}
    \end{tcolorbox}
    
    \begin{example}
        Let $\cc{P}$ be the set of all polynomials with real coefficients. Define $\cc{R}$ on $\cc{P}$ as follows: \\
        
        Given $f(x), g(x) \in \cc{P}$, let $f(x) \cc{R} g(x)$ mean that $f(x)$ and $g(x)$ have the same degree. Thus, $(x^2 + 3x - 4) \cc{R} (3x^2 - 2)$ and $(x^3 + 3x - 4), (3x^2 - 2) \notin \cc{R}$. For example, an equivalence classes is: 
            \begin{align*}
                [x^3 + 3x - 4] = \{ ax^2 + bx + c | c \in \bb{R}, a \neq 0 \}
            \end{align*}
        
    \end{example}










\newpage
\chapter{TOPICS IN FUNCTIONS AND MAPPING}
\section{Functions and Mapping}
    
    \begin{definition}
    Function \\
    
        \begin{tcolorbox}
            A \textbf{function} is a relation $R$ that have the property that if $(x,y \in R)$ and $(x,z \in R )$, then $y=z$. 
        \end{tcolorbox}
    \end{definition}














\end{document}
