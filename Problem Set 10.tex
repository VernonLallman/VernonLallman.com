\documentclass{book}

\usepackage[utf8]{inputenc}
\usepackage[english]{babel}
 
\usepackage{biblatex}
\usepackage{csquotes}
\addbibresource{references.bib}

\usepackage{amsfonts}
\usepackage{amssymb}
\usepackage{amsmath}
\usepackage{amsthm}

\usepackage{tcolorbox}
\usepackage{graphicx}
% \graphicspath{ {C:\Users\Logos Wealth\Desktop\Genenseo\0. Latex Images\ } }


\theoremstyle{definition}

\newtheorem{theorem}{Theorem}[section]
\newtheorem{corollary}{corollary}[theorem]
\newtheorem{lemma}[theorem]{Lemma}
\newtheorem{definition}[theorem]{Definition}
\newtheorem{example}[theorem]{Example}
\newtheorem{conjecture}[theorem]{Conjecture}



\makeatletter
\newcommand{\vo}{\vec{o}\@ifnextchar{^}{\,}{}}
\makeatother

 \title{\textsc{MATHEMATICAL PROOFS AND STRUCTURES}\\ {\bf Transition to Advanced Mathematics}\\ Notebook and Solved Exercises}
 \author{Vernon V. Lallman}
 \date{\today}

\begin{document}
 \maketitle

\tableofcontents



% \begin{appendix}
% 	\listoffigures
%	\listoftables
% \end{appendix}


\newpage
VERNON V. LALLMAN
\section{Problem Set 10 - Proving Set Relationships}
\date{\today}

\subsection{Problem 1}
Is the following proposition true or false? 
    \begin{center}
        For all sets $A$ and $B$ are the subsets of some universal set $U$, the sets $A \cap B$ and $A \setminus B$ are disjoint
    \end{center}

\begin{tcolorbox}
	\begin{theorem}
		Let $A$ and $B$ be subsets of some universal set $U$. Then $(A \cap B) \cap (A \setminus B)$ is disjoint 
	\end{theorem}
\end{tcolorbox}

\begin{proof}
    Let $A$ and $B$ be subsets of some universal set. We will proceed via direct proof to show that $(A \cap B) \cap (A \setminus B)$ is disjoint. Firstly, let $x = \emptyset$ and $y \in (A \cap B) \cap (A \setminus B)$. Using the properties of set difference, we get
        \begin{align*}
            \emptyset & = y \in (A \cap B) \cap (A \setminus B) \\
                & = y \in (A \cap B) \cap (A \cap \overline{B}) \\
        \end{align*}
    Next, the associative law of intersection to rearrange the equation as follows
            \begin{align*}
            \emptyset & = y \in (A \cap A) \cap (B \cap \overline{B}) \\
        \end{align*}
    Next, using the properties of the empty set, yields
        \begin{align*}
            \emptyset & = y \in (A \cap A) \cap \emptyset \\
        \end{align*}
    Next using the idempotent law of intersection, we get
        \begin{align*}
            \emptyset & = y \in A \cap \emptyset \\
        \end{align*}
    Again, using the properties of the empty set, we see that 
        \begin{center}
            $x = \emptyset$ and $y = \emptyset$
        \end{center}
    Since $x = y = \emptyset$, we have proven that $(A \cap B) \cap (A \setminus B)$ is disjoint.  
\end{proof}


\newpage
\subsection{Problem 2}
Prove or disprove
    \begin{center}
        $A \setminus (A \cap \overline{B}) = A \cap B$
    \end{center}

\begin{tcolorbox}
	\begin{theorem}
		Let $A$ and $B$ be subsets of some universal set $U$. Then $A \setminus (A \cap \overline{B}) = A \cap B$
	\end{theorem}
\end{tcolorbox}

\begin{proof}
    Let $A$ and $B$ be subsets of some universal set. We will proceed via direct proof to show that $A \setminus (A \cap \overline{B}) =  A \cap B$. Firstly, let $x \in A \cap B$ and $y \in A \setminus (A \cap \overline{B})$ \\
    
    Using the properties of set difference, we get
        \begin{align*}
            x \in A \cap B & =  y \in A \setminus (A \cap \overline{B}) \\
                & = A \cap \overline{(A \cap \overline{B})} \\
        \end{align*}
    Next, using DeMorgan's law and properties of set complement, we get
        \begin{align*}
             x \in A \cap B & = y \in A \cap (\overline{A} \cup \overline{\overline{B}}) \\
                & = y \in A \cap (\overline{A} \cup B) \\
        \end{align*}    
    Next, using the distributive laws for intersection over union, we get
        \begin{align*}   
            x \in A \cap B & = y \in (A \cap \overline{A}) \cup (A \cap B) \\
        \end{align*}
    Now, using the identity property of union set operation yields,
        \begin{align*}   
            x \in A \cap B & = y \in \emptyset \cup (A \cap B) \\
            & = y \in A \cap B 
        \end{align*}
    Since $x \in A \cap B$ and $y \in A \cap B$, we have proven that $A \setminus (A \cap \overline{B}) =  A \cap B$    
\end{proof}


\newpage
\subsection{Problem 3}
Prove or disprove the  following conjecture
    \begin{center}
        $A \setminus (B \cup C) = (A \setminus B) \cup (A \setminus C)$
    \end{center}

\begin{tcolorbox}
	\begin{theorem}
		Let $A$, $B$ and $C$ be subsets of some universal set $U$. Then $A \setminus (B \cup C) = (A \setminus B) \cup (A \setminus C)$
	\end{theorem}
\end{tcolorbox}

\begin{proof}
    Let $A$, $B$ and $C$ be subsets of some universal set. We will prove that $ A \setminus (B \cup C) = (A \setminus B) \cup (A \setminus C)$ by proving that  $A \setminus (B \cup C) \subseteq (A \setminus B) \cup (A \setminus C)$ and $(A \setminus B) \cup (A \setminus C) \subseteq A \setminus (B \cup C) $.
    
    First, let $x \in A \setminus (B \cup C)$. This means that 
        \begin{center}
            $x \in A$ and $x \notin (B \cup C)$
        \end{center}
    We know that an element is in $(B \cup C)$ if and only if it is in either in $B$ or $C$. Since $x \notin (B \cup C)$, we conclude that $x \notin B$ or $x \notin C$. However, we also know that  $x \in A$ and so we conclude that either $x \in A$ and $x \notin B$ or $x \in A$ and $x \notin C$. This proves that 
        \begin{center}
            $x \in (A \setminus B)$ or  $x \in (A \setminus C)$
        \end{center}
    This means that $x \in (A \setminus B) \cup (A \setminus C)$, and hence we have proved that $ A \setminus (B \cup C) = (A \setminus B) \cup (A \setminus C)$ \\
    
    Now, choose $y \in (A \setminus B) \cup (A \setminus C)$. This means that
        \begin{center}
            $y \in (A \setminus B)$ or  $y \in (A \setminus C)$         
        \end{center}
    We note that $y \in (B \cup C)$ if and only if either $y \in B$ or $y \in C$ and hence $y \notin (B \cup C)$ if and only if either $y \notin B$ or $y \notin C$. Since we have proved that $y \in C$, we conclude that $y \notin (B \cup C)$, and hence we have established that $y \in A$ and $y \notin (B \cup C)$. This proves that if $y \in (A \setminus B) \cup (A \setminus C)$, then $y \in A \setminus (B \cup C)$ and hence
        \begin{center}
            $(A \setminus B) \cup (A \setminus C) \subseteq A \setminus (B \cup C) $
        \end{center}
    Since we have proved that  $A \setminus (B \cup C) \subseteq (A \setminus B) \cup (A \setminus C)$ and $(A \setminus B) \cup (A \setminus C) \subseteq A \setminus (B \cup C) $., we conclude that  $A \setminus (B \cup C) = (A \setminus B) \cup (A \setminus C)$
   
\end{proof}



\newpage
\subsection{Problem 4}
Prove that $f(n) = 2n - 1$ for all natural numbers $n$ 

\begin{tcolorbox}
    \begin{theorem}
        If sequence $\{ a_n \}$ is defined recursively by 
            \begin{itemize}
                \item $a_1 = 1$
                \item $a_2 = 3$
                \item $a_n = 2a_{n-1} - a_{n-2}$, for $n \geqq 3$
            \end{itemize}
        Then, $a_n = 2n - 1$, for all natural numbers $n$

    \end{theorem}
\end{tcolorbox}

\begin{proof}
        We will use a proof by the Second Principle of Mathematical Induction. For each natural number $n$, we let $P(n)$ be sequence $\{a_n \}$ defined recursively by $a_1 = 1$, $a_2 = 3$ and $a_n = 2a_{n-1} - a_{n-2}$, for $n \geqq 3$, then for all natural numbers $n$
            \begin{equation*}
                a_n = 2n - 1
            \end{equation*}
        
        We first prove that $P(1)$. Notice that $a_1 = (2 \cdot 1) - 1 = 1$, which proves that $P(1)$ is true. This provides us the basis step for induction. \\ 
        
        For the inductive step, we prove that for all $k \in \mathbb{N}$ with $k \geqq 1$, we assume that $P(1), P(2), P(3), \cdots, P(k)$ are true. That is, we assume that each of the natural number $k$, the sequence $\{ a_k \}$ defined recursively by $a_1 = 1$, $a_2 = 3$ and $a_k = 2a_{k-1} - a_{k-2}$, for $n \geqq 3$, then
            \begin{equation*}
                a_k = 2k - 1
            \end{equation*}        
        
        The goal is to prove that $P(k+1)$ is true. That is, it must be proved that the closed form of the sequence $\{ a_{k+1} \}$ is  
            \begin{align*}
                a_{k+1} & = 2(k+1) - 1 \\
                & = 2k + 1 \\
            \end{align*} 
        
        To do this, notice that if $k=1$, then $a_{k+1} = a_2 = (2 \cdot 1) + 1 = 3$. Since $a_2 = 3$, it follows that $a_{k+1} = 2k + 1$ when $k=1$. Hence, we may assume that $k \geqq 2$. Since $k+1 \geqq 3$, it follows that 
            \begin{align*}
                a_{k+1} & = 2a_{(k+1) - 1} - a_{(k+1) - 2} \\
                    & = 2a_k - a_{k-1} \\
                    & = 2(2k - 1) - (2k - 3) \\ 
                    & = 2k + 1 \\
                    & = 2(k+1) - 1 \\
            \end{align*}
        Hence, the inductive step has been established, by the Second Principle of Mathematical Induction, we conclude the sequence $\{a_n \}$ defined recursively by $a_1 = 1$, $a_2 = 3$ and $a_n = 2a_{n-1} - a_{n-2}$, for $n \geqq 3$, then for all natural numbers $n$
            \begin{equation*}
                a_n = 2n - 1
            \end{equation*}
\end{proof}




\end{document}